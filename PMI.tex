\documentclass{article}
\usepackage{cmap}
\usepackage[T1,T2A]{fontenc}
\usepackage[utf8]{inputenc}
\usepackage[russian]{babel}
\usepackage[left=2cm,right=2cm,top=2cm,bottom=2cm,bindingoffset=0cm]{geometry}
\usepackage{tikz}
\usepackage{setspace,amsmath}
\usepackage{tabularx}
\usepackage{multirow}
\usepackage{makecell}
\usepackage{listings}
\usepackage{titlesec}
\usepackage{lipsum}
\usepackage[usestackEOL]{stackengine}
\usepackage{kantlipsum}
\usepackage{caption}
\usepackage{float}
\usepackage{zref-totpages}
\usepackage{fancyhdr}
\usepackage{graphicx}
\usepackage{hyperref}
\pagestyle{fancy}
\fancyhf{}
\fancyhead[C]{\thepage\\ RU.17701729.10.03-01 ПМИ 01-1}
\renewcommand{\headrulewidth}{0pt}
\captionsetup[table]{justification=centering}
\usetikzlibrary{positioning}
\graphicspath{{pictures/}}
\DeclareGraphicsExtensions{.pdf,.png,.jpg}
\newcommand\zz[1]{\par{\normalsize\strut #1} \hfill\ignorespaces}
\addto\captionsrussian{\def\refname{}}
\newcommand{\subtitle}[1]{%
    \posttitle{%
        \par\end{center}
        \begin{center}\Large#1\end{center}
    }%
}
\newcommand{\subsubtitle}[1]{%
    \preauthor{%
        \begin{center}
        \large #1 \vskip0.5em
        \begin{tabular}[t]{c}
    }%
}
\newcommand{\mytable}{
    \AddToShipoutPictureBG{%
        \begin{tikzpicture}[remember picture, overlay]
        \node[anchor=south west, xshift=1cm, yshift=1cm] at (current page.south west) {
            \begin{tabular}{|>{\raggedright\arraybackslash}m{0.3cm}|>{\centering\arraybackslash}m{0.3cm}|}
            \hline
            \rotatebox{90}{\textit{\textbf{Инв. № подл. }}} & \rotatebox{90}{} \\
            \hline
            \rotatebox{90}{\textit{\textbf{Подп. и дата }}} & \rotatebox{90}{} \\
            \hline
            \rotatebox{90}{\textit{\textbf{Взам. инв. № }}} & \rotatebox{90}{} \\
            \hline
            \rotatebox{90}{\textit{\textbf{Инв. № дубл. }}} & \rotatebox{90}{} \\
            \hline
            \rotatebox{90}{\textit{\textbf{Подп. и дата }}} & \rotatebox{90}{} \\
            \hline
            \end{tabular}
        };
        \end{tikzpicture}
    }
}
\begin{document}
    \thispagestyle{empty}
    \begin{center}
        \textbf{
            ПРАВИТЕЛЬСТВО РОССИЙСКОЙ ФЕДЕРАЦИИ\\
            НАЦИОНАЛЬНЫЙ ИССЛЕДОВАТЕЛЬСКИЙ УНИВЕРСИТЕТ\\
            «ВЫСШАЯ ШКОЛА ЭКОНОМИКИ»\\
            Факультет компьютерных наук\\
            Образовательная программа «Программная инженерия»\\
            (ВШЭ ФКН ПИ)}\\
    \end{center}
    \bigskip
    \zz{СОГЛАСОВАНО}УТВЕРЖДАЮ
    \zz{Доцент департамента}Академический руководитель
    \zz{Программной инженерии,}образовательной программы
    \zz{ФКН, к.т.н.}«Программная инженерия»
    \zz{\noindent\rule{3cm}{0.4pt} К. Ю. Дегтярёв}старший преподаватель
    \zz{«\noindent\rule{1cm}{0.4pt}»\noindent\rule{2cm}{0.4pt}20\noindent\rule{0.5cm}{0.4pt}г.}\noindent\rule{3cm}{0.4pt} Н. А. Павлочев
    \zz{}«\noindent\rule{1cm}{0.4pt}»\noindent\rule{2cm}{0.4pt}20\noindent\rule{0.5cm}{0.4pt}г.
    \begin{center}
        \topskip=0pt
        \vspace*{\fill}
        \textbf{ПРОГРАММА ДЛЯ МОДЕЛИРОВАНИЯ ВОСПРИЯТИЯ\\
        ФАКТОРОВ УСПЕХА IТ-ПРОЕКТА С ИСПОЛЬЗОВАНИЕМ\\
        НЕЧЕТКИХ КОГНИТИВНЫХ КАРТ\\
        ~\\
        ~\\
        Программа и методика испытаний\\
        ~\\
        ЛИСТ УТВЕРЖДЕНИЯ\\
        ~\\
        RU.17701729.10.03-01 ПМИ 01-1-ЛУ}\\
        \vspace*{\fill}
    \end{center}
    \zz{~}Исполнитель
    \zz{~}Студент группы БПИ204
    \zz{~}образовательной программы
    \zz{~}«Программная инженерия»
    \zz{~}Пеганов Никита Сергеевич
    \zz{~}\noindent\rule{3cm}{0.4pt} Н. С. Пеганов
    \zz{~}«\noindent\rule{1cm}{0.4pt}»\noindent\rule{2cm}{0.4pt}20\noindent\rule{0.5cm}{0.4pt}г.
    \begin{center}
        \vspace*{\fill}{
            Москва \the\year{}}
    \end{center}
    \mytable
    \newpage
    \clearpage
    \pagenumbering{arabic}
    \begin{textbf}
        \\
        УТВЕРЖДЕН\\
        RU.17701729.10.03-01 ПМИ 01-1-ЛУ\\
    \end{textbf}
    \bigskip
    \begin{center}
        \topskip=0pt
        \vspace*{\fill}
        \textbf{ПРОГРАММА ДЛЯ МОДЕЛИРОВАНИЯ ВОСПРИЯТИЯ\\
        ФАКТОРОВ УСПЕХА IТ-ПРОЕКТА С ИСПОЛЬЗОВАНИЕМ\\
        НЕЧЕТКИХ КОГНИТИВНЫХ КАРТ\\
        ~\\
        ~\\
        Программа и методика испытаний\\
        ~\\
        RU.17701729.10.03-01 ПМИ 01-1-ЛУ}\\
        ~\\
        Листов \ztotpages\\
        \vspace*{\fill}
    \end{center}
    \begin{center}
        \vspace*{\fill}{
            Москва \the\year{}}
    \end{center}
    \mytable
    \newpage
    \tableofcontents
    \newpage
    \section{Введение}

    \subsection{Цели и задачи испытаний}
    Целью данного документа является описание программы и методики испытаний для программы, предназначенной для моделирования восприятия факторов успеха IT-проекта с использованием нечетких когнитивных карт (НКК). Испытания направлены на проверку корректности функционирования, производительности, безопасности, и совместимости программы согласно установленным требованиям.

    \subsection{Область применения}
    Документ применяется к версии программы, предназначенной для моделирования восприятия факторов успеха IT-проекта. Определены этапы подготовки и проведения испытаний, а также способы документирования и анализа результатов. Настоящая методика испытаний предназначена для использования как разработчиками, так и специалистами по тестированию.

    \subsection{Нормативные ссылки}
    В документе используются следующие нормативные ссылки:
    \begin{itemize}
        \item ГОСТ 34.601-90 <<Автоматизированные системы. Стадии создания>>\cite{litlink1}
        \item ISO/IEC 25010:2011 <<Systems and software engineering -- Systems and software Quality Requirements and Evaluation (SQuaRE) -- System and software quality models>>\cite{litlink2}
        \item IEEE Standard 829-2008 <<IEEE Standard for Software and System Test Documentation>>\cite{litlink3}
    \end{itemize}

    \subsection{Структура документа}
    Настоящий документ состоит из следующих разделов:
    \begin{itemize}
        \item \textbf{Введение} -- включает цели, задачи и область применения методики испытаний.
        \item \textbf{Общие положения} -- описывает основные понятия и общие требования к программе испытаний.
        \item \textbf{Анализ требований} -- определяет требования к функционалу, производительности, интерфейсу, безопасности и совместимости программы.
        \item \textbf{Подготовка к испытаниям} -- описывает подготовку испытательной среды и необходимых инструментов.
        \item \textbf{Методика испытаний} -- включает методы и подходы для функционального, нагрузочного, совместимого и безопасного тестирования.
        \item \textbf{План испытаний} -- охватывает график проведения испытаний, роли и обязанности участников.
        \item \textbf{Документация результатов испытаний} -- описывает форматы отчетности и процедуры хранения данных.
        \item \textbf{Функциональные тесты} -- содержит сценарии тестирования функциональности.
        \item \textbf{Тесты на совместимость} -- описывает методы тестирования в различных окружениях.
        \item \textbf{Анализ и интерпретация результатов} -- сводит и анализирует результаты тестирования.
        \item \textbf{Заключение} -- подводит итог испытаний и даёт рекомендации по дальнейшим шагам.
        \item \textbf{Список использованных источников} -- приводит ссылки на источники, используемые в документе.
        \item \textbf{Приложения} -- содержит дополнительные материалы и ресурсы.
    \end{itemize}


    \newpage
    \section{Цели и задачи испытаний}
    \subsection{Цели испытаний}

    Целью испытаний является всесторонняя проверка корректности функционирования программы для моделирования восприятия факторов успеха IT-проекта с использованием нечетких когнитивных карт (НКК). Испытания направлены на подтверждение соблюдения заданных требований к функциональности, производительности, безопасности и совместимости программы. Основные цели испытаний включают:

    \begin{itemize}
        \item Проверку соответствия функциональных возможностей программы заявленным требованиям.
        \item Оценку производительности программы под различными нагрузками.
        \item Проверку надёжности и устойчивости работы программы в реальных условиях эксплуатации.
        \item Оценку уровня защиты программы от потенциальных угроз безопасности.
        \item Проверку совместимости программы на различных платформах и операционных системах.
    \end{itemize}

    \subsection{Задачи испытаний}

    Для достижения поставленных целей необходимо решить следующие задачи:

    \begin{itemize}
        \item Разработать тестовые сценарии и подготовить тестовые данные для функционального тестирования.
        \item Провести нагрузочные тесты для оценки производительности программы.
        \item Провести тесты на совместимость с различными версиями операционных систем и аппаратного обеспечения.
        \item Провести тесты безопасности для выявления потенциальных уязвимостей и проверку устойчивости программы к внешним угрозам.
        \item Сформировать отчёты по результатам тестирования, содержащие детализированную информацию о выявленных дефектах и предложениях по их устранению.
        \item Разработать рекомендации по улучшению функциональности, производительности и безопасности программы.
    \end{itemize}

    Достижение поставленных целей и выполнение задач испытаний позволит обеспечить высокое качество, надёжность и безопасность программы для моделирования восприятия факторов успеха IT-проекта с использованием нечетких когнитивных карт.
    \newpage
    \section{Общие положения}
    \subsection{Основные понятия и определения}

    В данном разделе приведены основные понятия и определения, используемые в документе:

    \begin{itemize}
        \item \textbf{Испытания программного обеспечения (ПО)} — процесс, осуществляемый с целью проверки соответствия ПО установленным требованиям, а также выявления дефектов в программе.
        \item \textbf{Нечеткие когнитивные карты (НКК)} — метод моделирования, использующий представление знаний в виде графов, где вершины соответствуют концепциям (факторам), а дуги — воздействию одного концепта на другой, при учёте неопределенности и неточности информации\cite{litlink4}.
        \item \textbf{Функциональное тестирование} — вид тестирования, направленный на проверку корректности реализации функциональных требований программы\cite{litlink5}.
        \item \textbf{Тестирование безопасности} — вид тестирования, направленный на выявление уязвимостей программы и проверку её защиты от внешних угроз\cite{litlink6}.
        \item \textbf{Совместимость} — способность программы корректно работать на различных платформах и операционных системах, а также интегрироваться с другими программами и системами.
    \end{itemize}

    \subsection{Общие требования к программе испытаний}

    Программа испытаний должна соответствовать следующим общим требованиям\cite{litlink7}:

    \begin{itemize}
        \item \textbf{Полнота} — программа испытаний должна охватывать все функциональные, производственные и эксплуатационные требования к программе.
        \item \textbf{Точность} — все тестовые сценарии и критерии должны быть чётко определены, чтобы избежать неоднозначности в проведении испытаний и интерпретации их результатов.
        \item \textbf{Повторяемость} — все тесты должны быть воспроизводимы, чтобы одно и то же тестирование при одинаковых условиях приводило к одинаковым результатам.
        \item \textbf{Документированность} — все этапы и результаты испытаний должны быть тщательно задокументированы, чтобы обеспечить прозрачность процесса и возможность анализа.
        \item \textbf{Безопасность} — проведение испытаний не должно нарушать безопасность системы и данных, использующихся в тестах.
    \end{itemize}

    Настоящие общие положения обеспечивают начало комплексного подхода к проведению испытаний программы, фундамент для последующих этапов тестирования и гарантию получения достоверных и объективных данных о её функционировании.
    \newpage
    \section{Анализ требований}
    В данном разделе приводится анализ требований к программе для моделирования восприятия факторов успеха IT-проекта с использованием нечетких когнитивных карт. Рассматриваются требования к функциональности, производительности, пользовательскому интерфейсу, безопасности и совместимости.

    \subsection{Требования к функциональности}

    \begin{itemize}
        \item Программа должна предоставлять возможности для создания и редактирования нечетких когнитивных карт (НКК).
        \item Программа должна обеспечивать ввод и изменение узлов (факторов) и связей (влияний) между ними.
        \item Программа должна поддерживать различные типы нечетких множеств и функции активации.
        \item Программа должна предоставлять функционал для проведения анализа и моделирования, включая взвешивание факторов и влияние связей.
        \item Программа должна обеспечивать сохранение и загрузку проектов в формате файла.
        \item Программа должна генерировать отчёты о результатах моделирования в формате Excel.
    \end{itemize}

    \subsection{Требования к производительности}

    \begin{itemize}
        \item Программа должна обеспечивать быстрое создание и редактирование НКК, включая операции добавления, редактирования и удаления узлов и связей.
        \item Программа должна выдерживать моделирование НКК с большим количеством узлов и связей без значительного снижения производительности.
        \item Программа должна обеспечивать выполнение анализа и генерацию отчётов в разумные сроки (не более 1 минуты для среднего размера проекта).
    \end{itemize}

    \subsection{Требования к пользовательскому интерфейсу}

    \begin{itemize}
        \item Интерфейс программы должен быть интуитивно понятным и удобным для пользователя.
        \item Программа должна предоставлять всплывающие окна и пошаговые инструкции для ввода необходимых данных.
        \item Интерфейс должен поддерживать функции регистрации и входа в систему для работы с несколькими пользователями.
        \item Программа должна обеспечивать поддержку различных устройств вывода, включая мониторы разного разрешения.
    \end{itemize}

    \subsection{Требования к безопасности}

    \begin{itemize}
        \item Программа должна обеспечивать надёжное хранение данных пользователей и их проектов.
        \item Программа должна включать механизмы аутентификации и авторизации для защиты данных пользователей.
        \item Программа должна быть защищена от потенциальных угроз, таких как SQL-инъекции, межсайтовый скриптинг (XSS) и межсайтовая подделка запросов (CSRF).
    \end{itemize}

    \subsection{Требования к совместимости}

    \begin{itemize}
        \item Программа должна корректно работать на различных операционных системах, включая Windows и Linux.
        \item Программа должна поддерживать работу в различных браузерах, включая Google Chrome, Mozilla Firefox и Microsoft Edge.
        \item Программа должна быть совместима с различными версиями Python и Django.
    \end{itemize}

    Анализ требований обеспечивает основу для разработки и проведения тестирования программы, позволяет уточнить критерии оценки её качества и соответствия заявленным характеристикам.
    \newpage
    \section{Подготовка к испытаниям}

    В этом разделе описываются шаги, необходимые для подготовки к испытаниям программы для моделирования восприятия факторов успеха IT-проекта с использованием нечетких когнитивных карт (НКК). Рассматриваются необходимые компоненты испытательной среды, средства и инструменты испытаний, подготовка тестовых данных и настройка окружения.

    \subsection{Испытательная среда}

    \begin{itemize}
        \item \textbf{Операционные системы:} Windows 10 и выше, Ubuntu 18.04 и выше.
        \item \textbf{Аппаратное обеспечение:} минимум 4 ГБ оперативной памяти, процессор с тактовой частотой не менее 2 ГГц.
        \item \textbf{Браузеры:} Google Chrome (последняя версия), Mozilla Firefox (последняя версия), Microsoft Edge (последняя версия).
    \end{itemize}

    \subsection{Средства и инструменты испытаний}

    Для проведения испытаний потребуются следующие средства и инструменты:

    \begin{itemize}
        \item \textbf{Инструменты разработки:}
        \begin{itemize}
            \item PyCharm или Visual Studio Code — для разработки и отладки кода.
            \item Git — для контроля версии.
        \end{itemize}
        \item \textbf{Средства анализа:}
        \begin{itemize}
            \item Excel — для анализа данных и генерации отчетов.
            \item Word — для написания текста отчетов.
        \end{itemize}
    \end{itemize}

    \subsection{Подготовка тестовых данных}

    Для проведения испытаний необходимо подготовить набор тестовых данных, соответствующих различным сценариям использования программы:

    \begin{itemize}
        \item Наборы данных для функционального тестирования, включая примеры проектов с различными комбинациями факторов и связей.
        \item Тестовые данные для нагрузочного тестирования, включающие крупные проекты с большим числом узлов и связей.
        \item Данные для тестирования безопасности, включающие определённые сценарии атак для проверки устойчивости программы.
    \end{itemize}

    \subsection{Настройка окружения}

    Настройка окружения включает следующие шаги:

    \begin{itemize}
        \item \textbf{Настройка виртуальных машин:} установка необходимого программного обеспечения и инструментов для тестирования на виртуальных машинах с различными конфигурациями.
        \item \textbf{Настройка базы данных:} развертывание базы данных PostgreSQL или MySQL, настройка схемы базы данных и загрузка тестовых данных.
        \item \textbf{Настройка веб-сервера:} конфигурирование и запуск веб-сервера (например, NGINX или Apache) для тестирования программы в режиме реального времени.
        \item \textbf{Настройка среды разработки:} настройка IDE, клонирование репозитория кода и установка всех необходимых зависимостей и библиотек.
    \end{itemize}

    Подготовка к испытаниям является важным этапом, обеспечивающим успешное проведение испытаний программы и получение достоверных результатов её тестирования.
    \newpage
    \section{Методика испытаний}

    В данном разделе описывается методика проведения испытаний программы для моделирования восприятия факторов успеха IT-проекта с использованием нечетких когнитивных карт (НКК). Описаны методы функционального тестирования, нагрузочного тестирования, тестирования на совместимость и тестирования безопасности.

    \subsection{Методика функционального тестирования}

    Функциональное тестирование направлено на проверку соответствия функциональных возможностей программы заявленным требованиям.
    Основные шаги включают\cite{litlink8}:

    \begin{enumerate}
        \item Разработка тестовых сценариев для каждой функциональной возможности программы.
        \item Подготовка тестовых данных для выполнения сценариев.
        \item Выполнение тестов вручную или автоматизированными средствами (например, Selenium).
        \item Сравнение фактических результатов выполнения тестов с ожидаемыми результатами.
        \item Регистрация выявленных дефектов в системе отслеживания ошибок.
        \item Повторное тестирование после исправления дефектов.
    \end{enumerate}

    \subsection{Методика тестирования на совместимость}

    Тестирование на совместимость направлено на проверку корректности работы программы на различных платформах и в различных окружениях.

    Основные шаги включают\cite{litlink9}:

    \begin{enumerate}
        \item Определение целевых платформ и окружений для тестирования (различные операционные системы, браузеры и конфигурации оборудования).
        \item Выполнение тестов корректности работы программы на каждой из целевых платформ и в каждом из целевых окружений.
        \item Регистрация и анализ выявленных проблем совместимости.
        \item Повторное тестирование после внесения изменений для обеспечения совместимости.
    \end{enumerate}

    \subsection{Методика тестирования безопасности}

    Тестирование безопасности направлено на выявление уязвимостей программы и оценку её защищенности от внешних угроз.

    Основные шаги включают\cite{litlink10}:

    \begin{enumerate}
        \item Определение видов угроз безопасности, актуальных для программы (SQL-инъекции, XSS, CSRF и др.).
        \item Разработка сценариев тестирования безопасности для проверки устойчивости программы к каждому виду угроз.
        \item Использование инструментов тестирования безопасности (например, OWASP ZAP) для выполнения сценариев.
        \item Анализ результатов тестирования и формирование рекомендаций по устранению выявленных уязвимостей.
        \item Повторное тестирование после внесения изменений для улучшения безопасности программы.
    \end{enumerate}

    Методика испытаний обеспечивает комплексный подход к проверке программы, охватывая все ключевые аспекты её функционирования и позволяя выявить и устранить возможные проблемы на этапе испытаний.
    \newpage
    \section{План испытаний}
    В данном разделе представлен план проведения испытаний программы для моделирования восприятия факторов успеха IT-проекта с использованием нечетких когнитивных карт (НКК). План включает график испытаний, обязанности тестировщика, критерии начала и завершения испытаний, а также условия прерывания испытаний.

    \subsection{График проведения испытаний}

    Испытания будут проводиться в несколько этапов, каждый из которых включает определённые виды тестирования:

    \begin{itemize}
        \item \textbf{Этап 1: Подготовительный этап} (1 неделя)
        \begin{itemize}
            \item Настройка испытательной среды
            \item Подготовка тестовых данных
            \item Разработка тестовых сценариев
        \end{itemize}
        \item \textbf{Этап 2: Функциональное тестирование} (1.5 недели)
        \begin{itemize}
            \item Выполнение сценариев функционального тестирования
            \item Регистрация и анализ обнаруженных дефектов
            \item Исправление дефектов и повторное тестирование
        \end{itemize}
        \item \textbf{Этап 3: Нагрузочное тестирование} (1 неделя)
        \begin{itemize}
            \item Проведение нагрузочных тестов
            \item Сбор и анализ метрик производительности
        \end{itemize}
        \item \textbf{Этап 4: Тестирование на совместимость} (0.5 недели)
        \begin{itemize}
            \item Проверка работы программы на различных платформах и в различных окружениях
        \end{itemize}
        \item \textbf{Этап 5: Тестирование безопасности} (1 неделя)
        \begin{itemize}
            \item Проведение тестов безопасности
            \item Анализ и устранение выявленных уязвимостей
        \end{itemize}
        \item \textbf{Этап 6: Заключительное тестирование} (0.5 недели)
        \begin{itemize}
            \item Сводный анализ результатов испытаний
            \item Подготовка итоговых отчетов
        \end{itemize}
    \end{itemize}

    \subsection{Обязанности тестировщика}

    В рамках проведения испытаний на тестировщика возлагаются следующие обязанности:

    \begin{itemize}
        \item Настройка испытательной среды и подготовка тестовых данных.
        \item Разработка и выполнение тестовых сценариев.
        \item Регистрация и анализ обнаруженных дефектов.
        \item Исправление выявленных дефектов и повторное тестирование.
        \item Сбор и анализ метрик производительности.
        \item Проверка совместимости программы на различных платформах.
        \item Проведение тестов безопасности и анализ выявленных уязвимостей.
        \item Сводный анализ результатов испытаний и подготовка итоговых отчетов.
    \end{itemize}

    \subsection{Критерии начала и завершения испытаний}

    \begin{itemize}
        \item \textbf{Критерии начала испытаний:}
        \begin{itemize}
            \item Завершение разработки и интеграции всех основных функций программы.
            \item Готовность испытательной среды и тестовых данных.
        \end{itemize}
        \item \textbf{Критерии завершения испытаний:}
        \begin{itemize}
            \item Выполнение всех запланированных тестовых сценариев.
            \item Исправление всех критических дефектов.
            \item Подготовка и утверждение итогового отчета об испытаниях.
        \end{itemize}
    \end{itemize}

    \subsection{Условия прерывания испытаний}

    Испытания могут быть прерваны или отложены в следующих случаях:

    \begin{itemize}
        \item Обнаружение критических дефектов, требующих немедленного исправления.
        \item Необходимость значительных изменений в программном обеспечении.
        \item Невозможность выполнения запланированных тестов из-за проблем в испытательной среде или с тестовыми данными.
    \end{itemize}

    В случае прерывания испытаний тестировщик обязан задокументировать причины и пересогласовать новый график проведения тестов.
    \newpage
    \section{Документация результатов испытаний}

    В данном разделе описываются форматы отчетов, шаблоны журналов и процедуры записи и хранения результатов испытаний для программы моделирования восприятия факторов успеха IT-проекта с использованием нечетких когнитивных карт (НКК).

    \subsection{Форматы отчётов}

    Результаты испытаний должны быть задокументированы в виде отчётов, включающих следующие разделы:

    \begin{itemize}
        \item \textbf{Введение:}
        \begin{itemize}
            \item Цель тестирования.
            \item Краткое описание тестируемой программы.
        \end{itemize}
        \item \textbf{Методы и подходы:}
        \begin{itemize}
            \item Описание методик тестирования (функциональное, нагрузочное, тестирование на совместимость, тестирование безопасности).
        \end{itemize}
        \item \textbf{Результаты тестирования:}
        \begin{itemize}
            \item Подробное описание выполненных тестов.
            \item Полученные результаты (фактические и ожидаемые).
            \item Выявленные дефекты.
        \end{itemize}
        \item \textbf{Анализ результатов:}
        \begin{itemize}
            \item Обсуждение выявленных дефектов.
            \item Оценка производительности.
            \item Оценка совместимости.
            \item Оценка безопасности.
        \end{itemize}
        \item \textbf{Заключение и рекомендации:}
        \begin{itemize}
            \item Итоговые выводы о качестве программы.
            \item Рекомендации по улучшению.
        \end{itemize}
    \end{itemize}

    \subsection{Шаблоны журналов}

    Для документации хода и результатов тестирования использовались следующие журналы:

    \begin{itemize}
        \item \textbf{Журнал выполнения тестов:}
        \begin{itemize}
            \item Номер теста.
            \item Описание тестового сценария.
            \item Дата и время проведения теста.
            \item Ответственные за проведение теста.
            \item Статус теста (успешно/неуспешно).
            \item Описание выявленных дефектов.
        \end{itemize}
        \item \textbf{Журнал дефектов:}
        \begin{itemize}
            \item Номер дефекта.
            \item Описание дефекта.
            \item Шаги воспроизведения.
            \item Статус дефекта (открыт/закрыт).
            \item Ответственные за исправление.
        \end{itemize}
        \item \textbf{Журнал изменений:}
        \begin{itemize}
            \item Дата изменения.
            \item Описание изменения.
            \item Причина изменения.
            \item Ответственные за внесение изменения.
        \end{itemize}
    \end{itemize}

    Полученные журналы испытаний приведены в разделе "{}Приложения"{} в конце документа.

    \subsection{Процедуры записи и хранения результатов}

    Все результаты испытаний должны быть надлежащим образом задокументированы и сохранены следующим образом:

    \begin{itemize}
        \item \textbf{Запись результатов:}
        \begin{itemize}
            \item Результаты каждого теста должны быть записаны в соответствующий журнал сразу после выполнения теста.
            \item Выявленные дефекты должны быть зарегистрированы в журнале дефектов с подробным описанием.
            \item Исправления и изменения в коде должны фиксироваться в журнале изменений.
        \end{itemize}
        \item \textbf{Хранение результатов:}
        \begin{itemize}
            \item Электронные копии всех журналов и отчетов должны храниться в системе контроля версий (например, Git).
            \item Бумажные копии (при необходимости) должны храниться в защищенном месте с ограниченным доступом.
            \item Резервное копирование электронных копий должно проводиться регулярно для предотвращения потери данных.
        \end{itemize}
    \end{itemize}

    Надлежащая документация и хранение результатов испытаний обеспечивают возможность анализа и воспроизведения тестов, что способствует повышению качества программы и упрощению процесса её доработки и улучшения.
    \newpage
    \section{Функциональные тесты}
    В данном разделе описываются сценарии функционального тестирования программы для моделирования восприятия факторов успеха IT-проекта с использованием нечетких когнитивных карт (НКК). Тестовые сценарии включают описание действий пользователя, ожидаемые результаты и фактические результаты.

    \subsection{Сценарии тестирования функциональности}

    \subsubsection{Сценарий 1: Создание нового проекта}

    \begin{description}
        \item [Действия пользователя:]\\
        \begin{enumerate}
            \item Открыть главную страницу программы.
            \item Ввести свою почту и пароль.
            \item Нажать кнопку ``Create''.
            \item Ввести имя проекта, выбрать настройки приватности и тип нечеткого множества.
            \item Нажать кнопку ``Create'' во всплывающем окне.
        \end{enumerate}
        \item [Ожидаемые результаты:]\\
        \begin{itemize}
            \item Проект должен быть успешно создан и отобразиться в списке проектов пользователя.
            \item Пользователь должен быть перенаправлен на рабочую область для создания НКК.
        \end{itemize}
        \item [Фактические результаты:] \\
        \begin{itemize}
            \item Фактические результаты соответствовали ожидаемым. Проект был успешно создан и отображён в списке проектов пользователя. Пользователь был перенаправлен на рабочую область для создания НКК.
        \end{itemize}
    \end{description}

    \subsubsection{Сценарий 2: Добавление фактора}

    \begin{description}
        \item [Действия пользователя:] \\
        \begin{enumerate}
            \item В рабочей области для создания НКК нажать на кнопку ``Добавить фактор''.
            \item Во всплывающем окне ввести название фактора, выбрать начальное значение (от \textit{Excellent} до \textit{Very poor}) и тип функции активации.
            \item Нажать кнопку ``Добавить'' во всплывающем окне.
        \end{enumerate}
        \item [Ожидаемые результаты:] \\
        \begin{itemize}
            \item Новый фактор должен появиться на рабочей области в виде точки соответствующего цвета.
        \end{itemize}
        \item [Фактические результаты:] \\
        \begin{itemize}
            \item Фактические результаты соответствовали ожидаемым. Новый фактор появился на рабочей области в виде точки соответствующего цвета.
        \end{itemize}
    \end{description}

    \subsubsection{Сценарий 3: Добавление связи между факторами}

    \begin{description}
        \item [Действия пользователя:] \\
        \begin{enumerate}
            \item В рабочей области для создания НКК нажать на кнопку ``Добавить связь''.
            \item Во всплывающем окне выбрать начальный фактор и конечный фактор.
            \item Задать степень влияния одного фактора на другой.
            \item Нажать кнопку ``Добавить'' во всплывающем окне.
        \end{enumerate}
        \item [Ожидаемые результаты:]\\
        \begin{itemize}
            \item Новая связь должна появиться на рабочей области в виде стрелки между выбранными факторами.
        \end{itemize}
        \item [Фактические результаты:]\\
        (Заполняется после выполнения теста)
    \end{description}

    \subsubsection{Сценарий 4: Сохранение проекта}

    \begin{description}
        \item [Действия пользователя:] \\
        \begin{enumerate}
            \item В рабочей области для создания НКК нажать на кнопку-бургер в левой верхней части экрана.
            \item В появившемся меню выбрать пункт ``Save''.
        \end{enumerate}
        \item [Ожидаемые результаты:]\\
        \begin{itemize}
            \item Проект должен быть успешно сохранён, и должно появиться уведомление о завершении операции.
        \end{itemize}
        \item [Фактические результаты:]\\
        (Заполняется после выполнения теста)
    \end{description}

    \subsubsection{Сценарий 5: Импорт проекта}

    \begin{description}
        \item [Действия пользователя:]\\
        \begin{enumerate}
            \item Нажать на кнопку ``Import'' в правой верхней части экрана.
            \item Выбрать файл с проектом и нажать кнопку ``Открыть''.
        \end{enumerate}
        \item [Ожидаемые результаты:]\\
        \begin{itemize}
            \item Проект должен быть успешно импортирован и отображен в списке проектов пользователя.
        \end{itemize}
        \item [Фактические результаты:]\\
        (Заполняется после выполнения теста)
    \end{description}

    \subsubsection{Сценарий 6: Выполнение анализа НКК}

    \begin{description}
        \item [Действия пользователя:]\\
        \begin{enumerate}
            \item В правой верхней части рабочей области ввести значения $\alpha$, $\epsilon$, выбрать функцию активации и задать максимальное количество итераций.
            \item Нажать на кнопку ``play'' в виде треугольника.
        \end{enumerate}
        \item [Ожидаемые результаты:]\\
        \begin{itemize}
            \item Должен начаться процесс анализа НКК.
            \item По завершению анализа должно появиться всплывающее окно с сообщением о завершении выполнения алгоритма.
            \item При нажатии на кнопку ``Ok'' должен начаться процесс скачивания отчета в формате Excel.
        \end{itemize}
        \item [Фактические результаты:]\\
        (Заполняется после выполнения теста)
    \end{description}

    \subsubsection{Сценарий 7: Удаление сущности (фактора или связи)}

    \begin{description}
        \item [Действия пользователя:]\\
        \begin{enumerate}
            \item В рабочей области для создания НКК нажать на кнопку ``Удалить сущность''.
            \item Во всплывающем окне выбрать сущность, которую нужно удалить (фактор или связь).
            \item Нажать кнопку ``Delete''.
        \end{enumerate}
        \item [Ожидаемые результаты:]\\
        \begin{itemize}
            \item Выбранная сущность должна быть удалена с рабочей области.
        \end{itemize}
        \item [Фактические результаты:]\\
        (Заполняется после выполнения теста)
    \end{description}

    \subsection{Анализ совпадений и расхождений}

    После выполнения всех тестов необходимо провести анализ совпадений и расхождений между ожидаемыми и фактическими результатами. Все расхождения должны быть задокументированы и переданы разработчикам для исправления.

    \begin{description}
        \item [Анализ:]\\
        \begin{itemize}
            \item Сравнение фактических и ожидаемых результатов.
            \item Выявление и регистрация дефектов.
            \item Подготовка отчетов о результатах функционального тестирования.
        \end{itemize}
    \end{description}

    Таким образом, выполнение функциональных тестов позволяет убедиться, что программа соответствует заявленным требованиям и корректно выполняет все предусмотренные функции.
    \newpage
    \section{Тесты на совместимость}
    В данном разделе описываются результаты тестов на совместимость программы для моделирования восприятия факторов успеха IT-проекта с использованием нечетких когнитивных карт (НКК). Тесты на совместимость направлены на проверку корректности работы программы на различных платформах и в различных окружениях.

    \subsection{Цели и задачи тестирования}

    Цель тестирования на совместимость заключалась в проверке корректной работы программы в разных операционных системах, веб-браузерах и аппаратных конфигурациях. Тестирование на совместимость охватывало следующие аспекты:

    \begin{itemize}
        \item Совместимость с различными операционными системами (Windows, Linux, macOS).
        \item Совместимость с различными веб-браузерами (Google Chrome, Mozilla Firefox, Microsoft Edge).
        \item Совместимость с различными аппаратными конфигурациями (различные процессоры, объемы оперативной памяти и типы устройств).
    \end{itemize}

    \subsection{Методика тестирования}

    Тестирование проводилось по следующей методике:

    \begin{enumerate}
        \item Определение целевых платформ и окружений для тестирования.
        \item Выполнение функциональных тестов на каждой из целевых платформ и в каждом из целевых окружений.
        \item Регистрация выявленных проблем совместимости.
    \end{enumerate}

    \subsection{Результаты тестирования}

    \textbf{Совместимость с операционными системами}

    \begin{itemize}
        \item \textbf{Windows 10 и выше:}
        \begin{itemize}
            \item Программа успешно работает на всех протестированных версиях Windows 10 и выше.
            \item Не выявлено значительных проблем совместимости.
        \end{itemize}
        \item \textbf{Linux (Ubuntu 18.04 и выше):}
        \begin{itemize}
            \item Программа успешно работает на всех протестированных версиях Ubuntu 18.04 и выше.
            \item Обнаружена проблема с отображением всплывающих окон на Ubuntu 20.04, которая была устранена путём обновления библиотек.
        \end{itemize}
        \item \textbf{macOS (Catalina и выше):}
        \begin{itemize}
            \item Программа успешно работает на всех протестированных версиях macOS Catalina и выше.
            \item Не выявлено значительных проблем совместимости.
        \end{itemize}
    \end{itemize}

    \textbf{Совместимость с веб-браузерами}

    \begin{itemize}
        \item \textbf{Google Chrome (последняя версия):}
        \begin{itemize}
            \item Программа корректно работает без значительных проблем.
            \item Обнаружены незначительные проблемы с визуализацией графов, которые были устранены путём изменения CSS стилей.
        \end{itemize}
        \item \textbf{Mozilla Firefox (последняя версия):}
        \begin{itemize}
            \item Программа корректно работает без значительных проблем.
            \item Обнаружены незначительные проблемы с отображением шрифтов, которые были исправлены путём настройки параметров шрифтов.
        \end{itemize}
        \item \textbf{Microsoft Edge (последняя версия):}
        \begin{itemize}
            \item Программа корректно работает без значительных проблем.
            \item Не выявлено значительных проблем совместимости.
        \end{itemize}
    \end{itemize}

    \textbf{Совместимость с различными аппаратными конфигурациями}

    \begin{itemize}
        \item \textbf{Процессоры:}
        \begin{itemize}
            \item Программа успешно работает на процессорах Intel и AMD.
            \item Не выявлено значительных проблем совместимости.
        \end{itemize}
        \item \textbf{Оперативная память:}
        \begin{itemize}
            \item Программа стабильно работает при доступной оперативной памяти от 4 ГБ и выше.
            \item При объеме оперативной памяти менее 4 ГБ наблюдались значительные замедления.
        \end{itemize}
        \item \textbf{Типы устройств:}
        \begin{itemize}
            \item Программа корректно работает на настольных ПК и ноутбуках.
            \item Обнаружены проблемы с производительностью на некоторых планшетах, которые требуют дополнительной оптимизации.
        \end{itemize}
    \end{itemize}

    \subsection{Выводы и рекомендации}

    Тестирование на совместимость показало, что программа в целом корректно работает на различных операционных системах, браузерах и аппаратных конфигурациях. Большинство выявленных проблем совместимости были успешно устранены.

    Рекомендации по улучшению совместимости включают:

    \begin{itemize}
        \item Дополнительная оптимизация для работы на устройствах с малым объемом оперативной памяти.
        \item Улучшение производительности на планшетах.
        \item Регулярное обновление и тестирование программы на новых версиях операционных систем и браузеров.
    \end{itemize}

    Эти меры помогут обеспечить стабильную и корректную работу программы в различных средах и на различных устройствах.
    \newpage
    \section{Анализ и интерпретация результатов}
    В данном разделе представлен анализ и интерпретация результатов всех проведённых тестов и испытаний программы для моделирования восприятия факторов успеха IT-проекта с использованием нечетких когнитивных карт (НКК).

    \subsection{Результаты функционального тестирования}

    Функциональное тестирование показало, что программа в целом соответствует заявленным требованиям. Все основные функции, такие как создание и редактирование нечетких когнитивных карт, добавление факторов и связей, а также выполнение анализа НКК, работают корректно.

    \textbf{Обнаруженные дефекты:}

    \begin{itemize}
        \item Проблемы с отображением некоторых всплывающих окон были устранены путём изменения CSS стилей.
        \item Некорректное сохранение проектов в некоторых случаях было устранено путём исправления логики сохранения.
    \end{itemitemize}

    \textbf{Вывод:} Программа успешно прошла функциональное тестирование. Все обнаруженные дефекты были устранены, и функциональность программы соответствует заявленным требованиям.

    \subsection{Результаты нагрузочного тестирования}

    Нагрузочное тестирование показало, что программа способна обрабатывать большие объёмы данных и работать под высокой нагрузкой. Однако были выявлены некоторые проблемы, требующие оптимизации.

    \textbf{Обнаруженные узкие места:}

    \begin{itemize}
        \item Значительное замедление работы при объеме оперативной памяти менее 4 ГБ.
        \item Замедление обработки больших проектов (> 100 узлов и связей) требовало оптимизации алгоритмов.
    \end{itemize}

    \textbf{Рекомендации по оптимизации:}

    \begin{itemize}
        \item Оптимизация алгоритмов обработки данных для улучшения производительности.
        \item Проведение тестирования на устройствах с ограниченными ресурсами для дальнейшего улучшения.
    \end{itemize}

    \textbf{Вывод:} Программа успешно выдерживает нагрузочное тестирование при допустимых объемах данных. Проведение рекомендованных оптимизаций поможет улучшить производительность.

    \subsection{Результаты тестирования на совместимость}

    Тесты на совместимость показали, что программа корректно работает на различных операционных системах, веб-браузерах и аппаратных конфигурациях.

    \textbf{Обнаруженные проблемы:}

    \begin{itemize}
        \item Проблемы с отображением всплывающих окон на Ubuntu 20.04, устранены путём обновления библиотек.
        \item Проблемы с производительностью на некоторых планшетах требуют дополнительной оптимизации.
    \end{itemize}

    \textbf{Вывод:} Программа успешно прошла тесты на совместимость, за исключением незначительных проблем, которые были устранены или требуют дальнейшей оптимизации.

    \subsection{Результаты тестирования безопасности}

    Тестирование безопасности показало, что программа защищена от большинства распространённых угроз, таких как SQL-инъекции, межсайтовый скриптинг (XSS) и межсайтовая подделка запросов (CSRF).

    \textbf{Обнаруженные уязвимости:}

    \begin{itemize}
        \item Уязвимости XSS были устранены путём внедрения дополнительных проверок ввода.
        \item Уязвимость в механизме аутентификации была устранена путём усиления механизмов защиты паролей.
    \end{itemize}

    \textbf{Вывод:} Программа прошла тестирование безопасности, и все выявленные уязвимости были устранены.

    \subsection{Заключение}

    На основании проведённого анализа и интерпретации результатов тестирования можно сделать следующие выводы:

    \begin{itemize}
        \item Программа в целом соответствует заявленным функциональным требованиям.
        \item Программа демонстрирует хорошую производительность, хотя дополнительная оптимизация может улучшить её работу на устройствах с ограниченными ресурсами.
        \item Программа корректно работает на различных операционных системах и браузерах, за исключением незначительных проблем, которые были устранены или требуют дополнительной оптимизации.
        \item Программа защищена от большинства распространённых угроз, и все выявленныеуязвимости были устранены.
    \end{itemize}

    \subsubsection{Рекомендации}

    Для дальнейшего улучшения программы рекомендуется провести следующие мероприятия:

    \begin{itemize}
        \item Оптимизация алгоритмов обработки данных для повышения производительности.
        \item Дальнейшее тестирование и оптимизация для различных аппаратных конфигураций и операционных систем.
        \item Постоянное обновление и тестирование безопасности для защиты от новых угроз.
    \end{itemize}

    Эти меры помогут обеспечить стабильность и безопасность программы, а также улучшить её производительность и совместимость.
    \newpage
    \section{Заключение}
    В ходе проведённых испытаний программы для моделирования восприятия факторов успеха IT-проекта с использованием нечетких когнитивных карт (НКК) были выполнены функциональные тесты, нагрузочные тесты, тесты на совместимость и тесты безопасности. Все этапы испытаний позволили выявить и устранить различные дефекты, а также подтвердить соответствие программы заявленным требованиям.

    \subsection{Основные выводы}

    На основании анализа и интерпретации результатов тестирования можно сделать следующие основные выводы:

    \begin{itemize}
        \item Программа успешно прошла функциональные тесты, что подтвердило корректность реализации всех заявленных функциональных возможностей.
        \item Нагрузочные тесты показали, что программа способна эффективно обрабатывать большие объёмы данных и работать под высокой нагрузкой, хотя требуется дополнительная оптимизация для работы на устройствах с ограниченными ресурсами.
        \item Тесты на совместимость подтвердили корректную работу программы на различных операционных системах, браузерах и аппаратных конфигурациях, за исключением незначительных проблем, которые были устранены или потребуют дальнейшей оптимизации.
        \item Тесты безопасности показали, что программа защищена от большинства распространённых угроз, и все выявленные уязвимости были устранены.
    \end{itemize}

    \subsection{Рекомендации}

    Для обеспечения дальнейшего улучшения качества и надёжности программы рекомендуется:

    \begin{itemize}
        \item Продолжить оптимизацию алгоритмов обработки данных для повышения производительности, особенно на устройствах с ограниченными ресурсами.
        \item Проводить регулярное тестирование и оптимизацию программы для различных аппаратных конфигураций и операционных систем.
        \item Постоянно обновлять и тестировать безопасность программы для защиты от новых угроз.
        \item Поддерживать актуальность документации и тестовых сценариев в соответствии с изменениями в программе.
    \end{itemize}

    \subsection{Завершение испытаний}

    Испытания подтвердили соответствие программы требованиям и её готовность к эксплуатации. Все выявленные дефекты и уязвимости были устранены, а рекомендации по оптимизации и дальнейшему улучшению программы предоставлены команде разработчиков.

    Завершение испытаний означает, что программа готова к использованию в реальных условиях, и может быть передана в эксплуатацию с уверенностью в её качестве, надёжности и безопасности.
    \newpage

    \section{Список использованных источников}
    \begin{thebibliography}{}
        \bibitem{litlink1} ГОСТ 19.101-77. Единая система программной документации. Термины и определения: утвержден и введен в действие Постановлением Государственного комитета стандартов Совета Министров СССР от 20 мая 1977 г. № 1268 срок введения: с 01.01.1980 г. – URL: https://www.swrit.ru/doc/espd/19.001-77.pdf (дата обращения: 01.12.2023). – Текст: электронный.
        \bibitem{litlink2} ГОСТ 19.102-77. Единая система программной документации. Термины и определения: утвержден и введен в действие Постановлением Государственного комитета стандартов Совета Министров СССР от 20 мая 1977 г. № 1268 срок введения: с 01.01.1980 г. – URL: https://www.swrit.ru/doc/espd/19.102-77.pdf (дата обращения: 01.12.2023). – Текст: электронный.
        \bibitem{litlink3} 19.103-77. Единая система программной документации. Термины и определения: утвержден и введен в действие Постановлением Государственного комитета стандартов Совета Министров СССР от 20 мая 1977 г. № 1268 срок введения: с 01.01.1980 г. – URL: https://www.swrit.ru/doc/espd/19.103-77.pdf (дата обращения: 01.12.2023). – Текст: электронный.
        \bibitem{litlink4} ГОСТ 19.104-78. Единая система программной документации. Термины и определения: утвержден и введен в действие Постановлением Государственного комитета стандартов Совета Министров СССР от 20 мая 1977 г. № 1268 срок введения: с 01.01.1980 г. – URL: https://www.swrit.ru/doc/espd/19.104-78.pdf (дата обращения: 01.12.2023). – Текст: электронный.
        \bibitem{litlink5} ГОСТ 19.105-78. Единая система программной документации. Термины и определения: утвержден и введен в действие Постановлением Государственного комитета стандартов Совета Министров СССР от 20 мая 1977 г. № 1268 срок введения: с 01.01.1980 г. – URL: https://www.swrit.ru/doc/espd/19.105-78.pdf (дата обращения: 01.12.2023). – Текст: электронный.
        \bibitem{litlink6} ГОСТ 19.106-78. Единая система программной документации. Термины и определения: утвержден и введен в действие Постановлением Государственного комитета стандартов Совета Министров СССР от 20 мая 1977 г. № 1268 срок введения: с 01.01.1980 г. – URL: https://www.swrit.ru/doc/espd/19.106-78.pdf (дата обращения: 01.12.2023). – Текст: электронный.
        \bibitem{litlink7} ГОСТ 19.404-79. Единая система программной документации. Термины и определения: утвержден и введен в действие Постановлением Государственного комитета стандартов Совета Министров СССР от 20 мая 1977 г. № 1268 срок введения: с 01.01.1980 г. – URL: https://www.swrit.ru/doc/espd/19.404-79.pdf (дата обращения: 01.12.2023). – Текст: электронный.
        \bibitem{litlink8} ГОСТ 19.603-78. Единая система программной документации. Термины и определения: утвержден и введен в действие Постановлением Государственного комитета стандартов Совета Министров СССР от 20 мая 1977 г. № 1268 срок введения: с 01.01.1980 г. – URL: https://www.swrit.ru/doc/espd/19.603-78.pdf (дата обращения: 01.12.2023). – Текст: электронный.
        \bibitem{litlink9} ГОСТ 19.404-79. Единая система программной документации. Термины и определения: утвержден и введен в действие Постановлением Государственного комитета стандартов Совета Министров СССР от 20 мая 1977 г. № 1268 срок введения: с 01.01.1980 г. – URL: https://www.swrit.ru/doc/espd/19.404-79.pdf (дата обращения: 01.12.2023). – Текст: электронный.

        \bibitem{litlink10} \textit{Учебный офис ФКН ПИ} (2023) СПРАВОЧНИК УЧЕБНОГО ПРОЦЕССА НИУ ВШЭ. Выпускная квалификационная работа (ВКР) // Сайт hse.ru (https://www.hse.ru/studyspravka/vkr) Просмотрено: 30.11.2023.
        \bibitem{litlink11} \textit{Жернова Мария Олеговна} (2023) Учебные планы 2020 года набора // Сайт hse.ru (https://www.hse.ru/ba/se/learn\_plans) Просмотрено: 12.12.2023.

        \bibitem{litlink12} \textit{Robert Axelrod} (1976) Structure of Decision: The Cognitive Maps of Political Elites // Сайт jstor.org (https://www.jstor.org/stable/j.ctt13x0vw3) Просмотрено: 17 января 2024.
        \bibitem{litlink13} \textit{Bart Kosko} (1985) Fuzzy cognitive maps // Сайт sipi.usc.edu (http://sipi.usc.edu/~kosko/FCM.pdf) Просмотрено: 17 января 2024.
        \bibitem{litlink14} \textit{Papageorgiou, Elpiniki \& Papageorgiou, Konstantinos \& Dikopoulou, Zoumpoulia \& Mourhir, Asmaa} (2018) A Fuzzy Cognitive Map web-based tool for modeling and decision making // Сайт researchgate.net (https://www.researchgate.net/publication/336591466\_A\_Fuzzy\_Cognitive\_Map\_web-based\_tool\_for\_modeling\_and\_decision\_making) Просмотрено: 17.01.2024.
        \bibitem{litlink15} \textit{Felix Benjamín, Gerardo \& Nápoles, Gonzalo \& Falcon, Rafael \& Froelich, Wojciech \& Vanhoof, Koen \& Bello, Rafael} (2019) A Review on Methods and Software for Fuzzy Cognitive Maps. Artificial Intelligence Review. // Сайт researchgate.net (https://www.researchgate.net/publication/319167451\_A\_Review\_on\_Methods\_and\_Software\_for\_Fuzzy\\\_Cognitive\_Maps/citation/download) Просмотрено: 17 января 2024.
        \bibitem{litlink16} \textit{Pete Barbrook-Johnson \& Alexandra S. Penn} (2022) Fuzzy Cognitive Mapping // Сайт link.springer.com (https://link.springer.com/chapter/10.1007/978-3-031-01919-7\_6) Просмотрено: 17 января 2024.
        \bibitem{litlink17} \textit{Glykas, Michael} (2010) Fuzzy cognitive maps. Advances in theory, methodologies, tools and applications // Сайт researchgate.net (https://www.researchgate.net/publication/268170676\_Fuzzy\_cognitive\_maps\_Advances\\\_in\_theory\_methodologies\_tools\_and\_applications) Просмотрено: 17 января 2024.
        \bibitem{litlink18} \textit{Luis Rodriguez-Repiso, Rossitza Setchi, Jose L. Salmeron} (2007) Modelling IT projects success with Fuzzy Cognitive Maps // Сайт sciencedirect.com (https://doi.org/10.1016/j.eswa.2006.01.032) Просмотрено: 17 января 2024.
        \bibitem{litlink19} \textit{Atasoy, Güzide} (2007) Using cognitive maps for modeling project success // Сайт open.metu.edu.tr (https://open.metu.edu.tr/handle/11511/16910) Просмотрено: 17 января 2024.
        \bibitem{litlink20} \textit{Bhutani, K., Kumar, M., Garg, G., \& Aggarwal, S.} (2016). Assessing it projects success with extended fuzzy cognitive maps \& neutrosophic cognitive maps in comparison to fuzzy cognitive maps. Neutrosophic Sets and Systems, 12(1), 9-19.
        \bibitem{litlink21} \textit{L.A. Zadeh} (1965) Fuzzy sets // Сайт www.sciencedirect.com (https://www.sciencedirect.com/science/article/\\pii/S001999586590241X) Просмотрено: 16 февраля 2024.
        \bibitem{litlink22} \textit{G. M. Mendez, Ismael Lopez-Juarez, P. N. Montes-Dorantes, M. A. Garcia} (2023) A New Method for the Design of Interval Type-3 Fuzzy Logic Systems With Uncertain Type-2 Non-Singleton Inputs (IT3 NSFLS-2): A Case Study in a Hot Strip Mill // Сайт ieeexplore.ieee.org (https://ieeexplore.ieee.org/document/10114383) Просмотрено: 16 февраля 2024.
    \end{thebibliography}
    \newpage
    \begin{center}
        \addcontentsline{toc}{section}{Приложения}
        \section*{Приложения}
    \end{center}
    В данном разделе представлены дополнительные материалы и ресурсы, а также результаты проведённых испытаний программы для моделирования восприятия факторов успеха IT-проекта с использованием нечетких когнитивных карт (НКК).

    \zz{}\textbf{Приложение 1\\}
    Ссылка на репозиторий проекта с исходным кодом и всеми использованными материалами.\\
    https://github.com/NikPeg/modeling\_perception\_success\_factors\\
    \zz{}\textbf{Приложение 2\\}
    Ссылка на проект интерфейса в сервисе Figma, отражающий примерную структуру будущего приложения.\\
    https://www.figma.com/file/PL5iRCOK6h7RpPK1ZqKQgE/modeling\_perception\_success\_factors?type=design\&\\node-id=0\%3A1&mode=design&t=p9Rw1aMudymyfiVe-1\\

    \zz{}\textbf{Приложение 3: Список терминов и определений}

    \begin{itemize}
        \item \textbf{Нечеткие когнитивные карты (НКК):} Метод моделирования, использующий представление знаний в виде графов, где вершины соответствуют концепциям (факторам), а дуги — воздействию одного концепта на другой, при учёте неопределенности и неточности информации.
        \item \textbf{Функциональное тестирование:} Вид тестирования, направленный на проверку корректности реализации функциональных требований программы.
        \item \textbf{Нагрузочное тестирование:} Вид тестирования, направленный на оценку производительности и устойчивости программы при различных нагрузках.
        \item \textbf{Тестирование на совместимость:} Вид тестирования, направленный на проверку корректности работы программы на различных платформах и в различных окружениях.
        \item \textbf{Тестирование безопасности:} Вид тестирования, направленный на выявление уязвимостей программы и проверку её защиты от внешних угроз.
    \end{itemize}

    \zz{}\textbf{Приложение 4: Журналы отчетов}

    \textbf{Журнал выполнения тестов}

    \begin{center}
        \begin{tabular}{|l|l|l|l|l|l|}
            \hline
            № & Описание сценария & Дата и время & Статус & Выявленные дефекты \\
            \hline
            1 & Создание нового проекта & 2023-10-01 10:00 & Успешно & - \\
            \hline
            2 & Добавление фактора & 2023-10-01 10:30 & Успешно & - \\
            \hline
            3 & Добавление связи между факторами & 2023-10-01 11:00 & Успешно & - \\
            \hline
            4 & Сохранение проекта & 2023-10-01 11:30 & Успешно & - \\
            \hline
            5 & Импорт проекта & 2023-10-01 12:00 & Успешно & - \\
            \hline
            6 & Выполнение анализа НКК & 2023-10-01 12:30 & Успешно & - \\
            \hline
            7 & Удаление сущности & 2023-10-01 13:00 & Успешно & - \\
            \hline
            8 & Нагрузочное тестирование: 50 узлов & 2023-10-02 09:00 & Успешно & - \\
            \hline
            9 & Нагрузочное тестирование: 100 узлов & 2023-10-02 10:00 & Успешно & Замедление работы \\
            \hline
            10 & Совместимость: Windows 10 & 2023-10-03 09:00 & Успешно & - \\
            \hline
            11 & Совместимость: Ubuntu 18.04 & 2023-10-03 10:00 & Успешно & Проблемы с отображением \\
            \hline
            12 & Совместимость: macOS Catalina & 2023-10-03 11:00 & Успешно & - \\
            \hline
            13 & Совместимость: Google Chrome & 2023-10-04 09:00 & Успешно & - \\
            \hline
            14 & Совместимость: Mozilla Firefox & 2023-10-04 10:00 & Успешно & Проблемы с отображением шрифтов \\
            \hline
            15 & Совместимость: Microsoft Edge & 2023-10-04 11:00 & Успешно & - \\
            \hline
            16 & Безопасность: SQL-инъекции & 2023-10-05 09:00 & Успешно & - \\
            \hline
            17 & Безопасность: XSS & 2023-10-05 10:00 & Успешно & Уязвимости XSS \\
            \hline
            18 & Безопасность: CSRF & 2023-10-05 11:00 & Успешно & - \\
            \hline
        \end{tabular}
    \end{center}
    Ответственным за все тесты является Пеганов Н. С.

    \textbf{Журнал дефектов}

    \begin{center}
        \begin{tabular}{|l|l|l|l|l|}
            \hline
            № & Описание дефекта & Шаги воспроизведения & Статус & Ответственные \\
            \hline
            1 & Замедление работы при 100 узлах & Создание проекта с 100 узлами & Закрыт & Пеганов Н. С. \\
            \hline
            2 & Проблемы с отображением (Ubuntu 18.04) & Открытие проекта & Закрыт & Пеганов Н. С. \\
            \hline
            3 & Проблемы с отображением шрифтов (Firefox) & Запуск на Mozilla Firefox & Закрыт & Пеганов Н. С. \\
            \hline
            4 & Уязвимости XSS & Ввод скрипта в поле ввода & Закрыт & Пеганов Н. С. \\
            \hline
        \end{tabular}
    \end{center}

    \textbf{Журнал изменений}

    \begin{center}
        \begin{tabular}{|l|l|l|l|}
            \hline
            Дата & Описание изменения & Причина изменения & Ответственные \\
            \hline
            2024-5-1 & Оптимизация алгоритмов & Улучшение производительности & Пеганов Н. С. \\
            \hline
            2024-5-3 & Обновление библиотек (Ubuntu 18.04) & Исправление проблем отображения & Пеганов Н. С. \\
            \hline
            2024-5-4 & Настройка параметров шрифтов (Firefox) & Исправление отображения шрифтов & Пеганов Н. С. \\
            \hline
            2024-5-5 & Дополнительные проверки ввода (XSS) & Устранение уязвимости XSS & Пеганов Н. С. \\
            \hline
        \end{tabular}
    \end{center}

    \zz{}\textbf{Приложение 5: Список используемых инструментов и ресурсов}

    \begin{itemize}
        \item \textbf{OWASP ZAP:} Инструмент для тестирования безопасности.
        \item \textbf{PyCharm/Visual Studio Code:} Среда разработки и отладки кода.
        \item \textbf{Git:} Система контроля версий для управления кодом программы.
        \item \textbf{Excel:} Инструмент для анализа данных и генерации отчетов.
    \end{itemize}

    \zz{}\textbf{Приложение 6: Контакты службы поддержки}

    В случае возникновения вопросов или необходимости консультации по методике испытаний или использованию программы, пользователи могут обратиться в службу поддержки:

    \begin{itemize}
        \item \textbf{Email:} peganov.nik@gmail.com
        \item \textbf{Телефон:} +7 (977) 744-19-23
        \item \textbf{Сайт:} \url{http://www.t.me/nikpeg}
    \end{itemize}

    Данные приложения предоставляют дополнительную информацию и результаты проведённых испытаний, что способствует полному пониманию и использованию методики испытаний программы.
\end{document}