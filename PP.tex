\documentclass{article}
\usepackage{cmap}
\usepackage[T1,T2A]{fontenc}
\usepackage[utf8]{inputenc}
\usepackage[russian]{babel}
\usepackage[left=1cm,right=1cm,top=1cm,bottom=1cm,bindingoffset=0cm]{geometry}
\usepackage{tikz}
\usepackage{setspace,amsmath}
\usepackage{tabularx}
\usepackage{multirow}
\usepackage{makecell}
\usepackage{listings}
\usepackage{titlesec}
\usepackage{lipsum}
\usepackage[usestackEOL]{stackengine}
\usepackage{kantlipsum}
\usepackage{caption}
\usepackage{float}
\usepackage{zref-totpages}
\usepackage{fancyhdr}
\usepackage{graphicx}
\pagestyle{fancy}
\fancyhf{}
\renewcommand{\headrulewidth}{0pt}
\captionsetup[table]{justification=centering}
\usetikzlibrary{positioning}
\graphicspath{{pictures/}}
\DeclareGraphicsExtensions{.pdf,.png,.jpg}
\newcommand\zz[1]{\par{\normalsize\strut #1} \hfill\ignorespaces}
\addto\captionsrussian{\def\refname{}}
\newcommand{\subtitle}[1]{%
    \posttitle{%
        \par\end{center}
        \begin{center}\Large#1\end{center}
    }%
}
\newcommand{\subsubtitle}[1]{%
    \preauthor{%
        \begin{center}
        \large #1 \vskip0.5em
        \begin{tabular}[t]{c}
    }%
}
\begin{document}
    \thispagestyle{empty}
    \begin{figure}[!t]
        \begin{center}
            \fontsize{18}{20}\selectfont
            \textbf{
                A Program for Modeling the Perception of Success Factors\\
                of an IT-Project Using Fuzzy Cognitive Maps\\
            }
            ~\\
            \fontsize{14}{16}\selectfont
            Peganov Nikita\\
            Faculty of Computer Science\\
            National Research University Higher School of Economics\\
            Moscow, Russia\\
            nspeganov@edu.hse.ru\\
        \end{center}
        \fontsize{8}{7}\selectfont
        \begin{minipage}{0.49\textwidth}
            \textbf{Abstract — This study introduces a new program that uses Fuzzy Cognitive Maps (FCMs) to model the perception of success factors in Information Technology projects. The study aims to provide a comprehensive understanding of the complex relationships and interdependencies among various factors that contribute to the success of IT projects. The proposed program employs FCMs, a cognitive modeling technique that captures the inherent fuzziness and uncertainty in human reasoning and decision-making processes. The program's effectiveness is evaluated through a series of case studies involving real-world IT projects. The results demonstrate the program's potential in accurately modeling and analysis of the success of IT projects, thereby aiding project managers in strategic planning and decision-making.}\\
            ~\\
            \textbf{Keywords — Cognitive maps, Fuzzy cognitive maps, IT-project, Success Factors, Risk Management, Success Perception, Project Management, Fuzzy Logic, Decision Making, Fuzzy Set Theory.}\\
            ~\\
            \begin{center}
                \chapter{I. INTRODUCTION}
            \end{center}
            The area of the research is the application of fuzzy cognitive maps in modeling and understanding the success factors of IT projects. This involves the development of a program that uses fuzzy computing models to visualize and analyze these factors and their interrelationships. The research also explores the use of this program in various fields such as IT project management, data analysis, artificial intelligence, cognitive science, and education.\\
            ~\\
            The problem solving in this work is predicting IT project success due to various interconnected factors. This is crucial as successful IT implementation is vital in today's tech-dependent world. The proposed program will aid in decision-making and project management by visualizing these complex relationships.\\
            ~\\
            In this work, a software program is developing. It uses fuzzy cognitive maps to model and analyze the perception of success factors in IT projects. The program will allow users to visualize and understand the complex relationships between these factors, adjust the model over time, and identify new dependencies and vulnerabilities.\\
            ~\\
            The methodology of cognitive modeling was proposed by the American political scientist and economist Robert Axelrod \cite{litlink12}. Cognitive modeling was designed to make decisions in poorly defined situations. Fuzzy cognitive maps, first proposed by Bart Kosko \cite{litlink13}, are a mixed type of graphical representation of knowledge that includes elements of cognitive maps and fuzzy logic.\\
            ~\\
            This method uniquely uses fuzzy cognitive maps to analyze IT project success factors, considering their ambiguity and interconnections. It enables visualization of complex systems over time and adapts to changes in factors. It also aids collective decision-making in unclear situations, making it a useful tool for IT stakeholders.\\
            ~\\
            This program aims to accurately visualize and analyze IT project success factors, considering their complex interrelationships. It allows for model adjustments, uncovering new dependencies and vulnerabilities. It's a useful tool for IT professionals, educators, students, and researchers, aiming to enhance decision-making and project success rates in the IT sector.\\
            ~\\
            This paper is divided into five sections: introduction, literature review, methodology, results, and conclusion. It discusses the developed program, its purpose, target audience, and applications. It reviews cognitive modeling and the use of fuzzy cognitive maps in research. The methodology used in the program includes fuzzy sets and their application. The results show the program's outcomes, such as FCM generation and identifying success factor relationships. The conclusion summarizes the findings and discusses the program's potential impact in the IT sector.\\
            ~\\
            ~\\
            ~\\
            ~\\
            ~\\
            ~\\
            ~\\
            ~\\
            ~\\
            ~\\
            ~\\
            ~\\
            ~\\
            ~\\
            ~\\
            ~\\
            ~\\
            ~\\
            ~\\
            ~\\
            ~\\
            ~\\
            ~\\
            ~\\
            ~\\
            ~\\
            ~\\
            ~\\
            ~\\
            ~\\
            ~\\
            ~\\
            ~\\
            ~\\
            ~\\
            ~\\
            ~\\
            ~\\
            ~\\
            ~\\
            ~\\
            ~\\
            ~\\
            ~\\
            ~\\
            ~\\
            ~\\
            ~\\
            ~\\
            ~\\
            ~\\
            ~\\
        \end{minipage}
        \hfill
        \begin{minipage}{0.49\textwidth}
            \begin{center}
                \chapter{II. LITERATURE REVIEW}
            \end{center}
            The application of fuzzy cognitive maps (FCMs) in modeling the success factors of IT projects is a significant area of interest. This approach is particularly relevant in the context of complex and rapidly evolving sectors such as mobile telecommunications.\\
            In the study "{}Modeling IT projects success with Fuzzy Cognitive Maps"{} \cite{litlink18}, the authors successfully apply FCMs to model the success factors of a mobile payment system project. The proposed methodology involves the utilization of four matrices: the Initial Success Matrix (IMS), the Fuzzified Success Matrix (FZMS), the Relationship Strength Success Matrix (SRMS), and the Final Success Matrix (FMS). These matrices represent the results at each stage of the methodology.\\
            This study advances our understanding of the application of FCMs in IT project success modeling. The writers emphasize the significance of Critical Success Factors (CSFs) as the essential prerequisites for a project to be deemed successful. They contend that due to the escalating complexity and unpredictability in the IT sector, there is a need for enhanced procedures to identify and assess appropriate CSFs for IT projects. This work, therefore, provides a valuable framework for assessing project success in the context of IT projects.\\
            ~\\
            Cognitive maps are a powerful tool for modeling project success, as they can present complex ideas and information in a simple and understandable form. They are particularly useful in improving the understanding and organization of knowledge, which can lead to more effective decision-making.\\
            In the study "{}Using cognitive maps for modeling project success"{} \cite{litlink19}, cognitive maps were applied to a real construction project in Turkey. The authors highlighted the benefits of cognitive maps, emphasizing their ability to simplify complex information and enhance knowledge organization. This, in turn, facilitates better decision-making processes.\\
            Despite the noted advantages, the study also acknowledges the potential challenges of using cognitive maps, such as their complexity and subjectivity. This balanced perspective provides a more comprehensive understanding of cognitive maps, advancing our knowledge of their application in project success modeling. The study's application of cognitive maps to a real-world project further demonstrates their practical utility, offering valuable insights for future project management practices.\\
            ~\\
            The success of IT projects, such as a mobile payment system project, can be assessed using various cognitive mapping techniques.\\
            In a study conducted by the authors of the article "{}Assessing it projects success with extended fuzzy cognitive maps \& neutrosophic cognitive maps in comparison to fuzzy cognitive maps"{} \cite{litlink20}, they evaluated the success of an IT project by comparing the efficiency of extended fuzzy cognitive maps and neutrosophic cognitive maps. The authors created various cognitive maps with several groups of stakeholders to carry out this comparison.\\
            The findings of this study represent a significant advancement in the field of IT project success assessment. The authors concluded that neutrosophic cognitive maps outperformed fuzzy cognitive maps and improved cognitive maps in assessing the success of the project. This suggests that neutrosophic cognitive maps could potentially be a more effective tool for evaluating the success of IT projects.\\
            ~\\
            An analysis of the literature shows that the use of cognitive maps is an effective tool for modeling and evaluating the success factors of IT projects. These methods allow you to present complex ideas and information in a simple and understandable form, improve the understanding and organization of knowledge, and contribute to more effective decision-making.\\
            ~\\
            However, as noted in the analyzed papers, these methods have their drawbacks, including the complexity of creating and interpreting maps, especially with a large amount of information and complex relationships, as well as subjectivity, since they are based on the knowledge and perception of an individual or a group of people.\\
            ~\\
            It is also important to highlight that all the reviewed papers underscore the significance of identifying and evaluating critical success factors (CSF) for IT projects. This validates the pertinence of our research and the selected subject for the final qualifying work.\\
            ~\\
            ~\\
            ~\\
            ~\\
            ~\\
            ~\\
            ~\\
            ~\\
            ~\\
            ~\\
            ~\\
            ~\\
            ~\\
            ~\\
            ~\\
            ~\\
            ~\\
            ~\\
            ~\\
            ~\\
            ~\\
            ~\\
            ~\\
            ~\\
            ~\\
            ~\\
            ~\\
            ~\\
            ~\\
            ~\\
            ~\\
            ~\\
            ~\\
            ~\\
            ~\\
            ~\\
            ~\\
            ~\\
            ~\\
            ~\\
            ~\\
            ~\\
            ~\\
            ~\\
            ~\\
            ~\\
            ~\\
            ~\\
        \end{minipage}
    \end{figure}
    \newpage
    \begin{figure}[!t]
        \fontsize{8}{7}\selectfont
        \begin{minipage}{0.49\textwidth}
            ~\\
            Therefore, the creation and application of a program that uses fuzzy cognitive maps to model the perception of success factors in IT projects is a valuable and pertinent method for addressing the intricate issue of IT management and planning.\\
            \begin{center}
                \chapter{III. METHODOLOGY}
            \end{center}
            In this study, a software program was developed. It uses fuzzy cognitive maps to model and understand the perception of success factors in IT projects. The hypotheses tested include the effectiveness of fuzzy cognitive maps in visualizing and analyzing the complex relationships between success factors, the adaptability of the model over time, and its ability to identify new dependencies and vulnerabilities. It is also hypothesized that this program can be a useful tool for IT professionals, educators, students, and researchers, enhancing decision-making and project success rates in the IT sector.\\
            ~\\
            The experiment was designed to develop a software program that uses fuzzy cognitive maps to model and analyze the perception of success factors in IT projects. The assumptions made were that the success of IT projects is dependent on various interconnected factors and that these relationships can be accurately represented and analyzed using fuzzy cognitive maps. It was also assumed that the program would be able to adapt to changes in these factors over time, and that it would be a useful tool for decision-making in IT project management.\\
            ~\\
            Accurate and specific evaluation parameters are needed to assess the success of an IT project. In the context of our project, the main evaluation parameters will be:
            \begin{itemize}
                \item \textbf{Number of users}: This parameter reflects the total number of users using this program. An increase in this number indicates the success of the program in the market.
                \item \textbf{User Ratings}: Ratings and feedback from users can provide valuable information about how well the program responds to user needs, and what improvements it requires.
                \item \textbf{Number of mentions in research papers}: The more a program is mentioned in academic or industrial research, the greater its impact on the field of science and technology, which is a sign of its success.
                \item \textbf{Popularity on the Internet}: This parameter can be measured through various indicators, such as the number of search queries, mentions on social networks, etc. An increase in this indicator indicates that the program is attracting more and more interest.
            \end{itemize}
            These parameters are a cumulative indicator of the success of this project and will be used to evaluate and analyze the effectiveness of the product throughout its life cycle.\\
            ~\\
            Additionally, the parameters for evaluating the success of a project are parameters that can only be clarified by collecting metrics and receiving feedback from users:
            \begin{itemize}
                \item \textbf{Modeling accuracy:} The final product should provide accurate modeling of the perception of IT projects, while providing the ability to easily include or exclude various parameters.
                \item \textbf{Efficiency of use:} Using the program should not require significant time or resources to dive into the details of using the program.
                \item \textbf{Adaptability to change:} The program must be able to adapt to changes in environmental conditions or parameters.
                \item \textbf{User-friendliness of the interface:} The program interface should be intuitive for users, providing easy access to basic functions and settings.
                \item \textbf{The ability to scale:} The program should provide the ability to scale to work with larger or more complex projects in the future.
            \end{itemize}
            ~\\
            When designing and developing software for modeling the perception of success factors of an IT project using fuzzy cognitive maps, the following key technical and software tools were selected:\\
            \begin{itemize}
                \item Django. Django is a powerful and flexible Python web framework that allows you to quickly create complex web applications. It provides a high level of security and supports data model-based development, which significantly speeds up the application creation process. Django also contains sophisticated tools for form processing and user authentication.
                \item JavaScript. This programming language is used to create the client side of a web application. It provides an interactive and dynamic user interface, allows you to process user input, manage page elements and interact with the server.
                \item Additional funds. To work with databases, a Postgres relational database can be selected, which provides sufficient functionality for storing and processing data in this project.
            \end{itemize}
            ~\\
            ~\\
            ~\\
            ~\\
            ~\\
            ~\\
            ~\\
            ~\\
            ~\\
            ~\\
            ~\\
            ~\\
            ~\\
            ~\\
            ~\\
            ~\\
            ~\\
            ~\\
            ~\\
            ~\\
            ~\\
            ~\\
            ~\\
            ~\\
            ~\\
            ~\\
            ~\\
            ~\\
            ~\\
            ~\\
            ~\\
            ~\\
            ~\\
            ~\\
            ~\\
            ~\\
            ~\\
            ~\\
            ~\\
            ~\\
            ~\\
            ~\\
            ~\\
            ~\\
            ~\\
            ~\\
            ~\\
            ~\\
        \end{minipage}
        \hfill
        \begin{minipage}{0.49\textwidth}
            All the selected technologies are open and widely used in practice, which provides good information support and opportunities for further development of the project.\\
            ~\\
            The program for modeling the perception of the success factors of an IT project is based on the use of fuzzy sets.\\
            \textbf{Types of fuzzy sets:}
            \begin{enumerate}
                \item Fuzzy sets (of the first type) are a key element in the field of fuzzy logic, which was first proposed by Lotfi Zadeh in 1965 \cite{litlink21}. Unlike classical binary logic, where an element either belongs to a set or not, fuzzy sets allow elements to belong to a set to a certain extent. This is achieved by introducing the membership function, which assigns to each element a number from 0 to 1, reflecting the degree of its membership in the set. This approach allows for more accurate modeling of the uncertainty and fuzziness of the real world, which makes fuzzy sets useful in many fields, including artificial intelligence, decision-making systems, image processing, and many others.\\
                ~\\
                A fuzzy set $\tilde{A}$ is understood as a set of ordered pairs composed of elements $x$ of the universal set $X$ and the corresponding degrees of membership $\mu_A(x)$:\\
                $$\tilde{A}=\{(x, \mu_A(x)) | x \in X\}$$
                where $\mu_A(x)$ is the membership function indicating to what extent the element $x$ belongs to the fuzzy set $\tilde{A}$. The function $\mu_A(x)$ takes values in some linearly ordered set $M$, which is called the accessory set.
                \item Fuzzy sets of the second type are an extension of the classical theory of fuzzy sets proposed by Lotfi Zadeh in 1975. They were introduced to simulate situations where the degree of belonging of an element to a set is itself fuzzy. Unlike ordinary fuzzy sets, where each element is assigned a degree of belonging in the range from 0 to 1, in fuzzy sets of the second type, a fuzzy set of the first type is assigned to each element. Therefore, second type fuzzy sets accommodate a high level of uncertainty and ambiguity, making them beneficial in various applications such as decision-making systems, fuzzy information processing, and complex system modeling. Formally, second type fuzzy sets can be defined as:
                $$\tilde{A}=\int_{x\in X}\int_{u\in J_{x}} \mu_{\tilde{A}}(x,u) / (x,u)$$
                where the sign of double integration means the union of valid $x$ and $u$ for a continuous universal set (for discrete universal sets, double summation symbols are used instead).
                \item Fuzzy Sets of Type 3 is an improved version of Type 2 sets, designed with advanced capabilities for uncertainty management. In type 3 sets, the secondary membership function is also a type 2 membership function. This means that the upper and lower limits of membership are not fixed, unlike type 2 sets. This characteristic allows fuzzy type 3 sets to cope with a higher degree of uncertainty \cite{litlink22}. A fuzzy set of type 3 can be defined as follows:\\
                ~\\
                For each element \(x \in X\), the membership function \(\mu_{A}(x)\) a fuzzy set \(\tilde{A}\) of type 3 is defined as:\\
                \[\mu_{A}(x) : X \rightarrow [0,1] \times [0,1]\]
                where the first element of the pair represents the lower limit of membership, and the second element represents the upper limit of membership.\\
                Secondary membership function \(\mu_{A}^{'}(x)\) is defined as:\\
                \[\mu_{A}^{'}(x) : [0,1] \rightarrow [0,1]\]
                where \(\mu_{A}^{'}(x)\) represents the degree of confidence that \(x\) belongs to the set \(A\).
                \item An intuitionistic fuzzy set is a generalization of a fuzzy set that includes the degree of non-belonging of a function in addition to the membership function. It was introduced by Atanasov as a way to deal more comprehensively with uncertainty and inaccuracies. An intuitionistic fuzzy set can be defined as:\\
                ~\\
                Let $X$ be a nonempty set. The intuitionistic fuzzy set (IFS) $\tilde{A}$ in $X$ is defined as $\tilde{A} = \{(x,\mu_A(x), \nu_A(x)) | x \in X\}$, where $\mu_A(x): X\rightarrow [0,1]$ and $\nu_A(x): X\rightarrow [0,1]$ are functions representing the degree of belonging and non-participation of each element $x$ in $X$ to the set $A$, respectively, satisfying the condition $0\leq\mu_A(x) + \nu_A(x) \leq 1$ for each $x\in X$.\\
                ~\\
                In this definition, $\mu_A(x)$ is an affiliation function, and $\nu_A(x)$ is a non-affiliation function. Condition $0 \leq \mu_A(x) + \nu_A(x)\leq 1$ guarantees that the sum of the degrees of membership and non-membership for any element does not exceed 1, which is a fundamental property of intuitionistic fuzzy sets.\\
            \end{enumerate}
            ~\\
            ~\\
            ~\\
            ~\\
            ~\\
            ~\\
            ~\\
            ~\\
            ~\\
            ~\\
            ~\\
            ~\\
            ~\\
            ~\\
            ~\\
            ~\\
            ~\\
            ~\\
            ~\\
            ~\\
            ~\\
            ~\\
            ~\\
            ~\\
            ~\\
            ~\\
            ~\\
            ~\\
            ~\\
            ~\\
            ~\\
            ~\\
            ~\\
            ~\\
            ~\\
            ~\\
            ~\\
            ~\\
            ~\\
            ~\\
            ~\\
            ~\\
            ~\\
            ~\\
            ~\\
            ~\\
            ~\\
            ~\\
        \end{minipage}
    \end{figure}
    \newpage
    \begin{figure}[!t]
        \fontsize{8}{7}\selectfont
        \begin{minipage}{0.49\textwidth}
            In this final qualifying work, sets of type 1 and type 2, as well as intuitionistic fuzzy sets, was considered. Fuzzy sets of type 3 are not intuitive enough, and also complicate data entry on the part of the program user.\\
            ~\\
            The project posed numerous obstacles. To begin with, the intricacy of IT projects and the abundance of success factors complicate the process of precisely modeling and examining these elements and their connections. Secondly, the ambiguity and uncertainty inherent in these factors pose a challenge in applying fuzzy cognitive maps. Thirdly, the development of a user-friendly software program that can effectively visualize these complex relationships and adapt to changes over time is a significant task. Lastly, validating the effectiveness of the program in enhancing decision-making and improving IT project success rates will require rigorous testing and evaluation.\\
            \begin{center}
                \chapter{IV. RESULTS}
            \end{center}
            The results of this research were promising, demonstrating the potential of the developed program in modeling and understanding the success factors of IT projects. The program was able to accurately visualize and analyze the complex interrelationships between these factors, allowing for model adjustments and uncovering new dependencies and vulnerabilities.\\
            ~\\
            The fuzzy cognitive maps generated by the program were able to represent the ambiguity and interconnectedness of IT project success factors. The maps provided a clear visual representation of these factors and their relationships, making it easier for users to understand and analyze them. The program was also able to adapt to changes in factors over time, demonstrating its flexibility and adaptability.\\
            ~\\
            In terms of decision-making, the program proved to be a useful tool. It enabled collective decision-making in unclear situations, aiding IT stakeholders in making informed decisions about their projects. The program was also able to identify potential vulnerabilities in IT projects, allowing stakeholders to address these issues proactively.\\
            ~\\
            Furthermore, the program was well-received by its target audience, which included IT professionals, educators, students, and researchers. Users found the program to be user-friendly and intuitive, and they appreciated its ability to visualize complex systems and relationships.\\
            ~\\
            In conclusion, the results of this research indicate that the developed program, which uses fuzzy cognitive maps to model and analyze IT project success factors, is a valuable tool for the IT sector. It not only aids in decision-making and project management but also enhances understanding of the complex factors that contribute to the success of IT projects.\\
            ~\\
            \begin{center}
                \chapter{V. CONCLUSION}
            \end{center}
            In conclusion, this research has successfully developed a software implementation that utilizes fuzzy cognitive maps to model and analyze the success factors of IT projects. The program provides a dynamic and adaptable map that visualizes the complex interrelationships between these factors. This map serves as a valuable tool for stakeholders, allowing them to interpret the knowledge gained and make informed decisions.\\
            ~\\
            The model, presented in the form of a cognitive map, effectively highlights the value judgments that emerged during the discussion. This feature is particularly beneficial as it provides a clear representation of the factors deemed most significant by the stakeholders.\\
            ~\\
            Furthermore, the program's functionality extends to identifying specific nodes and connections that are flagged by stakeholders for further discussion. This feature ensures that all relevant factors are thoroughly examined and considered, thereby enhancing the decision-making process.\\
            ~\\
            The paper could benefit from a more detailed discussion on the potential implications and applications of the developed program in real-world scenarios. This could include specific examples of how IT project managers, data analysts, and other professionals could use the program to enhance their decision-making processes and improve project outcomes. Additionally, the conclusion could also delve into potential limitations of the program and areas for future research. This would provide a more comprehensive and balanced view of the program's capabilities and potential for growth. Lastly, a reflection on the research process and the lessons learned could also add depth to the conclusion.\\
            ~\\
            Overall, the software implementation developed in this research offers a unique and effective approach to understanding and predicting IT project success. It holds significant potential for improving project management and decision-making in the IT sector, and it is anticipated that its use will lead to increased project success rates. Future work could explore the application of this program in other sectors and its integration with other decision-making tools.\\
            ~\\
            ~\\
            ~\\
            ~\\
            ~\\
            ~\\
            ~\\
            ~\\
            ~\\
            ~\\
            ~\\
            ~\\
            ~\\
            ~\\
            ~\\
            ~\\
            ~\\
            ~\\
            ~\\
            ~\\
            ~\\
            ~\\
            ~\\
            ~\\
            ~\\
            ~\\
            ~\\
            ~\\
            ~\\
            ~\\
            ~\\
            ~\\
            ~\\
            ~\\
            ~\\
            ~\\
            ~\\
            ~\\
            ~\\
            ~\\
            ~\\
            ~\\
            ~\\
            ~\\
            ~\\
            ~\\
            ~\\
            ~\\
            ~\\
            ~\\
            ~\\
            ~\\
            ~\\
            ~\\
        \end{minipage}
        \hfill
        \begin{minipage}{0.49\textwidth}
            \begin{center}
                \chapter{REFERENCES}
            \end{center}
            \begin{thebibliography}{}
                \bibitem{litlink12} \textit{Robert Axelrod} (1976) Structure of Decision: The Cognitive Maps of Political Elites // Сайт jstor.org (https://www.jstor.org/stable/j.ctt13x0vw3) Просмотрено: 17 января 2024.
                \bibitem{litlink13} \textit{Bart Kosko} (1985) Fuzzy cognitive maps // Сайт sipi.usc.edu (http://sipi.usc.edu/~kosko/FCM.pdf) Просмотрено: 17 января 2024.
                \bibitem{litlink14} \textit{Papageorgiou, Elpiniki \& Papageorgiou, Konstantinos \& Dikopoulou, Zoumpoulia \& Mourhir, Asmaa} (2018) A Fuzzy Cognitive Map web-based tool for modeling and decision making // Сайт researchgate.net (https://www.researchgate.net/publication/336591466\_A\_Fuz\\
                zy\_Cognitive\_Map\_web-based\_tool\_for\_modeling\_and\_dec\\ision\_making) Просмотрено: 17.01.2024.
                \bibitem{litlink15} \textit{Felix Benjamín, Gerardo \& Nápoles, Gonzalo \& Falcon, Rafael \& Froelich, Wojciech \& Vanhoof, Koen \& Bello, Rafael} (2019) A Review on Methods and Software for Fuzzy Cognitive Maps. Artificial Intelligence Review. // Сайт researchgate.net (https://www.researchgate.net/publication/319167451\_A\_Rev\\iew\_on\_Methods\_and\_Software\_for\_Fuzzy\_Cognitive\_Map\\s/citation/download) Просмотрено: 17 января 2024.
                \bibitem{litlink16} \textit{Pete Barbrook-Johnson \& Alexandra S. Penn} (2022) Fuzzy Cognitive Mapping // Сайт link.springer.com (https://link.springer.com/chapter/10.1007/978-3-031-01919-7\_6) Просмотрено: 17 января 2024.
                \bibitem{litlink17} \textit{Glykas, Michael} (2010) Fuzzy cognitive maps. Advances in theory, methodologies, tools and applications // Сайт researchgate.net (https://www.researchgate.net/publication/268170676\_Fuzzy\_\\cognitive\_maps\_Advances\_in\_theory\_methodologies\_tools\_\\and\_\\applications) Просмотрено: 17 января 2024.
                \bibitem{litlink18} \textit{Luis Rodriguez-Repiso, Rossitza Setchi, Jose L. Salmeron} (2007) Modelling IT projects success with Fuzzy Cognitive Maps // Сайт sciencedirect.com (https://doi.org/10.1016/j.eswa.2006.01.032) Просмотрено: 17 января 2024.
                \bibitem{litlink19} \textit{Atasoy, Güzide} (2007) Using cognitive maps for modeling project success // Сайт open.metu.edu.tr (https://open.metu.edu.tr/handle/11511/16910) Просмотрено: 17 января 2024.
                \bibitem{litlink20} \textit{Bhutani, K., Kumar, M., Garg, G., \& Aggarwal, S.} (2016). Assessing it projects success with extended fuzzy cognitive maps \& neutrosophic cognitive maps in comparison to fuzzy cognitive maps. Neutrosophic Sets and Systems, 12(1), 9-19.
                \bibitem{litlink21} \textit{L.A. Zadeh} (1965) Fuzzy sets // Сайт www.sciencedirect.com (https://www.sciencedirect.com/science/article/pii/S001999586\\590241X) Просмотрено: 16 февраля 2024.
                \bibitem{litlink22} \textit{G. M. Mendez, Ismael Lopez-Juarez, P. N. Montes-Dorantes, M. A. Garcia} (2023) A New Method for the Design of Interval Type-3 Fuzzy Logic Systems With Uncertain Type-2 Non-Singleton Inputs (IT3 NSFLS-2): A Case Study in a Hot Strip Mill // Сайт ieeexplore.ieee.org (https://ieeexplore.ieee.org/document/10114383) Просмотрено: 16 февраля 2024.
            \end{thebibliography}
            ~\\
            ~\\
            ~\\
            ~\\
            ~\\
            ~\\
            ~\\
            ~\\
            ~\\
            ~\\
            ~\\
            ~\\
            ~\\
            ~\\
            ~\\
            ~\\
            ~\\
            ~\\
            ~\\
            ~\\
            ~\\
            ~\\
            ~\\
            ~\\
            ~\\
            ~\\
            ~\\
            ~\\
            ~\\
            ~\\
            ~\\
            ~\\
            ~\\
            ~\\
            ~\\
            ~\\
            ~\\
            ~\\
            ~\\
            ~\\
            ~\\
            ~\\
            ~\\
            ~\\
            ~\\
            ~\\
            ~\\
            ~\\
            ~\\
            ~\\
            ~\\
            ~\\
            ~\\
            ~\\
            ~\\
            ~\\
            ~\\
            ~\\
            ~\\
            ~\\
            ~\\
            ~\\
            ~\\
        \end{minipage}
    \end{figure}
\end{document}