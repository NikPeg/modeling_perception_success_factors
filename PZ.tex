\documentclass{article}
\usepackage{cmap}
\usepackage[T1,T2A]{fontenc}
\usepackage[utf8]{inputenc}
\usepackage[russian]{babel}
\usepackage[left=2cm,right=2cm,top=2cm,bottom=2cm,bindingoffset=0cm]{geometry}
\usepackage{tikz}
\usepackage{setspace,amsmath}
\usepackage{tabularx}
\usepackage{multirow}
\usepackage{makecell}
\usepackage{listings}
\usepackage{titlesec}
\usepackage{lipsum}
\usepackage[usestackEOL]{stackengine}
\usepackage{kantlipsum}
\usepackage{caption}
\usepackage{float}
\usepackage{zref-totpages}
\usepackage{fancyhdr}
\usepackage{graphicx}
\pagestyle{fancy}
\fancyhf{}
\fancyhead[C]{\thepage\\ RU.17701729.10.03-01 ПЗ 01-1}
\renewcommand{\headrulewidth}{0pt}
\captionsetup[table]{justification=centering}
\usetikzlibrary{positioning}
\graphicspath{{pictures/}}
\DeclareGraphicsExtensions{.pdf,.png,.jpg}
\newcommand\zz[1]{\par{\normalsize\strut #1} \hfill\ignorespaces}
\addto\captionsrussian{\def\refname{}}
\newcommand{\subtitle}[1]{%
    \posttitle{%
        \par\end{center}
        \begin{center}\Large#1\end{center}
    }%
}
\newcommand{\subsubtitle}[1]{%
    \preauthor{%
        \begin{center}
        \large #1 \vskip0.5em
        \begin{tabular}[t]{c}
    }%
}
\begin{document}
    \thispagestyle{empty}
    \begin{center}
        \textbf{
            ПРАВИТЕЛЬСТВО РОССИЙСКОЙ ФЕДЕРАЦИИ\\
            НАЦИОНАЛЬНЫЙ ИССЛЕДОВАТЕЛЬСКИЙ УНИВЕРСИТЕТ\\
            «ВЫСШАЯ ШКОЛА ЭКОНОМИКИ»\\
            Факультет компьютерных наук\\
            Образовательная программа «Программная инженерия»\\
            (ВШЭ ФКН ПИ)}\\
    \end{center}
    \bigskip
    \zz{СОГЛАСОВАНО}УТВЕРЖДАЮ
    \zz{Доцент департамента}Академический руководитель
    \zz{Программной инженерии,}образовательной программы
    \zz{ФКН, к.т.н.}«Программная инженерия»
    \zz{\noindent\rule{3cm}{0.4pt} К. Ю. Дегтярёв}старший преподаватель
    \zz{«\noindent\rule{1cm}{0.4pt}»\noindent\rule{2cm}{0.4pt}20\noindent\rule{0.5cm}{0.4pt}г.}\noindent\rule{3cm}{0.4pt} Н. А. Павлочев
    \zz{}«\noindent\rule{1cm}{0.4pt}»\noindent\rule{2cm}{0.4pt}20\noindent\rule{0.5cm}{0.4pt}г.
    \begin{center}
        \topskip=0pt
        \vspace*{\fill}
        \textbf{ПРОГРАММА ДЛЯ МОДЕЛИРОВАНИЯ ВОСПРИЯТИЯ\\
        ФАКТОРОВ УСПЕХА IТ-ПРОЕКТА С ИСПОЛЬЗОВАНИЕМ\\
        НЕЧЕТКИХ КОГНИТИВНЫХ КАРТ\\
        ~\\
        ~\\
        Пояснительная записка\\
        ~\\
        ЛИСТ УТВЕРЖДЕНИЯ\\
        ~\\
        RU.17701729.10.03-01 ПЗ 01-1-ЛУ}\\
        \vspace*{\fill}
    \end{center}
    \zz{~}Исполнитель
    \zz{~}Студент группы БПИ204
    \zz{~}образовательной программы
    \zz{~}«Программная инженерия»
    \zz{~}Пеганов Никита Сергеевич
    \zz{~}\noindent\rule{3cm}{0.4pt} Н. С. Пеганов
    \zz{~}«\noindent\rule{1cm}{0.4pt}»\noindent\rule{2cm}{0.4pt}20\noindent\rule{0.5cm}{0.4pt}г.
    \begin{center}
        \vspace*{\fill}{
            Москва \the\year{}}
    \end{center}
    \newpage
    \clearpage
    \pagenumbering{arabic}
    \begin{textbf}\\
    УТВЕРЖДЕН\\
    RU.17701729.10.03-01 ПЗ 01-1-ЛУ\\
    \end{textbf}
    \bigskip
    \begin{center}
        \topskip=0pt
        \vspace*{\fill}
        \textbf{ПРОГРАММА ДЛЯ МОДЕЛИРОВАНИЯ ВОСПРИЯТИЯ\\
        ФАКТОРОВ УСПЕХА IТ-ПРОЕКТА С ИСПОЛЬЗОВАНИЕМ\\
        НЕЧЕТКИХ КОГНИТИВНЫХ КАРТ\\
        ~\\
        ~\\
        Пояснительная записка\\
        ~\\
        RU.17701729.10.03-01 ПЗ 01-1-ЛУ}\\
        ~\\
        Листов \ztotpages\\
        \vspace*{\fill}
    \end{center}
    \begin{center}
        \vspace*{\fill}{
            Москва \the\year{}}
    \end{center}
    \newpage
    \tableofcontents
    \newpage
    \section {Аннотация}
    В представленной пояснительной записке описывается работа программы "{}IT-success-factors-model.exe"{}, которая используется для моделирования восприятия факторов успеха IT-проекта с использованием метода нечетких когнитивных карт. Задачей данной программы является обеспечение возможности визуализации, анализа и понимания динамики развития IT-проектов посредством моделирования взаимного влияния ключевых факторов их успешности.\\
    ~\\
    Основные требования к содержанию и оформлению данной пояснительной записки разработаны в соответствии с:
    \begin{itemize}
        \item ГОСТ 19.101-77 Виды программ и программных документов \cite{litlink1};
        \item ГОСТ 19.102-77 Стадии разработки \cite{litlink2};
        \item ГОСТ 19.103-77 Обозначения программ и программных документов \cite{litlink3};
        \item ГОСТ 19.104-78 Основные надписи \cite{litlink4};
        \item ГОСТ 19.105-78 Общие требования к программным документам \cite{litlink5};
        \item ГОСТ 19.106-78 Требования к программным документам, выполненным печатным
        способом \cite{litlink6};
        \item ГОСТ 19.404-79 Пояснительная записка. Требования к содержанию и оформлению \cite{litlink7}.
    \end{itemize}
    Изменения к данной пояснительной записке оформляются согласно ГОСТ 19.603-78 \cite{litlink8}, ГОСТ 19.604-78 \cite{litlink9}.
    \newpage
    \section {Введение}
    \subsection {Наименование программы на русском языке}
    Программа для моделирования восприятия факторов успеха IТ-проекта с использованием нечетких когнитивных карт.
    \subsection {Наименование программы на английском языке}
    A Program for Modeling the Perception of Success Factors of an IT-Project Using Fuzzy Cognitive Maps.
    \subsection {Документы, на основании которых ведется разработка}
    Программа разработана в рамках выполнения выпускной квалификационной работы — "{}Программа для моделирования восприятия факторов успеха IТ-проекта с использованием нечетких когнитивных карт"{}, в соответствии с учебным планом 4 курса бакалавриата направления 09.03.04  «Программная инженерия» \cite{litlink10}.\\
    ~\\
    Основание для разработки — учебный план подготовки бакалавров по направлению 09.03.04 «Программная инженерия» \cite{litlink11} и утвержденная академическим руководителем программы тема дипломной работы.
    \newpage
    \section {Назначение и область применения}
    \subsection {Назначение программы}
    Общим назначением разрабатываемой программы является визуализация, анализ и понимание факторов успеха IT-проектов. Это достигается путем использования нечетких когнитивных карт, что позволяет включить в модель любые переменные (факторы), даже те, которые сложно или невозможно измерить в количественных терминах.  Основное назначение этого ПО — определение и визуализация взаимосвязей между различными факторами с точки зрения стейкхолдеров.\\
    ~\\
    Программа реализует нечеткие модели вычислений, с помощью которых аналитики могут оценивать и анализировать полученные данные, опираясь на предложенные нечеткие модели вычислений. Нечеткие когнитивные карты (Fuzzy Cognitive Maps, FCM) дают возможность моделировать одну и ту же систему по-разному, в зависимости от целей и профессиональных навыков людей или групп людей, фиксируя изменяемые во времени величины моделируемой ситуации.\\
    ~\\
    Программа генерирует FCM, которые можно использовать для визуализации сложных систем и отображения их развития во времени. При этом, в ряде случаев, применяется SWOT-анализ — это позволяет более полно охарактеризовать исследуемые факторы.\\
    ~\\
    С течением времени, могут меняться не только сами факторы, но и связи между ними. Программа позволяет учесть это, перестраивая и модифицируя карты. Это обеспечивает возможность итеративной корректировки модели и поиск новых зависимостей и уязвимостей.\\
    \subsection {Целевая аудитория продукта}
    Целевой аудиторией данной программы являются, преимущественно, специалисты, работающие в IT-секторе, а именно аналитики, менеджеры проектов и IT-директора. Это связано с тем, что программа позволяет моделировать восприятие факторов успеха IT-проектов и может быть полезной для исследования и управления различными аспектами таких проектов.\\
    ~\\
    Вместе с тем, данная программа может быть использована в обучающих целях и обладает потенциалом быть полезной для студентов и преподавателей IT-специальностей, особенно для тех, кто изучает или преподает курсы, связанные с управлением IT-проектами, анализом данных, искусственным интеллектом или когнитивной наукой.\\
    ~\\
    Наконец, потенциальными пользователями данной программы могут быть и авторы научных исследований из области IT и когнитивистики. Она может оказаться полезной при изучении восприятия факторов, влияющих на успех IT-проектов, и при исследовании механизмов принятия решений в рамках таких проектов.\\
    ~\\
    В то же время, следует отметить, что оперировать данной программой могут преимущественно люди, обладающие нужным навыком и знаниями для работы с нечеткими когнитивными картами. Подразумевается использование программы одним аналитиком и множеством стейкхолдеров для создания результата коллективного обсуждения.\\
    \subsection {Актуальность проблемы}
    В последнее время, в свете растущей зависимости бизнеса от технологий, успешное выполнение IT-проектов становится особенно важным для организаций различных сфер деятельности и масштаба. Однако, измерение и предсказание успеха в случае IT-проектов всё ещё являются сложной задачей, так как они зависят от множества факторов, характеризуемых неоднозначностью и взаимной связью с другими аспектами рассмотрения.\\
    ~\\
    В связи с этим, программа для моделирования восприятия факторов успеха IT-проектов с использованием нечетких когнитивных карт обретает значительную актуальность. Факторы успеха проекта часто являются нечетко определенными и интерпретируемыми, что делает использование нечетких когнитивных карт подходящим выбором для их анализа и моделирования.\\
    ~\\
    Методология когнитивного моделирования была предложена американским политологом и экономистом Робертом Аксельродом \cite{litlink12}. Когнитивное моделирование предназначалось для принятия решений в плохо определенных ситуациях. Нечеткие когнитивные карты, впервые предложенные Бартом Коско \cite{litlink13}, являются смешанным типом графического представления знаний, включающего в себя элементы когнитивных карт и нечеткой логики.\\
    ~\\
    В последние годы они вновь привлекают внимание исследователей, подобно тому как нейронные сети после своего «забвения» в 90-х годах 20-го века, сейчас снова переживают свой пик популярности. Например, нечеткие когнитивные карты используются в исследовательских работах, написанных в 2018, 2019 и 2022 годах \cite{litlink14, litlink15, litlink16}. Как и нейронные сети, нечеткие когнитивные карты могут быть применены для моделирования сложных взаимосвязей и получения результатов на основе неопределенной и нечеткой информации.\\
    ~\\
    В многочисленных исследовательских работах авторы рассматривают нечеткие когнитивные карты как удобный и наглядный аппарат моделирования. Факторы и связи между факторами располагаются в FCM в стуктуре, подобной структуре головного мозга человека (в очень упрощенном виде), поэтому получаемая модель легко воспринимается и удобна для обсуждения. Также нечеткие когнитивные карты отличаются универсальностью, что позволяет использовать их в множестве различных областей \cite{litlink17}.\\
    ~\\
    Моделирование факторов успеха IT-проекта — одна из таких областей. Например, в работе "{}Modelling IT projects success with Fuzzy Cognitive Maps"{} \cite{litlink18} авторы применяют FCM для моделирования факторов успеха мобильной платежной системы, проекта, связанного с быстро развивающимся миром мобильных телекоммуникаций.\\
    ~\\
    В работе "{}Using cognitive maps for modeling project success"{} \cite{litlink19}, вышедшей в том же году, применяются когнитивные карты. Для наглядности в ней рассматривается реальный строительный проект, реализованный в Турции. В работе также описаны приемущества и недостатки когнитивных карт.\\
    ~\\
    Несмотря на нейросетеподобную структуру нечетких когнитивных карт, в рамках данной выпускной квалификационной работы не предполагается использование сложных алгоритмов нейронных сетей. Главным образом, это обусловлено спецификой выбранной методологии — нечетких когнитивных карт. Данный подход предполагает создание модели с использованием сетевой структуры, раскрывающей прямые и обратные связи между различными факторами успеха IT-проекта.\\
    ~\\
    Таким образом, разработка и использование программы для моделирования восприятия факторов успеха IT-проектов с использованием нечетких когнитивных карт является полезным и актуальным подходом к решению сложной проблемы IT-управления и планирования.\\
    \subsubsection {Функциональное назначение}
    Программа предназначена для моделирования восприятия различных факторов, которые способствуют или препятствуют успеху IT-проектов. Она использует принципы нечетких когнитивных карт для преобразования качественных оценок в количественные данные, что позволяет более точно анализировать и визуализировать динамику проекта.\\
    ~\\
    Основные функции программы включают:
    \begin{itemize}
        \item Создание списка факторов, влияющих на итоговый успех проекта. Эти факторы могут быть определены пользователем или группой пользователей, что обеспечивает гибкость системы и возможность учета уникальных особенностей каждой отдельной ситуации.
        \item Ввод данных о том, как каждый фактор влияет на другие, и преобразование этих данных в формальный вид с использованием нечетких множеств.
        \item Визуализация связей между факторами и расчет относительных значений в узлах, что позволяет увидеть, какой фактор является решающим и как он влияет на общую картину.
        \item Генерация выводов на основе анализа ситуации и формирование выводов о будущем ходе проекта.
        \item Запись и чтение созданных нечетких когнитивных карт в файл.
    \end{itemize}
    Таким образом, данная программа служит инструментом для анализа и улучшения процесса управления IT-проектами, позволяя более эффективно определять стратегии развития и принимать управленческие решения.
    \subsubsection {Эксплуатационное назначение}
    Эксплуатационное назначение разработанной программы заключается в моделировании восприятия влияния различных факторов на успех IT-проекта с использованием нечетких когнитивных карт.\\
    ~\\
    Программа предназначена для:
    \begin{itemize}
        \item Представления с каждым фактором определенной характеристики, которая помогает оператору понимать важность и актуальность данного фактора для проекта в целом;
        \item Возможности наблюдения и анализа весов связей между различными факторами, что позволяет оператору определить ключевые связи и элементы в структуре проекта;
        \item Предоставления информации о динамике изменения значений различных факторов во времени, что позволяет оператору реагировать на изменения во внешнем и внутреннем окружении проекта вовремя и принимать корректировки в стратегию проекта при необходимости;
        \item Выделения наиболее значимых факторов для успеха проекта, что позволяет сфокусироваться на ключевых элементах и не тратить ресурсы на менее важные аспекты;
        \item Получения дополнительной информации для лучшего понимания тех аспектов, которые в значительной или несущественной степени влияют на успех проекта.
    \end{itemize}
    Для каждого фактора учитывается гибкость настроек его влияния, относительные значения в узлах и влияние на другие факторы. Программа позволяет преобразовать оценки влияния с использованием нечеткой логики и формальных представлений нечетких множеств.\\
    ~\\
    Результатами работы программы становятся визуализированные когнитивные карты, на основе которых можно сделать выводы о наиболее важных моментах и факторах успеха IT-проекта.\\
    ~\\
    Одним из важных преимуществ программы является возможность имитации изменения взаимодействия факторов во времени, что позволяет проследить эволюцию проекта в долгосрочной перспективе.
    \subsection {Область применения программы}
    Программа для моделирования восприятия факторов успеха IT-проекта с использованием нечетких когнитивных карт предназначена для использования IT-специалистами, управляющими и исследователями в области управления информационными технологиями.\\
    ~\\
    Основные области применения программы включают:
    \begin{itemize}
        \item Разработку и управление IT-проектами. Моделирование факторов успеха проекта помогает управляющим эффективно управлять ресурсами и контролировать процесс реализации проекта;
        \item Исследование в области IT. Использование нечетких когнитивных карт позволяет формировать более точное и объективное представление об исследуемых объектах и процессах;
        \item Образование. Программа может быть использована для обучения студентов и участников профессиональных курсов основам управления IT-проектами и технологиям моделирования;
        \item Использование в комплексе с другими методами управления и предсказания для увеличения точности анализа и принятия решений.
    \end{itemize}
    Таким образом, данная программа может быть применена в различных областях связанных с IT-технологиями, включая научные исследования, обучение, планирование и управление IT-проектами.
    \newpage
    \section {Технические характеристики}
    \subsection {Постановка задачи на разработку программы}
    Разрабатываемая программа должна быть наделена следующими функциями:
    \begin{enumerate}
        \item  Обеспечивать оператору механизмы для создания моделей влияния факторов успеха IT-проекта, основываясь на предоставленных им данных и применяя метод нечетких когнитивных карт;
        \item  Давать возможность оператору создавать свои индивидуальные модели влияния действующих факторов на успех IT-проекта;
        \item  Позволять оператору выбирать факторы IT-проекта из списка существующих, наиболее часто встречающихся факторов;
        \item  Реализовывать функционал, позволяющий вносить изменения в построенные модели, в том числе добавлять или удалять факторы, изменять взаимосвязи между факторами и так далее;
        \item  Давать возможность оператору сохранять построенные или измененные модели во внешний файл;
        \item  Предоставлять возможность оператору сохранять разные версии когнитивной карты в процессе изменения;
        \item  Обеспечивать оператору возможность анализа информации о взаимосвязях факторов в различных вариантах;
        \item  Обеспечивать оператору возможность ввода и анализа лингвистических термов и соответствующих им функций принадлежности;
        \item  Давать возможность оператору заменять значения весов когнитивной карты на формальное представление лингвистических термов;
        \item  Предоставлять оператору возможность анализа ключевой информации, связанной с когнитивной картой, которая используется для выводов о значимости и влиянии конкретных факторов на успешность IT-проекта.
    \end{enumerate}
    \subsection{Описание алгоритмов функционирования программы}
    Программа для моделирования восприятия факторов успеха IT-проекта базируется на использовании нечетких когнитивных карт.
    \subsubsection{Типы нечетких множеств}
    \begin{enumerate}
        \item Нечеткие множества (первого типа) являются ключевым элементом в области нечеткой логики, которая была впервые предложена Лотфи Заде в 1965 году \cite{litlink20}. В отличие от классической бинарной логики, где элемент либо принадлежит множеству, либо нет, нечеткие множества позволяют элементам принадлежать множеству в определенной степени. Это достигается путем введения функции принадлежности, которая каждому элементу ставит в соответствие число от 0 до 1, отражающее степень его принадлежности к множеству. Такой подход позволяет более точно моделировать неопределенность и нечеткость реального мира, что делает нечеткие множества полезными во многих областях, включая искусственный интеллект, системы принятия решений, обработку изображений и многие другие.
        \item Нечеткие множества второго типа представляют собой расширение классической теории нечетких множеств, предложенное Лотфи Заде в 1975 году. Они были введены для моделирования ситуаций, когда степень принадлежности элемента к множеству сама является нечеткой. В отличие от обычных нечетких множеств, где каждому элементу ставится в соответствие степень принадлежности в диапазоне от 0 до 1, в нечетких множествах второго типа каждому элементу ставится в соответствие нечеткое множество первого типа. Таким образом, нечеткие множества второго типа позволяют учесть большую степень неопределенности и нечеткости, что делает их полезными во многих приложениях, включая системы принятия решений, обработку нечеткой информации и моделирование сложных систем.
        \item Нечеткие множества типа 3 является усовершенствованной версией множеств типа 2, разработанной с расширенными возможностями для управления неопределенностями. В множествах типа 3 вторичная функция принадлежности также является функцией принадлежности типа 2. Это означает, что верхний и нижний пределы членства не являются фиксированными, в отличие от множеств типа 2. Эта характеристика позволяет нечетким множествам типа 3 справляться с более высокой степенью неопределенности \cite{litlink21}.
    \item \end{enumerate}
    \subsubsection{Алгоритмы дефаззификации}
    \begin{enumerate}
        \item Метод центра тяжести (COG - Center Of Gravity), который вычисляет центр тяжести функции принадлежности.
        \item Метод биссектрисы площади (Bisector Of Area - BOA), где вычисляется биссектриса площади под функцией принадлежности.
        \item Метод среднего максимума (Mean Of Maximum - MOM), где вычисляется среднее значение максимальных значений функции принадлежности.
        \item Метод максимума максимумов (Maximum Of Maximum - MOM), где выбирается максимальное из максимальных значений функции принадлежности.
        \item Метод минимума максимумов (Minimum Of Maximum - MOM), где значение функции принадлежности на множестве определяется минимальным из максимальных значений.
    \end{enumerate}
    \subsection {Описание входных и выходных данных программы}
    \subsubsection{Входные данные}
    Входными данными для программы являются пользовательские наборы факторов успеха IT-проекта и связей между ними. Фактор является представлением определенной особенности или аспекта IT-проекта, который может влиять на его успех. Связи между факторами представляют собой отношения между ними в формальной или относительной форме.\\
    ~\\
    Однако, стоит отметить, что факторы успеха IT-проекта необязательно являются входными параметрами в готовом виде. Они могут формироваться в процессе обсуждения проекта со стейкхолдерами и являются результатом командной работы нескольких человек.\\
    ~\\
    Каждый фактор успеха определяется оператором и представляет собой идентификатор и соответствующее ему имя. При добавлении факторов оператор может выбрать из списка популярных факторов, а также добавить собственный.\\
    ~\\
    Связи между факторами определяются оператором с помощью выбора двух добавленных факторов и указания степени влияния. Они представляют собой кортеж из трех элементов: исходного фактора, конечного фактора и веса связи в неформальном виде.\\
    ~\\
    Факторы и связи между ними задаются оператором в интуитивно понятном интерфейсе.
    \subsubsection{Выходные данные}
    Программа предоставляет следующие выходные данные:\\
    \begin{enumerate}
        \item  Когнитивная Карта: Программа выдает когнитивную карту, которая иллюстрирует влияние различных факторов на успех IT-проекта. Эта карта включает в себя узлы, представляющие факторы, и связи между ними, показывающие степень влияния этих факторов друг на друга.
        \item  Визуализация Карты: Программа обеспечивает средства для визуализации этой карты, позволяющие оператору и стейкхолдерам лучше понять, как факторы связаны друг с другом.
        \item  Анализ Сценариев: Программа может выполнять анализ сценариев, изменяя наличие или влияние отдельных факторов, чтобы помочь оператору предсказать, как изменения в этих факторах могут повлиять на итоговый успех IT-проекта через влияние на другие факторы модели.
        \item  Результаты Моделирования: Программа выдает итоговые результаты моделирования, демонстируя, как изменения в отдельных факторах или их комбинации могут влиять на успех IT-проекта. Это также может включать кумулятивное впечатление от всех факторов, отражающее общее состояние успешности проекта.
        \item  Экспорт Данных: Программа обеспечивает возможность экспорта выходных данных для дальнейшего анализа, отчетности или представления результатов третьим лицам. Это может быть выполнено в различных форматах, таких как CSV, PDF или HTML.
    \end{enumerate}
    \subsection {Интерфейс программы}
    Интерфейс программы включает в себя панель инструментов и область визуализации когнитивной карты.\\
    ~\\
    Панель инструментов включает в себя поля ввода для добавления факторов успеха и соответствующих связей, а также кнопки для добавления, изменения и удаления факторов и связей. оператору доступен функционал перемещения факторов на области визуализации нечеткой когнитивной карты для создания более удобного и наглядного расположения факторов.\\
    ~\\
    Область визуализации карты представляет собой интерактивное поле, на котором отображаются факторы и связи между ними. При добавлении фактора или связи в панели инструментов, они автоматически появляются на карте, предоставляя оператору непосредственную обратную связь.\\
    ~\\
    Связи между факторами подсвечиваются разными цветами в зависимости от степени связи, облегчая интерпретацию взаимного влияния факторов.\\
    ~\\
    Щелчок мыши по связи приводит к появлянию окна, в котором можно редактировать детали (характеристики) связи. Аналогично, щелчок мыши по фактору вызывает окно с редактированием характеристик выбранного фактора.\\
    ~\\
    Кнопка "{}Проанализировать карту"{} запускает алгоритм анализа карты. При этом оператор имеет возможность указать число шагов, которые должен произвести алгоритм. В любой момент выполнения алгоритма анализа оператор может остановить его работу посредством кнопки "{}Стоп"{}.\\
    ~\\
    Также оператору доступно скачивание полученной картины в форматах PNG или SVG, что облегчает дальнейшую работу с результатами моделирования.\\
    ~\\
    Все элементы интерфейса разработаны в унифицированном стиле, обеспечивая удобное и интуитивно понятное взаимодействие с программой.\\
    \subsection {Выбор технических и программных средств}
    При проектировании и разработке программного обеспечения для моделирования восприятия факторов успеха IT-проекта с использованием нечетких когнитивных карт был осуществлен выбор следующих ключевых технических и программных средств:\\
    \begin{enumerate}
        \item Django. Django является мощным и гибким веб-фреймворком на Python, который позволяет быстро создавать сложные веб-приложения. Он обеспечивает высокий уровень безопасности и поддерживает разработку на основе модели данных, что значительно ускоряет процесс создания приложения. Django также содержит сложные инструменты для обработки форм и аутентификации пользователей.
        \item JavaScript. Этот язык программирования используется для создания клиентской части веб-приложения. Он обеспечивает интерактивность и динамичность интерфейса пользователя, позволяет обрабатывать пользовательский ввод, управлять элементами страницы и взаимодействовать с сервером.
        \item Дополнительные средства. Для работы с базами данных может быть выбрана реляционная база данных Postgres, которая обеспечивает достаточный функционал для хранения и обработки данных в данном проекте.
    \end{enumerate}
    Все выбранные технологии являются открытыми и широко используются в практической деятельности, что обеспечивает хорошую информационную поддержку и возможности для дальнейшего развития проекта.
    \newpage
    \section {Ожидаемые технико-экономические показатели}
    \subsection {Предполагаемая потребность}
    Программа для моделирования восприятия факторов успеха IT-проекта с использованием нечетких когнитивных карт предназначена для обеспечения качественного и объективного анализа ключевых факторов, определяющих успех реализации IT-проекта.
    \subsubsection{Организации и ИТ-отделы}
    Основными потребителями программы могут стать ИТ-отделы различных организаций. Программа позволит учет влияния различных факторов на итоговый успех проекта, таких как качество руководства проектом, навыки и опыт команды, используемые технологии и методологии, соответствие требованиям заказчика и т.д.
    \subsubsection{Исследовательские центры и учебные учреждения}
    В учебных целях программа может использоваться в исследовательских центрах и вузах для изучения принципов моделирования и анализа факторов успеха в IT-проектах.
    \subsubsection{IT-консалтинг и аналитические компании}
    Аналитические компании и IT-консалтинговые агентства могут использовать программу для предоставления услуг по оценке и прогнозированию успешности IT-проектов на основе моделирования взаимосвязи факторов успеха.\\
    ~\\
    Таким образом, данная программа позволяет не только получить количественное и качественное представление о будущем успехе проекта, но и выявить основные пути оптимизации ресурсов и рисков.
    \subsection {Первоначальная оценка успеха проекта}
    Первоначальная оценка успеха разработанной программы для моделирования восприятия факторов успеха IТ-проекта с использованием нечетких когнитивных карт проведена по следующим критериям:
    \begin{enumerate}
        \item \textbf{Соответствие заявленному техническому заданию:} Разработанная программа должна полностью соответствовать требованиям и функционалу, описанным в техническом задании. Должна быть проработана каждая деталь, начиная от общей концепции и заканчивая отдельными элементами интерфейса.
        \item \textbf{Качество документации:} К программе должна прилагаться подробная и понятная документация, которая позволит оператору без проблем воспользоваться всеми функциями программы. Документация должна отражать все аспекты использования программы, включая описание возможных ошибок и способов их решения.
        \item \textbf{Удобство использования:} Программа должна быть удобной в использовании. Интерфейс должен быть интуитивно понятным, а возможности программы - легко доступными.
        \item \textbf{Стабильность работы:} Программа должна работать стабильно и без сбоев, вне зависимости от объема обрабатываемых данных и сложности задач.
    \end{enumerate}
    По результатам оценки по вышеуказанным критериям можно судить о первоначальном успехе проекта. При наличии значимых недостатков и отклонений от требований ТЗ должна быть проведена доработка программы и устранить выявленные проблемы.
    \subsection {Последующая оценка успеха проекта}
    Последующая оценка успеха проекта проводится по следующим критериям:
    \begin{enumerate}
        \item \textbf{Решение комиссии о дипломной работе:} Оценка комиссии является непосредственным показателем успеха проекта. Комиссия будет изучать все аспекты работы, начиная от проработанности задания до его выполнения.

        \item \textbf{Полученная оценка:} Конечная оценка на дипломную работу — важный показатель, но не единственный. Она является отражением всех сильных и слабых сторон дипломной работы, которые были замечены в ходе ее защиты.

        \item \textbf{Комментарии и оценка руководителя работы:} Научный руководитель оценивает работу как научное исследование и анализирует ее на основе его понимания предметной области и опыта проведения исследований.
    \end{enumerate}
    \subsection {Конечный параметр оценки успеха проекта}
    Для оценки успеха IT-проекта необходимы точные и конкретные параметры оценки. В контексте нашего проекта, главными параметрами оценки будут:
    \begin{itemize}
        \item \textbf{Количество пользователей}: Этот параметр отражает общее количество пользователей, использующих данную программу. Увеличение этого числа указывает на успех программы на рынке.
        \item \textbf{Оценки пользователей}: Оценки и отзывы от пользователей могут дать ценную информацию о том, насколько хорошо программа отвечает на потребности пользователей, и каких улучшений она требует.
        \item \textbf{Количество упоминаний в исследовательских работах}: Чем больше программа упоминается в академических или промышленных исследованиях, тем больше у неё влияние на сферу науки и технологии, что является признаком её успеха.
        \item \textbf{Популярность в интернете}: Этот параметр можно измерить через различные индикаторы, такие как количество поисковых запросов, упоминаний в социальных сетях и т.д. Повышение этого показателя говорит о том, что программа привлекает все больше и больше интереса.
    \end{itemize}
    Эти параметры являются совокупным показателем успеха данного проекта и будут использоваться для оценки и анализа эффективности продукта на протяжении всего его жизненного цикла.\\
    ~\\
    Дополнительно параметрами оценки успеха проекта являются параметры, которые могут быть выяснены только с помощью сбора метрик и получения обратной связи от пользователей:
    \begin{itemize}
        \item \textbf{Точность моделирования:} Итоговый продукт должен обеспечивать точное моделирование восприятия IT-проектов, при этом обеспечивая возможность легко включать или исключать различные параметры.
        \item \textbf{Эффективность использования:} Использование программы не должно требовать значительных затрат времени или ресурсов для погружения в детали использования программы.
        \item \textbf{Адаптивность к изменениям:} Программа должна быть способна адаптироваться к изменению условий или параметров внешней среды.
        \item \textbf{Удобство интерфейса:} Интерфейс программы должен быть интуитивно понятен для пользователей, обеспечивая легкий доступ к основным функциям и настройкам.
        \item \textbf{Возможность масштабирования:} Программа должна предоставлять возможность масштабирования для работы с более крупными или сложными проектами в будущем.
    \end{itemize}
    Успех проекта будет определен по достижению этих целей и учету обратной связи от пользователей для дальнейших усовершенствований.
    \subsection {Экономические преимущества разработки по сравнению с отечественными и зарубежными аналогами}
    В сравнении с отечественными и зарубежными аналогами, разрабатываемое программное обеспечение обладает рядом значительных экономических преимуществ. Прежде всего, оно предоставляется оператору бесплатно, тем самым исключая необходимость трат на его приобретение. В качестве веб-приложения, оно не требует дополнительного внедрения и поддержки, что существенно снижает затраты на эксплуатацию.\\
    ~\\
    Кроме того, на данный момент на рынке не представлено программ, специально ориентированных на моделирование восприятия IT-проектов с использованием нечетких когнитивных карт. Данное программное обеспечение является специализированным инструментом в этой области, что повышает его ценность для IT-компаний, команд разработчиков и людей, использующих программу в образовательных целях.\\
    ~\\
    Таким образом, использование данного ПО позволяет оптимизировать расходы, связанные с прогнозированием успеха IT-проектов, а также повысить точность и оперативность соответствующих аналитических работ.
    \newpage
    \section{Список использованных источников}
    \begin{thebibliography}{}
        \bibitem{litlink1} ГОСТ 19.101-77. Единая система программной документации. Термины и определения: утвержден и введен в действие Постановлением Государственного комитета стандартов Совета Министров СССР от 20 мая 1977 г. № 1268 срок введения: с 01.01.1980 г. – URL: https://www.swrit.ru/doc/espd/19.001-77.pdf (дата обращения: 01.12.2023). – Текст: электронный.
        \bibitem{litlink2} ГОСТ 19.102-77. Единая система программной документации. Термины и определения: утвержден и введен в действие Постановлением Государственного комитета стандартов Совета Министров СССР от 20 мая 1977 г. № 1268 срок введения: с 01.01.1980 г. – URL: https://www.swrit.ru/doc/espd/19.102-77.pdf (дата обращения: 01.12.2023). – Текст: электронный.
        \bibitem{litlink3} 19.103-77. Единая система программной документации. Термины и определения: утвержден и введен в действие Постановлением Государственного комитета стандартов Совета Министров СССР от 20 мая 1977 г. № 1268 срок введения: с 01.01.1980 г. – URL: https://www.swrit.ru/doc/espd/19.103-77.pdf (дата обращения: 01.12.2023). – Текст: электронный.
        \bibitem{litlink4} ГОСТ 19.104-78. Единая система программной документации. Термины и определения: утвержден и введен в действие Постановлением Государственного комитета стандартов Совета Министров СССР от 20 мая 1977 г. № 1268 срок введения: с 01.01.1980 г. – URL: https://www.swrit.ru/doc/espd/19.104-78.pdf (дата обращения: 01.12.2023). – Текст: электронный.
        \bibitem{litlink5} ГОСТ 19.105-78. Единая система программной документации. Термины и определения: утвержден и введен в действие Постановлением Государственного комитета стандартов Совета Министров СССР от 20 мая 1977 г. № 1268 срок введения: с 01.01.1980 г. – URL: https://www.swrit.ru/doc/espd/19.105-78.pdf (дата обращения: 01.12.2023). – Текст: электронный.
        \bibitem{litlink6} ГОСТ 19.106-78. Единая система программной документации. Термины и определения: утвержден и введен в действие Постановлением Государственного комитета стандартов Совета Министров СССР от 20 мая 1977 г. № 1268 срок введения: с 01.01.1980 г. – URL: https://www.swrit.ru/doc/espd/19.106-78.pdf (дата обращения: 01.12.2023). – Текст: электронный.
        \bibitem{litlink7} ГОСТ 19.404-79. Единая система программной документации. Термины и определения: утвержден и введен в действие Постановлением Государственного комитета стандартов Совета Министров СССР от 20 мая 1977 г. № 1268 срок введения: с 01.01.1980 г. – URL: https://www.swrit.ru/doc/espd/19.404-79.pdf (дата обращения: 01.12.2023). – Текст: электронный.
        \bibitem{litlink8} ГОСТ 19.603-78. Единая система программной документации. Термины и определения: утвержден и введен в действие Постановлением Государственного комитета стандартов Совета Министров СССР от 20 мая 1977 г. № 1268 срок введения: с 01.01.1980 г. – URL: https://www.swrit.ru/doc/espd/19.603-78.pdf (дата обращения: 01.12.2023). – Текст: электронный.
        \bibitem{litlink9} ГОСТ 19.404-79. Единая система программной документации. Термины и определения: утвержден и введен в действие Постановлением Государственного комитета стандартов Совета Министров СССР от 20 мая 1977 г. № 1268 срок введения: с 01.01.1980 г. – URL: https://www.swrit.ru/doc/espd/19.404-79.pdf (дата обращения: 01.12.2023). – Текст: электронный.

        \bibitem{litlink10} \textit{Учебный офис ФКН ПИ} (2023) СПРАВОЧНИК УЧЕБНОГО ПРОЦЕССА НИУ ВШЭ. Выпускная квалификационная работа (ВКР) // Сайт hse.ru (https://www.hse.ru/studyspravka/vkr) Просмотрено: 30.11.2023.
        \bibitem{litlink11} \textit{Жернова Мария Олеговна} (2023) Учебные планы 2020 года набора // Сайт hse.ru (https://www.hse.ru/ba/se/learn\_plans) Просмотрено: 12.12.2023.

        \bibitem{litlink12} \textit{Robert Axelrod} (1976) Structure of Decision: The Cognitive Maps of Political Elites // Сайт jstor.org (https://www.jstor.org/stable/j.ctt13x0vw3) Просмотрено: 17 января 2024.
        \bibitem{litlink13} \textit{Bart Kosko} (1985) Fuzzy cognitive maps // Сайт sipi.usc.edu (http://sipi.usc.edu/~kosko/FCM.pdf) Просмотрено: 17 января 2024.
        \bibitem{litlink14} \textit{Papageorgiou, Elpiniki \& Papageorgiou, Konstantinos \& Dikopoulou, Zoumpoulia \& Mourhir, Asmaa} (2018) A Fuzzy Cognitive Map web-based tool for modeling and decision making // Сайт researchgate.net (https://www.researchgate.net/publication/336591466\_A\_Fuzzy\_Cognitive\_Map\_web-based\_tool\_for\_modeling\_and\_decision\_making) Просмотрено: 17.01.2024.
        \bibitem{litlink15} \textit{Felix Benjamín, Gerardo \& Nápoles, Gonzalo \& Falcon, Rafael \& Froelich, Wojciech \& Vanhoof, Koen \& Bello, Rafael} (2019) A Review on Methods and Software for Fuzzy Cognitive Maps. Artificial Intelligence Review. // Сайт researchgate.net (https://www.researchgate.net/publication/319167451\_A\_Review\_on\_Methods\_and\_Software\_for\_Fuzzy\_Cognitive\_Maps/citation/download) Просмотрено: 17 января 2024.
        \bibitem{litlink16} \textit{Pete Barbrook-Johnson \& Alexandra S. Penn} (2022) Fuzzy Cognitive Mapping // Сайт link.springer.com (https://link.springer.com/chapter/10.1007/978-3-031-01919-7\_6) Просмотрено: 17 января 2024.
        \bibitem{litlink17} \textit{Glykas, Michael} (2010) Fuzzy cognitive maps. Advances in theory, methodologies, tools and applications // Сайт researchgate.net (https://www.researchgate.net/publication/268170676\_Fuzzy\_cognitive\_maps\_Advances\_in\_theory\_methodologies\_tools\_and\_applications) Просмотрено: 17 января 2024.
        \bibitem{litlink18} \textit{Luis Rodriguez-Repiso, Rossitza Setchi, Jose L. Salmeron} (2007) Modelling IT projects success with Fuzzy Cognitive Maps // Сайт sciencedirect.com (https://doi.org/10.1016/j.eswa.2006.01.032) Просмотрено: 17 января 2024.
        \bibitem{litlink19} \textit{Atasoy, Güzide} (2007) Using cognitive maps for modeling project success // Сайт open.metu.edu.tr (https://open.metu.edu.tr/handle/11511/16910) Просмотрено: 17 января 2024.
        \bibitem{litlink20} \textit{L.A. Zadeh} (1965) Fuzzy sets // Сайт www.sciencedirect.com (https://www.sciencedirect.com/science/article/pii/S001999586590241X) Просмотрено: 16 февраля 2024.
        \bibitem{litlink21} \textit{G. M. Mendez, Ismael Lopez-Juarez, P. N. Montes-Dorantes, M. A. Garcia} (2023) A New Method for the Design of Interval Type-3 Fuzzy Logic Systems With Uncertain Type-2 Non-Singleton Inputs (IT3 NSFLS-2): A Case Study in a Hot Strip Mill // Сайт ieeexplore.ieee.org (https://ieeexplore.ieee.org/document/10114383) Просмотрено: 16 февраля 2024.
    \end{thebibliography}
    \newpage
    \begin{center}
        \addcontentsline{toc}{section}{Приложения}
        \section*{Приложения}
    \end{center}
    \zz{}\textbf{Приложение 1\\}
    Ссылка на репозиторий проекта с исходным кодом и всеми использованными материалами.\\
    https://github.com/NikPeg/modeling\_perception\_success\_factors\\
    \zz{}\textbf{Приложение 2\\}
    Ссылка на проект интерфейса в сервисе Figma, отражающий примерную структуру будущего приложения.\\
    https://www.figma.com/\ldots\\
    \zz{}\textbf{Приложение 3\\}
    \zz{}\textbf{Терминология\\}
    \begin{enumerate}
        \item \textbf{Информационные технологии (IT)}: Термин используется для обозначения комплекса технологий, связанных с созданием, хранением, обработкой и передачей информации с помощью компьютеров и компьютерных сетей.
        \item \textbf{Когнитивные карты}: Психологический инструмент, используемый для представления знаний, представлений и восприятий. Применяются в моделировании сложных систем и проблем.
        \item \textbf{Нечеткие когнитивные карты (Fuzzy Cognitive Maps, FCM)}: Расширение обычных когнитивных карт, позволяющее представить информацию об отношениях между элементами системы в виде нечетких значений.
        \item \textbf{IT-проект}: Проект, связанный с разработкой, внедрением или поддержкой информационных систем или технологий.
        \item \textbf{Моделирование}: Процесс создания модели - упрощенного представления реального объекта или процесса с целью его исследования и оптимизации.
        \item \textbf{Факторы успеха}: Элементы или условия, которые способствуют успешной реализации проекта.
        \item \textbf{Методы анализа}: Статистические и математические инструменты, используемые для изучения и распределения данных.
        \item \textbf{Алгоритмы}: Указания или набор правил, которые следует выполнить в определенном порядке для достижения конкретного результата.
        \item \textbf{Прогнозирование}: Использование статистических и математических методов для предсказания будущих показателей на основе определенного набора данных.
        \item \textbf{Данные о проекте}: Информация, собранная в процессе выполнения проекта, которая используется для анализа и прогнозирования.
        \item \textbf{Риск-менеджмент}: Процесс, включающий идентификацию, оценку и приоритизацию рисков (определенные как комбинации их вероятности и последствий) и последующую координацию и экономическую эффективность использования ресурсов для контроля вероятности и/или влияния неприемлемых событий.
    \end{enumerate}
\end{document}