\documentclass[draft]{article}
\usepackage{cmap}
\usepackage[T1,T2A]{fontenc}
\usepackage[utf8]{inputenc}
\usepackage[russian]{babel}
\usepackage[left=2cm,right=2cm,top=2cm,bottom=2cm,bindingoffset=0cm]{geometry}
\usepackage{tikz}
\usepackage{setspace,amsmath}
\usepackage{titlesec}
\usepackage{lipsum}
\usepackage[usestackEOL]{stackengine}
\usepackage{kantlipsum}
\usepackage{graphicx}
\usepackage{caption}
\usepackage{float}
\usepackage{zref-totpages}
\usepackage{fancyhdr}
\pagestyle{fancy}
\fancyhf{}
\fancyhead[C]{\thepage\\ RU.17701729.10.03-01 ТЗ 01-1}
\renewcommand{\headrulewidth}{0pt}
\captionsetup[table]{justification=centering}
\usetikzlibrary{positioning}
\graphicspath{{pictures/}}
\DeclareGraphicsExtensions{.pdf,.png,.jpg}
\newcommand\zz[1]{\par{\normalsize\strut #1} \hfill\ignorespaces}
\addto\captionsrussian{\def\refname{}}
\newcommand{\subtitle}[1]{%
    \posttitle{%
        \par\end{center}
        \begin{center}\Large#1\end{center}
    }%
}
\newcommand{\subsubtitle}[1]{%
    \preauthor{%
        \begin{center}
        \large #1 \vskip0.5em
        \begin{tabular}[t]{c}
    }%
}
\begin{document}
    \thispagestyle{empty}
    \begin{center}
        \textbf{
            ПРАВИТЕЛЬСТВО РОССИЙСКОЙ ФЕДЕРАЦИИ\\
            НАЦИОНАЛЬНЫЙ ИССЛЕДОВАТЕЛЬСКИЙ УНИВЕРСИТЕТ\\
            «ВЫСШАЯ ШКОЛА ЭКОНОМИКИ»\\
            Факультет компьютерных наук\\
            Образовательная программа «Программная инженерия»\\
            (ВШЭ ФКН ПИ)}\\
    \end{center}
    \bigskip
    \zz{СОГЛАСОВАНО}УТВЕРЖДАЮ
    \zz{Доцент департамента}Академический руководитель
    \zz{Программной инженерии,}образовательной программы
    \zz{ФКН, к.т.н.}«Программная инженерия»
    \zz{\noindent\rule{3cm}{0.4pt} К. Ю. Дегтярёв}профессор департамента программной
    \zz{«\noindent\rule{1cm}{0.4pt}»\noindent\rule{2cm}{0.4pt}20\noindent\rule{0.5cm}{0.4pt}г.}инженерии, к.т.н.
    \zz{~}\noindent\rule{3cm}{0.4pt} В.В. Шилов
    \zz{~}«\noindent\rule{1cm}{0.4pt}»\noindent\rule{2cm}{0.4pt}20\noindent\rule{0.5cm}{0.4pt}г.
    \begin{center}
        \topskip=0pt
        \vspace*{\fill}
        \textbf{ПРОГРАММА ДЛЯ ЗАПОМИНАНИЯ ЧИСЛОВЫХ\\
        ДАННЫХ С ИСПОЛЬЗОВАНИЕМ ОСНОВНОЙ\\
        МНЕМОНИЧЕСКОЙ И ДОМИНИКАНСКОЙ СИСТЕМ\\
        ~\\
        ~\\
        Пояснительная записка\\
        ~\\
        ЛИСТ УТВЕРЖДЕНИЯ\\
        ~\\
        RU.17701729.10.03-01 ТЗ 01-1-ЛУ}\\
        \vspace*{\fill}
    \end{center}
    \zz{~}Исполнитель
    \zz{~}Студент группы БПИ204
    \zz{~}образовательной программы
    \zz{~}«Программная инженерия»
    \zz{~}Пеганов Никита Сергеевич
    \zz{~}\noindent\rule{3cm}{0.4pt} Н. С. Пеганов
    \zz{~}«\noindent\rule{1cm}{0.4pt}»\noindent\rule{2cm}{0.4pt}20\noindent\rule{0.5cm}{0.4pt}г.
    \begin{center}
        \vspace*{\fill}{
            Москва \the\year{}}
    \end{center}
    \newpage
    \clearpage
    \pagenumbering{arabic}
    \begin{textbf}\\
    УТВЕРЖДЕН\\
    RU.17701729.10.03-01 ТЗ 01-1-ЛУ\\
    \end{textbf}
    \bigskip
    \begin{center}
        \topskip=0pt
        \vspace*{\fill}
        \textbf{ПРОГРАММА ДЛЯ ЗАПОМИНАНИЯ ЧИСЛОВЫХ\\
        ДАННЫХ С ИСПОЛЬЗОВАНИЕМ ОСНОВНОЙ\\
        МНЕМОНИЧЕСКОЙ И ДОМИНИКАНСКОЙ СИСТЕМ\\
        ~\\
        ~\\
        Пояснительная записка\\
        ~\\
        RU.17701729.10.03-01 ТЗ 01-1-ЛУ}\\
        ~\\
        Листов \ztotpages\\
        \vspace*{\fill}
    \end{center}
    \begin{center}
        \vspace*{\fill}{
            Москва \the\year{}}
    \end{center}
    \newpage
    \begin{center}
        \section {Содержание}
        \tableofcontents
    \end{center}
    \newpage
    \section {Аннотация}
    \newpage
    \begin{center}
        \addcontentsline{toc}{section}{Приложения}
        \section*{Приложения}
    \end{center}
    \zz{}\textbf{Приложение 1\\}
    Ссылка на репозиторий проекта с исходным кодом и всеми использованными материалами.\\
    https://github.com/NikPeg/mnemonic\_systems\_app\\
    \zz{}\textbf{Приложение 2\\}
    Ссылка на проект интерфейса в сервисе Figma, отражающий примерную структуру будущего приложения.\\
    https://www.figma.com/file/jBcJmt0PREwHvBQRowhaHO/Mnemonic-systems?node-id=38\%3A250\&t=\\
    Q8JXDdb3HXM9gGPh-1\\
\end{document}