\documentclass{article}
\usepackage{cmap}
\usepackage[T1,T2A]{fontenc}
\usepackage[utf8]{inputenc}
\usepackage[russian]{babel}
\usepackage[left=2cm,right=2cm,top=2cm,bottom=2cm,bindingoffset=0cm]{geometry}
\usepackage{tikz}
\usepackage{setspace,amsmath}
\usepackage{tabularx}
\usepackage{multirow}
\usepackage{makecell}
\usepackage{listings}
\usepackage{titlesec}
\usepackage{lipsum}
\usepackage[usestackEOL]{stackengine}
\usepackage{kantlipsum}
\usepackage{caption}
\usepackage{float}
\usepackage{zref-totpages}
\usepackage{fancyhdr}
\usepackage{graphicx}
\pagestyle{fancy}
\fancyhf{}
\fancyhead[C]{\thepage\\ RU.17701729.10.03-01 ПЗ 01-1}
\renewcommand{\headrulewidth}{0pt}
\captionsetup[table]{justification=centering}
\usetikzlibrary{positioning}
\graphicspath{{pictures/}}
\DeclareGraphicsExtensions{.pdf,.png,.jpg}
\newcommand\zz[1]{\par{\normalsize\strut #1} \hfill\ignorespaces}
\addto\captionsrussian{\def\refname{}}
\newcommand{\subtitle}[1]{%
    \posttitle{%
        \par\end{center}
        \begin{center}\Large#1\end{center}
    }%
}
\newcommand{\subsubtitle}[1]{%
    \preauthor{%
        \begin{center}
        \large #1 \vskip0.5em
        \begin{tabular}[t]{c}
    }%
}
\begin{document}
    \thispagestyle{empty}
    \begin{center}
        \textbf{
            ПРАВИТЕЛЬСТВО РОССИЙСКОЙ ФЕДЕРАЦИИ\\
            НАЦИОНАЛЬНЫЙ ИССЛЕДОВАТЕЛЬСКИЙ УНИВЕРСИТЕТ\\
            «ВЫСШАЯ ШКОЛА ЭКОНОМИКИ»\\
            Факультет компьютерных наук\\
            Образовательная программа «Программная инженерия»\\
            (ВШЭ ФКН ПИ)}\\
    \end{center}
    \bigskip
    \zz{СОГЛАСОВАНО}УТВЕРЖДАЮ
    \zz{Доцент департамента}Академический руководитель
    \zz{Программной инженерии,}образовательной программы
    \zz{ФКН, к.т.н.}«Программная инженерия»
    \zz{\noindent\rule{3cm}{0.4pt} К. Ю. Дегтярёв}старший преподаватель
    \zz{«\noindent\rule{1cm}{0.4pt}»\noindent\rule{2cm}{0.4pt}20\noindent\rule{0.5cm}{0.4pt}г.}\noindent\rule{3cm}{0.4pt} Н. А. Павлочев
    \zz{}«\noindent\rule{1cm}{0.4pt}»\noindent\rule{2cm}{0.4pt}20\noindent\rule{0.5cm}{0.4pt}г.
    \begin{center}
        \topskip=0pt
        \vspace*{\fill}
        \textbf{ПРОГРАММА ДЛЯ МОДЕЛИРОВАНИЯ ВОСПРИЯТИЯ\\
        ФАКТОРОВ УСПЕХА IТ-ПРОЕКТА С ИСПОЛЬЗОВАНИЕМ\\
        НЕЧЕТКИХ КОГНИТИВНЫХ КАРТ\\
        ~\\
        ~\\
        Пояснительная записка\\
        ~\\
        ЛИСТ УТВЕРЖДЕНИЯ\\
        ~\\
        RU.17701729.10.03-01 ПЗ 01-1-ЛУ}\\
        \vspace*{\fill}
    \end{center}
    \zz{~}Исполнитель
    \zz{~}Студент группы БПИ204
    \zz{~}образовательной программы
    \zz{~}«Программная инженерия»
    \zz{~}Пеганов Никита Сергеевич
    \zz{~}\noindent\rule{3cm}{0.4pt} Н. С. Пеганов
    \zz{~}«\noindent\rule{1cm}{0.4pt}»\noindent\rule{2cm}{0.4pt}20\noindent\rule{0.5cm}{0.4pt}г.
    \begin{center}
        \vspace*{\fill}{
            Москва \the\year{}}
    \end{center}
    \newpage
    \clearpage
    \pagenumbering{arabic}
    \begin{textbf}\\
    УТВЕРЖДЕН\\
    RU.17701729.10.03-01 ПЗ 01-1-ЛУ\\
    \end{textbf}
    \bigskip
    \begin{center}
        \topskip=0pt
        \vspace*{\fill}
        \textbf{ПРОГРАММА ДЛЯ МОДЕЛИРОВАНИЯ ВОСПРИЯТИЯ\\
        ФАКТОРОВ УСПЕХА IТ-ПРОЕКТА С ИСПОЛЬЗОВАНИЕМ\\
        НЕЧЕТКИХ КОГНИТИВНЫХ КАРТ\\
        ~\\
        ~\\
        Пояснительная записка\\
        ~\\
        RU.17701729.10.03-01 ПЗ 01-1-ЛУ}\\
        ~\\
        Листов \ztotpages\\
        \vspace*{\fill}
    \end{center}
    \begin{center}
        \vspace*{\fill}{
            Москва \the\year{}}
    \end{center}
    \newpage
    \tableofcontents
    \newpage
    \section {Аннотация}
    В данной пояснительной записке описывается работа программы "{}IT-success-factors-model.exe"{}, которая используется для моделирования восприятия факторов успеха IT-проекта с применением метода нечетких когнитивных карт. Задачей данной программы является обеспечение возможности анализа и прогнозирования динамики развития IT-проектов посредством моделирования взаимного влияния ключевых факторов их успешности.\\
    ~\\
    Основные требования к содержанию и оформлению данной пояснительной записки разработаны в соответствии с:
    \begin{itemize}
        \item ГОСТ 19.101-77 Виды программ и программных документов \cite{litlink1};
        \item ГОСТ 19.102-77 Стадии разработки \cite{litlink2};
        \item ГОСТ 19.103-77 Обозначения программ и программных документов \cite{litlink3};
        \item ГОСТ 19.104-78 Основные надписи \cite{litlink4};
        \item ГОСТ 19.105-78 Общие требования к программным документам \cite{litlink5};
        \item ГОСТ 19.106-78 Требования к программным документам, выполненным печатным
        способом \cite{litlink6};
        \item ГОСТ 19.404-79 Пояснительная записка. Требования к содержанию и оформлению \cite{litlink7}.
    \end{itemize}
    Изменения к данной пояснительной записке оформляются согласно ГОСТ 19.603-78 \cite{litlink8}, ГОСТ 19.604-78 \cite{litlink9}.
    \newpage
    \section {Введение}
    \subsection {Наименование программы на русском языке}
    Программа для моделирования восприятия факторов успеха IТ-проекта с использованием нечетких когнитивных карт.
    \subsection {Наименование программы на английском языке}
    A Program for Modeling the Perception of Success Factors of an IT-Project Using Fuzzy Cognitive Maps.
    \subsection {Документы, на основании которых ведется разработка}
    Программа разработана в рамках выполнения выпускной квалификационной работы — "{}Программа для моделирования восприятия факторов успеха IТ-проекта с использованием нечетких когнитивных карт"{}, в соответствии с учебным планом 4 курса бакалавриата направления 09.03.04  «Программная инженерия» \cite{litlink10}.\\
    ~\\
    Основание для разработки — учебный план подготовки бакалавров по направлению 09.03.04 «Программная инженерия» \cite{litlink11} и утвержденная академическим руководителем программы тема дипломной работы.
    \newpage
    \section {Назначение и область применения}
    \subsection {Назначение программы}
    Разрабатываемая программа предназначена для моделирования восприятия факторов успеха IТ-проекта с использованием нечетких когнитивных карт. Основное назначение этого ПО — определение и визуализация взаимосвязей между различными факторами из точки зрения стейк-холдеров.\\
    ~\\
    Программа реализует нечеткие модели вычислений, с помощью которых аналитики могут оценивать и проанализировать полученные данные, опираясь на предложенные нечеткие модели вычислений. Нечеткие когнитивные карты (Fuzzy Cognitive Maps, FCM) дают возможность моделей одну и ту же систему по-разному в зависимости от целей и профессиональных навыков людей или групп людей, фиксируя изменяемые во времени величины моделируемой ситуации.\\
    ~\\
    Программа генерирует FCM, которые можно использовать для визуализации сложных систем и отображения их развития во времени. При этом, в ряде случаев, применяется SWOT-анализ — это позволяет более полно охарактеризовать исследуемые факторы.\\
    ~\\
    С течением времени, могут меняться не только сами факторы, но и связи между ними. Программа позволяет учесть это, перестраивая и модифицируя карты. Это обеспечивает возможность итеративной корректировки модели и поиск новых зависимостей и уязвимостей.\\
    \subsection {Целевая аудитория продукта}
    Целевой аудиторией данной программы для моделирования восприятия факторов успеха IТ-проекта с использованием нечетких когнитивных карт являются IТ-специалисты, исследователи и аналитики в области управления проектами, преподаватели и студенты, специализирующиеся на IТ-проектах, руководители IT-проектов, IT-менеджеры, владельцы IT-бизнеса и другие лица, интересующиеся определением и анализом факторов успеха IT-проектов. Результаты моделирования с использованием данной программы могут быть использованы при принятии решений на всех стадиях IT-проекта, а также при анализе и прогнозировании успешности IT-проектов.\\
    ~\\
    Основная целевая аудитория программы — аналитики. Программа может быть использована в качестве инструмента для сбора, записи и анализа данных о факторах успеха и их взаимодействиях во время встреч аналитиков со стейк-холдерами.\\
    \subsection {Актуальность проблемы}
    В последнее время, в свете растущей зависимости бизнеса от технологий, успешное выполнение IT-проектов становится особенно важным для организаций различного рода и размера. Однако, измерение и предсказание успеха в случае IT-проектов всё ещё являются сложной задачей, так как они зависят от множества факторов, обладающих неоднозначностью и взаимной связью.\\
    ~\\
    В связи с этим, программа для моделирования восприятия факторов успеха IT-проектов с использованием нечетких когнитивных карт обретает значительную актуальность. Факторы успеха проекта часто являются нечетко определенными и интерпретируемыми, что делает использование нечетких когнитивных карт подходящим выбором для их анализа и моделирования.\\
    ~\\
    Нечеткие когнитивные карты, впервые предложенные Коско (1986), являются смешанным типом графического представления знаний, включающего в себя элементы когнитивных карт и нечеткой логики. В последние годы они вновь привлекают внимание исследователей, подобно тому как нейронные сети после своего «забвения» в 90-х годах 20-го века, сейчас снова переживают свой пик популярности. Как и нейронные сети, нечеткие когнитивные карты могут быть применены для моделирования сложных взаимосвязей и предсказания результатов на основе неопределенной и нечеткой информации.\\
    ~\\
    Таким образом, разработка и использование программы для моделирования восприятия факторов успеха IT-проектов с использованием нечетких когнитивных карт является полезным и актуальным подходом к решению сложной проблемы IT-управления.\\
    \subsubsection {Функциональное назначение}
    Программа предназначена для моделирования восприятия различных факторов, которые способствуют или препятствуют успеху IT-проектов. Она использует принципы нечетких когнитивных карт для преобразования качественных оценок в количественные данные, что позволяет более точно анализировать и визуализировать динамику проекта.\\
    ~\\
    Основные функции программы включают:
    \begin{itemize}
        \item Создание списка факторов, влияющих на итоговый успех проекта. Эти факторы могут быть определены пользователем, что обеспечивает гибкость системы и возможность учета уникальных особенностей каждой отдельной ситуации.
        \item Ввод данных о том, как каждый фактор влияет на другие, и преобразование этих данных в формальный вид с использованием нечетких множеств.
        \item Визуализация моментов и расчет относительных значений в узлах, что позволяет увидеть, какой фактор является решающим и как он влияет на общую картину.
        \item Генерация выводов на основе анализа ситуации и предоставление прогноза о будущем ходе проекта.
    \end{itemize}
    Таким образом, данная программа служит инструментом для анализа и оптимизации процесса управления IT-проектами, позволяя более эффективно определять стратегии развития и принимать управленческие решения.
    \subsubsection {Эксплуатационное назначение}
    Эксплуатационное назначение разработанной программы заключается в моделировании восприятия влияния различных факторов на успех IT-проекта с использованием нечетких когнитивных карт.\\
    ~\\
    Программа предназначена для:
    \begin{itemize}
        \item Создания списка факторов, связанных с IT-проектом, в т.ч. универсальных и специфических;
        \item Оценки влияния каждого фактора на успех проекта по условной шкале;
        \item Определения взаимного влияния факторов и выявления наиболее значимых;
        \item Визуализации и анализа полученных результатов.
    \end{itemize}
    Для каждого фактора учитывается гибкость настроек его влияния, относительные значения в узлах и влияние на другие факторы. Программа позволяет преобразовать оценку влияния с использованием нечеткой логики и формальных представлений нечетких множеств.\\
    ~\\
    Результатами работы программы становятся визуализированные когнитивные карты, на основе которых можно сделать выводы о наиболее важных momentents и факторах успеха IT-проекта.\\
    ~\\
    Одним из важных преимуществ программы является возможность имитации изменения взаимодействия факторов со временем, что позволяет проследить эволюцию проекта в долгосрочной перспективе.
    \subsection {Область применения программы}
    Программа для моделирования восприятия факторов успеха IT-проекта с использованием нечетких когнитивных карт предназначена для использования IT-специалистами, управляющими и исследователями в области управления информационными технологиями.\\
    ~\\
    Основные области применения программы включают:
    \begin{itemize}
        \item Разработка и управление IT-проектами. Моделирование факторов успеха проекта помогает управляющим эффективно управлять ресурсами и контролировать процесс реализации проекта;
        \item Исследование в области IT. Использование нечетких когнитивных карт позволяет формировать более точное и объективное представление об исследуемых объектах и процессах;
        \item Образование. Программа может быть использована для обучения студентов и участников профессиональных курсов основам управления IT-проектами и технологиями моделирования;
        \item Использование в комплексе с другими методами управления и предсказания для увеличения точности анализа и прогнозирования.
    \end{itemize}
    Таким образом, данная программа может быть применена в различных областях связанных с IT-технологиями, включая научные исследования, обучение, планирование и управление IT-проектами.
    \newpage
    \section {Технические характеристики}
    \subsection {Постановка задачи на разработку программы}
    Разрабатываемая программа должна обладать следующими функциями:
    \begin{enumerate}
        \item Обеспечивать пользователю механизмы для создания моделей влияния факторов успеха IT-проекта, основываясь на предоставленных им данных и применяя метод нечетких когнитивных карт;
        \item  Давать возможность пользователю создавать свои индивидуальные модели влияния действующих факторов на успех IT-проекта;
        \item  Позволять пользователю выбирать факторы IT-проекта из списка существующих, наиболее часто встречающихся факторов;
        \item  Реализовывать функционал, позволяющий вносить изменения в построенные модели, в том числе добавлять или удалять факторы, изменять взаимосвязи между факторами и так далее;
        \item  Давать возможность пользователю сохранять построенные или измененные модели во внешний файл;
        \item  Обеспечивать пользователю возможность анализа информации о взаимосвязях факторов в различных вариантах;
        \item  Обеспечивать пользователю возможность ввода и анализа лингвистических термов и соответствующих им функций принадлежности;
        \item  Давать возможность пользователю заменять значения весов когнитивной карты на формальное представление лингвистических термов;
        \item  Предоставлять пользователю возможность анализа ключевой информации, связанной с когнитивной картой, которая используется для выводов о значимости и влиянии конкретных факторов на успешность IT-проекта.
    \end{enumerate}
    \subsection {Описание алгоритмов и функционирования программы}
    \subsection {Описание входных и выходных данных программы}
    \subsubsection{Входные данные}
    Входными данными для программы являются пользовательские наборы факторов успеха IT-проекта и связей между ними. Фактор является представлением определенной особенности или аспекта IT-проекта, который может влиять на его успех. Связи между факторами представляют собой отношения между ними в формальной или относительной форме.\\
    ~\\
    Каждый фактор успеха определяется пользователем и представляет собой идентификатор и соответствующее ему имя. При добавлении факторов пользователь может выбрать из списка популярных факторов, а также добавить собственный.\\
    ~\\
    Связи между факторами определяются пользователем с помощью выбора двух добавленных факторов и указания степени влияния. Они представляют собой кортеж из трех элементов: исходного фактора, конечного фактора и веса связи в неформальном виде.\\
    ~\\
    И факторы, и связи между факторами задаются пользователем в интуитивно понятном интерфейсе.
    \subsubsection{Выходные данные}
    Выходными данными программы являются нечеткая когнитивная карта и ее упрощенная версия, полученная после выполнения нескольких шагов преобразования. Каждая карта представлена в удобном для понимания формате, позволяющем человеку просматривать и анализировать связи между факторами.\\
    ~\\
    Когнитивные карты выражены в виде графа с вершинами и ребрами. Вершины соответствуют факторам успеха ИТ-проекта. Ребра связывают вершины и имеют направление, что указывает на направление влияния одного фактора на другой. Каждому ребру соответствует вес, выражающий степень влияния между связанными факторами.\\
    ~\\
    Упрощенная карта представляет собой результат применения алгоритма к входным данным для упрощения когнитивной карты. Она также представляет собой граф, в котором вершины - это факторы успеха, а ребра - связи между ними. На ребрах указывается вес связи. Однако, в отличие от полной когнитивной карты, она показывает только основные факторы и связи, отбрасывая менее важные.\\
    \subsection {Интерфейс программы}
    Интерфейс программы включает в себя панель инструментов и область визуализации когнитивной карты.\\
    ~\\
    Панель инструментов включает в себя поля ввода для добавления факторов успеха и соответствующих связей, а также кнопки для добавления, изменения и удаления факторов и связей. Пользователю доступен функционал перемещения факторов на области визуализации нечеткой когнитивной карты для создания более удобного и наглядного расположения факторов.\\
    ~\\
    Область визуализации карты представляет собой интерактивное поле, на котором отображаются факторы и связи между ними. При добавлении фактора или связи в панели инструментов, они автоматически появляются на карте, предоставляя пользователю непосредственную обратную связь.\\
    ~\\
    Связи между факторами подсвечиваются разными цветами в зависимости от степени связи, облегчая интерпретацию взаимного влияния факторов.\\
    ~\\
    Кнопка "{}Упростить карту"{} запускает алгоритм упрощения карты. При этом пользователь имеет возможность указать число шагов, которые должен произвести алгоритм. В любой момент выполнения алгоритма упрощения пользователь может остановить его работу посредством кнопки "{}Стоп"{}.\\
    ~\\
    Также пользователю доступно скачивание полученной картины в форматах PNG или SVG, что облегчает дальнейшую работу с результатами моделирования.\\
    ~\\
    Все элементы интерфейса разработаны в унифицированном стиле, обеспечивая удобное и интуитивно понятное взаимодействие с программой.\\
    \subsection {Выбор технических и программных средств}
    \newpage
    \section {Ожидаемые технико-экономические показатели}
    \subsection {Предполагаемая потребность}
    Программа для моделирования восприятия факторов успеха IT-проекта с использованием нечетких когнитивных карт предназначена для обеспечения качественного и объективного анализа ключевых факторов, определяющих успех реализации IT-проекта.
    \subsubsection{Организации и ИТ-отделы}
    Основными потребителями программы могут стать ИТ-отделы различных организаций. Программа позволит учет влияния различных факторов на итоговый успех проекта, таких как качество руководства проектом, навыки и опыт команды, используемые технологии и методологии, соответствие требованиям заказчика и т.д.
    \subsubsection{Исследовательские центры и учебные учреждения}
    В учебных целях программа может использоваться в исследовательских центрах и вузах для изучения принципов моделирования и анализа факторов успеха в IT-проектах.
    \subsubsection{IT-консалтинг и аналитические компании}
    Аналитические компании и IT-консалтинговые агентства могут использовать программу для предоставления услуг по оценке и прогнозированию успешности IT-проектов на основе моделирования взаимосвязи факторов успеха.\\
    ~\\
    Таким образом, данная программа позволяет не только получить количественное и качественное представление о будущем успехе проекта, но и выявить основные пути оптимизации ресурсов и рисков.
    \subsection {Первоначальная оценка успеха проекта}
    Первоначальная оценка успеха разработанной программы для моделирования восприятия факторов успеха IТ-проекта с использованием нечетких когнитивных карт проведена по следующим критериям:
    \begin{enumerate}
        \item \textbf{Соответствие заявленному техническому заданию:} Разработанная программа должна полностью соответствовать требованиям и функционалу, описанным в техническом задании. Должна быть проработана каждая деталь, начиная от общей концепции и заканчивая отдельными элементами интерфейса.
        \item \textbf{Качество документации:} К программе должна прилагаться подробная и понятная документация, которая позволит пользователю без проблем воспользоваться всеми функциями программы. Документация должна отражать все аспекты использования программы, включая описание возможных ошибок и способов их решения.
        \item \textbf{Удобство использования:} Программа должна быть удобной в использовании. Интерфейс должен быть интуитивно понятным, а возможности программы - легко доступными.
        \item \textbf{Стабильность работы:} Программа должна работать стабильно и без сбоев, вне зависимости от объема обрабатываемых данных и сложности задач.
    \end{enumerate}
    По результатам оценки по вышеуказанным критериям можно судить о первоначальном успехе проекта. При наличии значимых недостатков и отклонений от требований ТЗ должна быть проведена доработка программы и устранить выявленные проблемы.
    \subsection {Последующая оценка успеха проекта}
    Последующая оценка успеха проекта проводится по следующим критериям:
    \begin{enumerate}
        \item \textbf{Решение комиссии о дипломной работе:} Оценка комиссии является непосредственным показателем успеха проекта. Комиссия будет изучать все аспекты работы, начиная от проработанности задания до его выполнения.

        \item \textbf{Поставленная оценка:} Конечная оценка на дипломную работу — важный показатель, но не единственный. Она является отражением всех сильных и слабых сторон дипломной работы, которые были замечены в ходе ее защиты.

        \item \textbf{Комментарии и оценка научного руководителя:} Научный руководитель оценивает работу как научное исследование и анализирует ее на основе его понимания предметной области и опыта проведения исследований.
    \end{enumerate}
    \subsection {Конечный параметр оценки успеха проекта}
    Для оценки успеха IT-проекта необходимы точные и конкретные параметры оценки. В контексте нашего проекта, главными параметрами оценки будут:
    \begin{itemize}
        \item \textbf{Количество пользователей}: Этот параметр отражает общее количество пользователей, использующих данную программу. Увеличение этого числа указывает на успех программы на рынке.
        \item \textbf{Оценки пользователей}: Оценки и отзывы от пользователей могут дать ценную информацию о том, насколько хорошо программа отвечает на потребности пользователей, и каких улучшений она требует.
        \item \textbf{Количество упоминаний в исследовательских работах}: Чем больше программа упоминается в академических или промышленных исследованиях, тем больше у неё влияние на сферу науки и технологии, что является признаком её успеха.
        \item \textbf{Популярность в интернете}: Этот параметр можно измерить через различные индикаторы, такие как количество поисковых запросов, упоминаний в социальных сетях и т.д. Повышение этого показателя говорит о том, что программа привлекает все больше и больше интереса.
    \end{itemize}
    Эти параметры являются совокупным показателем успеха данного проекта и будут использоваться для оценки и анализа эффективности продукта на протяжении всего его жизненного цикла.\\
    ~\\
    Дополнительно параметрами оценки успеха проекта являются параметры, которые могут быть выяснены только с помощью сбора метрик и получения обратной связи от пользователей:
    \begin{itemize}
        \item \textbf{Точность моделирования:} Итоговый продукт должен обеспечивать точное моделирование восприятия IT-проектов, при этом обеспечивая возможность легко включать или исключать различные параметры.
        \item \textbf{Эффективность использования:} Использование программы не должно требовать значительных затрат времени или ресурсов.
        \item \textbf{Адаптивность к изменениям:} Программа должна быть способна адаптироваться к изменению условий или параметров внешней среды.
        \item \textbf{Удобство интерфейса:} Интерфейс программы должен быть интуитивно понятен для пользователей, обеспечивая легкий доступ к основным функциям и настройкам.
        \item \textbf{Возможность масштабирования:} Программа должна предоставлять возможность масштабирования для работы с более крупными или сложными проектами в будущем.
    \end{itemize}
    Успех проекта будет определен по достижению этих целей и учету обратной связи от пользователей для дальнейших усовершенствований.
    \subsection {Экономические преимущества разработки по сравнению с отечественными и зарубежными аналогами}
    В сравнении с отечественными и зарубежными аналогами, разрабатываемое программное обеспечение обладает рядом значительных экономических преимуществ. Прежде всего, оно предоставляется пользователю бесплатно, тем самым исключая необходимость трат на его приобретение. В качестве веб-приложения, оно не требует дополнительного внедрения и поддержки, что существенно снижает затраты на эксплуатацию.\\
    ~\\
    Кроме того, на данный момент на рынке не представлено программ, специально ориентированных на моделирование восприятия IT-проектов с использованием нечетких когнитивных карт. Данное программное обеспечение является специализированным инструментом в этой области, что повышает его ценность для IT-компаний и команд разработчиков.\\
    ~\\
    Таким образом, использование данного ПО позволяет оптимизировать расходы, связанные с прогнозированием успеха IT-проектов, а также повысить точность и оперативность соответствующих аналитических работ.
    \newpage
    \section*{Список использованных источников}
    \begin{thebibliography}{}
        \bibitem{litlink1} ГОСТ 19.101-77. Единая система программной документации. Термины и определения: утвержден и введен в действие Постановлением Государственного комитета стандартов Совета Министров СССР от 20 мая 1977 г. № 1268 срок введения: с 01.01.1980 г. – URL: https://www.swrit.ru/doc/espd/19.001-77.pdf (дата обращения: 01.12.2023). – Текст: электронный.
        \bibitem{litlink2} ГОСТ 19.102-77. Единая система программной документации. Термины и определения: утвержден и введен в действие Постановлением Государственного комитета стандартов Совета Министров СССР от 20 мая 1977 г. № 1268 срок введения: с 01.01.1980 г. – URL: https://www.swrit.ru/doc/espd/19.102-77.pdf (дата обращения: 01.12.2023). – Текст: электронный.
        \bibitem{litlink3} 19.103-77. Единая система программной документации. Термины и определения: утвержден и введен в действие Постановлением Государственного комитета стандартов Совета Министров СССР от 20 мая 1977 г. № 1268 срок введения: с 01.01.1980 г. – URL: https://www.swrit.ru/doc/espd/19.103-77.pdf (дата обращения: 01.12.2023). – Текст: электронный.
        \bibitem{litlink4} ГОСТ 19.104-78. Единая система программной документации. Термины и определения: утвержден и введен в действие Постановлением Государственного комитета стандартов Совета Министров СССР от 20 мая 1977 г. № 1268 срок введения: с 01.01.1980 г. – URL: https://www.swrit.ru/doc/espd/19.104-78.pdf (дата обращения: 01.12.2023). – Текст: электронный.
        \bibitem{litlink5} ГОСТ 19.105-78. Единая система программной документации. Термины и определения: утвержден и введен в действие Постановлением Государственного комитета стандартов Совета Министров СССР от 20 мая 1977 г. № 1268 срок введения: с 01.01.1980 г. – URL: https://www.swrit.ru/doc/espd/19.105-78.pdf (дата обращения: 01.12.2023). – Текст: электронный.
        \bibitem{litlink6} ГОСТ 19.106-78. Единая система программной документации. Термины и определения: утвержден и введен в действие Постановлением Государственного комитета стандартов Совета Министров СССР от 20 мая 1977 г. № 1268 срок введения: с 01.01.1980 г. – URL: https://www.swrit.ru/doc/espd/19.106-78.pdf (дата обращения: 01.12.2023). – Текст: электронный.
        \bibitem{litlink7} ГОСТ 19.404-79. Единая система программной документации. Термины и определения: утвержден и введен в действие Постановлением Государственного комитета стандартов Совета Министров СССР от 20 мая 1977 г. № 1268 срок введения: с 01.01.1980 г. – URL: https://www.swrit.ru/doc/espd/19.404-79.pdf (дата обращения: 01.12.2023). – Текст: электронный.
        \bibitem{litlink8} ГОСТ 19.603-78. Единая система программной документации. Термины и определения: утвержден и введен в действие Постановлением Государственного комитета стандартов Совета Министров СССР от 20 мая 1977 г. № 1268 срок введения: с 01.01.1980 г. – URL: https://www.swrit.ru/doc/espd/19.603-78.pdf (дата обращения: 01.12.2023). – Текст: электронный.
        \bibitem{litlink9} ГОСТ 19.404-79. Единая система программной документации. Термины и определения: утвержден и введен в действие Постановлением Государственного комитета стандартов Совета Министров СССР от 20 мая 1977 г. № 1268 срок введения: с 01.01.1980 г. – URL: https://www.swrit.ru/doc/espd/19.404-79.pdf (дата обращения: 01.12.2023). – Текст: электронный.

        \bibitem{litlink10} \textit{Учебный офис ФКН ПИ} (2023) СПРАВОЧНИК УЧЕБНОГО ПРОЦЕССА НИУ ВШЭ. Выпускная квалификационная работа (ВКР) // Сайт hse.ru (https://www.hse.ru/studyspravka/vkr) Просмотрено: 30.11.2023.
        \bibitem{litlink11} \textit{Жернова Мария Олеговна} (2023) Учебные планы 2020 года набора // Сайт hse.ru (https://www.hse.ru/ba/se/learn\_plans) Просмотрено: 12.12.2023.
    \end{thebibliography}
    \newpage
    \begin{center}
        \addcontentsline{toc}{section}{Приложения}
        \section*{Приложения}
    \end{center}
    \zz{}\textbf{Приложение 1\\}
    Ссылка на репозиторий проекта с исходным кодом и всеми использованными материалами.\\
    https://github.com/NikPeg/modeling\_perception\_success\_factors\\
    \zz{}\textbf{Приложение 2\\}
    Ссылка на проект интерфейса в сервисе Figma, отражающий примерную структуру будущего приложения.\\
    https://www.figma.com/\ldots\\
\end{document}