\documentclass{article}
\usepackage{cmap}
\usepackage[T1,T2A]{fontenc}
\usepackage[utf8]{inputenc}
\usepackage[russian]{babel}
\usepackage[left=2cm,right=2cm,top=2cm,bottom=2cm,bindingoffset=0cm]{geometry}
\usepackage{tikz}
\usepackage{setspace,amsmath}
\usepackage{tabularx}
\usepackage{multirow}
\usepackage{makecell}
\usepackage{listings}
\usepackage{titlesec}
\usepackage{lipsum}
\usepackage[usestackEOL]{stackengine}
\usepackage{kantlipsum}
\usepackage{caption}
\usepackage{float}
\usepackage{zref-totpages}
\usepackage{fancyhdr}
\usepackage{graphicx}
\usepackage{hyperref}
\pagestyle{fancy}
\fancyhf{}
\fancyhead[C]{\thepage\\ RU.17701729.10.03-01 РО 01-1}
\renewcommand{\headrulewidth}{0pt}
\captionsetup[table]{justification=centering}
\usetikzlibrary{positioning}
\graphicspath{{pictures/}}
\DeclareGraphicsExtensions{.pdf,.png,.jpg}
\newcommand\zz[1]{\par{\normalsize\strut #1} \hfill\ignorespaces}
\addto\captionsrussian{\def\refname{}}
\newcommand{\subtitle}[1]{%
    \posttitle{%
        \par\end{center}
        \begin{center}\Large#1\end{center}
    }%
}
\newcommand{\subsubtitle}[1]{%
    \preauthor{%
        \begin{center}
        \large #1 \vskip0.5em
        \begin{tabular}[t]{c}
    }%
}
\newcommand{\mytable}{
    \AddToShipoutPictureBG{%
        \begin{tikzpicture}[remember picture, overlay]
        \node[anchor=south west, xshift=1cm, yshift=1cm] at (current page.south west) {
            \begin{tabular}{|>{\raggedright\arraybackslash}m{0.3cm}|>{\centering\arraybackslash}m{0.3cm}|}
            \hline
            \rotatebox{90}{\textit{\textbf{Инв. № подл. }}} & \rotatebox{90}{} \\
            \hline
            \rotatebox{90}{\textit{\textbf{Подп. и дата }}} & \rotatebox{90}{} \\
            \hline
            \rotatebox{90}{\textit{\textbf{Взам. инв. № }}} & \rotatebox{90}{} \\
            \hline
            \rotatebox{90}{\textit{\textbf{Инв. № дубл. }}} & \rotatebox{90}{} \\
            \hline
            \rotatebox{90}{\textit{\textbf{Подп. и дата }}} & \rotatebox{90}{} \\
            \hline
            \end{tabular}
        };
        \end{tikzpicture}
    }
}
\begin{document}
    \thispagestyle{empty}
    \begin{center}
        \textbf{
            ПРАВИТЕЛЬСТВО РОССИЙСКОЙ ФЕДЕРАЦИИ\\
            НАЦИОНАЛЬНЫЙ ИССЛЕДОВАТЕЛЬСКИЙ УНИВЕРСИТЕТ\\
            «ВЫСШАЯ ШКОЛА ЭКОНОМИКИ»\\
            Факультет компьютерных наук\\
            Образовательная программа «Программная инженерия»\\
            (ВШЭ ФКН ПИ)}\\
    \end{center}
    \bigskip
    \zz{СОГЛАСОВАНО}УТВЕРЖДАЮ
    \zz{Доцент департамента}Академический руководитель
    \zz{Программной инженерии,}образовательной программы
    \zz{ФКН, к.т.н.}«Программная инженерия»
    \zz{\noindent\rule{3cm}{0.4pt} К. Ю. Дегтярёв}старший преподаватель
    \zz{«\noindent\rule{1cm}{0.4pt}»\noindent\rule{2cm}{0.4pt}20\noindent\rule{0.5cm}{0.4pt}г.}\noindent\rule{3cm}{0.4pt} Н. А. Павлочев
    \zz{}«\noindent\rule{1cm}{0.4pt}»\noindent\rule{2cm}{0.4pt}20\noindent\rule{0.5cm}{0.4pt}г.
    \begin{center}
        \topskip=0pt
        \vspace*{\fill}
        \textbf{ПРОГРАММА ДЛЯ МОДЕЛИРОВАНИЯ ВОСПРИЯТИЯ\\
        ФАКТОРОВ УСПЕХА IТ-ПРОЕКТА С ИСПОЛЬЗОВАНИЕМ\\
        НЕЧЕТКИХ КОГНИТИВНЫХ КАРТ\\
        ~\\
        ~\\
        Руководство оператора\\
        ~\\
        ЛИСТ УТВЕРЖДЕНИЯ\\
        ~\\
        RU.17701729.10.03-01 РО 01-1-ЛУ}\\
        \vspace*{\fill}
    \end{center}
    \zz{~}Исполнитель
    \zz{~}Студент группы БПИ204
    \zz{~}образовательной программы
    \zz{~}«Программная инженерия»
    \zz{~}Пеганов Никита Сергеевич
    \zz{~}\noindent\rule{3cm}{0.4pt} Н. С. Пеганов
    \zz{~}«\noindent\rule{1cm}{0.4pt}»\noindent\rule{2cm}{0.4pt}20\noindent\rule{0.5cm}{0.4pt}г.
    \begin{center}
        \vspace*{\fill}{
            Москва \the\year{}}
    \end{center}
    \mytable
    \newpage
    \clearpage
    \pagenumbering{arabic}
    \begin{textbf}
        \\
        УТВЕРЖДЕН\\
        RU.17701729.10.03-01 РО 01-1-ЛУ\\
    \end{textbf}
    \bigskip
    \begin{center}
        \topskip=0pt
        \vspace*{\fill}
        \textbf{ПРОГРАММА ДЛЯ МОДЕЛИРОВАНИЯ ВОСПРИЯТИЯ\\
        ФАКТОРОВ УСПЕХА IТ-ПРОЕКТА С ИСПОЛЬЗОВАНИЕМ\\
        НЕЧЕТКИХ КОГНИТИВНЫХ КАРТ\\
        ~\\
        ~\\
        Руководство оператора\\
        ~\\
        RU.17701729.10.03-01 РО 01-1-ЛУ}\\
        ~\\
        Листов \ztotpages\\
        \vspace*{\fill}
    \end{center}
    \begin{center}
        \vspace*{\fill}{
            Москва \the\year{}}
    \end{center}
    \mytable
    \newpage
    \tableofcontents
    \newpage
    \section{Введение}
    Данный документ представляет из себя руководство оператора программы для моделирования восприятия факторов успеха IT-проекта с использованием нечетких когнитивных карт. Эта программа предназначена для аналитиков, исследователей и менеджеров проектов, стремящихся глубже понять и оценить комплексные взаимодействия множества факторов, оказывающих влияние на успех IT-проектов.\\
    ~\\
    В современных условиях разработки IT-проектов, управление и моделирование факторов успеха требуют анализа множества параметров и их взаимосвязей. Традиционные методы анализа часто бывают недостаточно эффективны из-за их неспособности учитывать неопределенность и непрямые связи между факторами. Здесь на помощь приходят нечеткие когнитивные карты (НКК) — инструмент, позволяющий моделировать и анализировать сложные системы с высоким уровнем неопределенности.\\
    ~\\
    Цель данной программы — предоставить пользователю удобный и доступный инструмент для создания, редактирования и анализа нечетких когнитивных карт. С ее помощью можно выявить ключевые факторы успеха IT-проекта, определить их влияние друг на друга и построить целостную картину причинно-следственных связей.\\
    ~\\
    Задачи программы включают:\\
    \begin{itemize}
        \item Позволить пользователю создавать и настраивать нечеткие когнитивные карты с учетом специфики IT-проектов.
        \item Обеспечить удобный интерфейс для добавления и редактирования узлов, связей и весовых коэффициентов.
        \item Предоставлять инструменты для анализа НКК с целью выявления критических факторов успеха.
        \item Генерировать понятные и интерпретируемые отчеты, которые могут быть использованы для принятия управленческих решений.
    \end{itemize}
    Данное руководство поможет оператору освоить основные функции программы, начиная с установки и настройки рабочего пространства, и заканчивая продвинутыми методами моделирования и анализом результатов.
    \section{Установка и настройка}
    Данный раздел поможет вам установить и настроить программу для моделирования восприятия факторов успеха IT-проекта с использованием нечетких когнитивных карт. Следуйте приведённым инструкциям, чтобы быстро приступить к работе.

    \subsection{Системные требования}
    Перед началом установки убедитесь, что ваша система соответствует следующим требованиям:
    \begin{itemize}
        \item Операционная система: Linux или Windows
        \item Установленный Python версии 3.6 или выше
        \item Установленный менеджер пакетов pip
    \end{itemize}

    \subsection{Шаг 1: Активируйте виртуальную среду}
    Для изоляции окружения и установки всех необходимых зависимостей, необходимо активировать виртуальную среду. В командной строке выполните следующую команду:
    \subsubsection{Для Linux:}
    \begin{verbatim}
source succenv/bin/activate
    \end{verbatim}

    \subsubsection{Для Windows:}
    \begin{verbatim}
succenv\Scripts\activate
    \end{verbatim}

    \subsection{Шаг 2: Перейдите в папку django-проекта}
    После активации виртуальной среды, перейдите в директорию, содержащую ваш django-проект. Выполните команду:

    \begin{verbatim}
cd success
    \end{verbatim}

    \subsection{Шаг 3: Установите зависимости}
    Теперь установите все необходимые зависимости, указанные в файле \verb|requirements.txt|. Это можно сделать с помощью следующей команды:

    \begin{verbatim}
pip3 install -r requirements.txt
    \end{verbatim}

    \subsection{Шаг 4: Примените миграции}
    Миграции нужны для создания необходимых таблиц в базе данных. Выполните команду:

    \begin{verbatim}
python manage.py migrate
    \end{verbatim}

    \subsection{Шаг 5: Создайте суперпользователя}
    Для доступа к административной панели вам необходимо создать суперпользователя. Введите команду:

    \begin{verbatim}
python manage.py createsuperuser
    \end{verbatim}

    Следуйте инструкциям в командной строке для ввода имени пользователя, адреса электронной почты и пароля.

    \subsection{Шаг 6: Запустите сервер}
    После выполнения всех предыдущих шагов, вы можете запустить сервер для проверки корректности установки:

    \begin{verbatim}
python manage.py runserver
    \end{verbatim}

    Сервер будет доступен по адресу: \href{http://127.0.0.1:8000/}{http://127.0.0.1:8000/}.

    Теперь вы успешно установили и настроили программу. Перейдите по указанному адресу, чтобы начать использование программы для моделирования восприятия факторов успеха IT-проекта.
    \section{Основные функции и интерфейс}

    Первый экран представляет собой главную страницу программы. На ней представлены следующие элементы:

    \begin{itemize}
        \item \textbf{Вкладка About:} информация о программе.
        \item \textbf{Вкладка GitHub:} ссылка на GitHub проекта.
        \item \textbf{Вкладка Contacts:} ссылка на контакт разработчика программы, к которому можно обратиться для уточнения деталей.
        \item \textbf{Кнопка Get started:} позволяет зарегистрироваться (ввести email и пароль дважды).
        \item \textbf{Кнопка Log in:} позволяет войти в учетную запись.
    \end{itemize}

    Второй экран представляет собой список проектов. Его элементы расположены следующим образом:

    \begin{itemize}
        \item \textbf{Верхняя часть экрана:} популярные шаблоны нечетких когнитивных карт (НКК), которые могут быть использованы для создания собственных НКК.
        \item \textbf{Посередине экрана My projects:} список проектов пользователя.
        \item \textbf{Нижняя часть экрана Public projects:} общедоступные проекты, которые могут использоваться всеми.
        \item \textbf{Правая верхняя часть экрана:} кнопка \textbf{Import} для загрузки НКК из файла и кнопка \textbf{Create} для создания новой НКК (данный процесс будет описан в следующем разделе).
    \end{itemize}

    Каждый из этих экранов предоставляет пользователю доступ к основным функциям программы, обеспечивая интуитивно понятный интерфейс для выполнения ключевых задач.

    \section{Создание и редактирование нечетких когнитивных карт}

    После нажатия кнопки \textbf{Create} открывается всплывающее окно. В нём необходимо ввести имя проекта, выбрать настройки приватности и тип нечеткого множества. Затем откроется рабочая область для создания НКК.

    В левой верхней части экрана отображается название проекта и кнопка-бургер. При нажатии на неё появляются два варианта: \textbf{Save} и \textbf{Import}, чтобы сохранить НКК или загрузить из файла соответственно. В левой части экрана также расположены три кнопки: \textbf{Добавить фактор}, \textbf{Добавить связь между факторами} и \textbf{Удалить сущность}.

    \subsection{Добавление фактора}
    При нажатии на кнопку \textbf{Добавить фактор} (кнопка \textbf{плюс}) открывается всплывающее окно. В нём необходимо ввести:
    \begin{itemize}
        \item Название фактора,
        \item Начальное значение фактора (используя лингвистическую переменную — от \textit{Excellent} до \textit{Very poor}),
        \item Выбрать тип функции активации.
    \end{itemize}

    Фактор появится на рабочей области в виде точки соответствующего цвета.

    \subsection{Добавление связи между факторами}
    Чтобы добавить связь между факторами, нажмите на кнопку \textbf{стрелка}. Появится всплывающее окно, в котором можно задать:
    \begin{itemize}
        \item Степень влияния одного фактора на другой,
        \item Исходный фактор,
        \item Конечный фактор.
    \end{itemize}

    Учтите, что фактор может влиять сам на себя.

    \subsection{Удаление сущности}
    Чтобы удалить сущность, нажмите кнопку \textbf{ножницы}. В всплывающем окне выберите сущность, которую нужно удалить (фактор или связь). Затем нажмите \textbf{Delete}.
    ~\\
    Каждая из этих функций обеспечивает удобство и гибкость при создании и редактировании нечетких когнитивных карт, направленных на моделирование факторов успеха IT-проекта.
    \section{Анализ и моделирование}

    В правой верхней части рабочей области находится контейнер с несколькими полями ввода:
    \begin{itemize}
        \item $\alpha$ — коэффициент сглаживания,
        \item $\epsilon$ — порог сходимости,
        \item Функция $f$ — функция активации,
        \item \textit{Number of steps} — максимальное количество итераций.
    \end{itemize}

    Справа от контейнера расположена кнопка \textbf{play} в виде треугольника. При её нажатии начнётся выполнение алгоритма анализа НКК (reasoning process).

    После завершения выполнения алгоритма появится всплывающее окно с сообщением: \textit{Выполнение алгоритма завершено}. При нажатии на кнопку \textbf{Ok} начнётся скачивание отчёта о результатах работы алгоритма в формате \textit{Excel}.

    Этот процесс анализа НКК обеспечивает пользователю возможность провести моделирование и анализ причинно-следственных связей среди факторов успеха IT-проекта, а также получить детализированный отчёт для дальнейшего изучения и принятия решений.
    \section{Отчёт о результатах работы алгоритма}

    Программа предоставляет подробный отчёт о результатах работы алгоритма, который включает:
    \begin{itemize}
        \item \textbf{Матрица весов факторов:} Отражает степень влияния одного фактора на другой. Помогает понять, какие факторы имеют наибольшее влияние на успех проекта.
        \item \textbf{Анализ критических факторов успеха (CSF):} Идентифицирует наиболее значимые факторы, влияющие на успех проекта, позволяя концентрировать усилия на этих аспектах.
        \item \textbf{Влияние отдельных факторов:} Оценивает влияние каждого фактора на общий успех проекта, помогая определить, какие факторы нуждаются в усилении или изменении.
        \item \textbf{Итерационные вычисления:} Отображает процесс итеративных вычислений, показывающий изменения в системе при изменении одного из факторов и симулирующий различные сценарии.
        \item \textbf{Конвергенция карты:} Уровень стабилизации модели после нескольких итераций, показывающий, насколько система приближается к равновесию и оценивающий устойчивость предложенной модели.
    \end{itemize}

    Этот отчёт может быть использован при дальнейшем обсуждении результатов со стейкхолдерами, что способствует более эффективному управлению и планированию IT-проектов.\\
    ~\\
    Таким образом, разработанное веб-приложение предоставляет пользователям инструмент для моделирования и анализа факторов успеха IT-проектов, что подтверждает его полезность и актуальность для решения поставленных задач.\\

    \section{Примеры использования}

    Этот раздел предоставляет примеры использования программы для моделирования восприятия факторов успеха IT-проекта с помощью нечетких когнитивных карт (НКК). Эти примеры помогут вам понять, как применять программу на практике и максимально использовать её возможности.

    \subsection{Пример 1: Анализ влияния команды на успех проекта}

    \textbf{Описание задачи:} Руководитель проекта хочет понять, как различные параметры команды (например, опыт, сотрудничество и мотивация) влияют на успех IT-проекта.

    \textbf{Шаги:}
    \begin{enumerate}
        \item Создайте новый проект, используя кнопку \textbf{Create}, введите имя проекта, настройки приватности и тип нечеткого множества.
        \item Добавьте факторы: опыт команды, уровень сотрудничества и мотивация.
        \item Установите начальные значения факторов с помощью лингвистических переменных (например, от \textit{Excellent} до \textit{Very poor}).
        \item Добавьте связи между факторами, указывая степень их влияния друг на друга и на общий успех проекта.
        \item Задайте параметр $\alpha = 0.8$, $\epsilon = 0.01$, функцию активации $f$ и максимальное количество итераций (\textit{Number of steps}) равное 100.
        \item Запустите анализ, нажав кнопку \textbf{play}.
        \item После завершения анализа скачайте отчёт в формате \textit{Excel} и изучите результаты.
    \end{enumerate}

    \subsection{Пример 2: Влияние технологий на проект}

    \textbf{Описание задачи:} Менеджер по разработке хочет оценить, как выбор технологий (например, языки программирования, платформы и инструменты) влияет на продуктивность команды и качество продукта.

    \textbf{Шаги:}
    \begin{enumerate}
        \item Создайте новый проект и введите необходимые параметры.
        \item Добавьте факторы: язык программирования, платформа и инструменты.
        \item Установите начальные значения для каждого фактора.
        \item Добавьте связи между факторами, указывая, как они влияют на продуктивность команды и качество продукта.
        \item Настройте параметры анализа ($\alpha$, $\epsilon$, $f$, \textit{Number of steps}).
        \item Выполните анализ, используя кнопку \textbf{play}.
        \item Изучите отчёт в формате \textit{Excel} для понимания взаимосвязей.
    \end{enumerate}

    \subsection{Пример 3: Оценка рисков проекта}

    \textbf{Описание задачи:} Аналитик рисков хочет построить модель для оценки того, как различные риски (например, задержки поставок, недостаток ресурсов, недостаток квалификации) могут повлиять на успех проекта.

    \textbf{Шаги:}
    \begin{enumerate}
        \item Создайте новый проект с соответствующими настройками.
        \item Добавьте факторы: задержки поставок, недостаток ресурсов и недостаток квалификации.
        \item Установите начальные значения факторов и их влияния друг на друга.
        \item Добавьте связи и настроите параметры анализа ($\alpha$, $\epsilon$, $f$, \textit{Number of steps}).
        \item Запустите процесс анализа, нажав кнопку \textbf{play}.
        \item Скачайте и изучите отчёт для выявления критических рисков и их воздействия на проект.
    \end{enumerate}

    Эти примеры показывают, как программа может быть использована для моделирования и анализа различных аспектов IT-проектов, предоставляя ценную информацию для принятия обоснованных решений.

    \section{Использование результатов программы стейкхолдерами}

    Разработанное веб-приложение предоставляет пользователям ценную информацию о факторах, влияющих на успех IT-проектов. Важно подчеркнуть, что программа не делает самостоятельных выводов, а лишь предоставляет значения и результаты анализа. Конечное решение всегда остаётся за стейкхолдерами, которые интерпретируют данные и принимают управленческие решения на основе представленных результатов. Ниже описаны этапы и рекомендации, которые помогут стейкхолдерам более эффективно использовать результаты программы.

    \subsection{Этапы использования результатов программы}

    \begin{itemize}
        \item \textbf{Получение результатов анализа:}
        После моделирования факторов успеха с использованием нечеткой когнитивной карты и запуска процесса рассуждения пользователю предоставляется подробный отчет о результатах анализа. Этот отчет включает в себя матрицу весов факторов, критические факторы успеха, влияние отдельных факторов, итерационные вычисления и графическую визуализацию.
        \item \textbf{Первичный анализ результатов:}
        Стейкхолдеры проводит первичный анализ предоставленных данных, обращая особое внимание на критические факторы успеха и взаимосвязи между ними. Это позволяет выявить ключевые аспекты, требующие внимания.
        \item \textbf{Обсуждение результатов:}
        После первичного анализа стейкхолдеры должны провести обсуждение результатов. Это обеспечивает всесторонний обзор представленных данных и совместное понимание выявленных факторов и их взаимосвязей.
    \end{itemize}

    \subsection{Рекомендации по интерпретации результатов}

    \begin{itemize}
        \item \textbf{Погружение в контекст:}
        Прежде чем начать интерпретацию результатов, стейкхолдерам следует погрузиться в контекст проекта, чтобы лучше понимать характеристики каждой переменной и ее роль в проекте. Это поможет избежать неверной интерпретации значений.
        \item \textbf{Использование визуализаций:}
        Графическая визуализация нечеткой когнитивной карты помогает наглядно представить взаимосвязи между факторами. Стейкхолдеры должны обратить внимание на узлы с наибольшими весами и их связи, что поможет в выявлении критических точек.
        \item \textbf{Обсуждение сценариев:}
        Совместное обсуждение различных сценариев с участием всех заинтересованных сторон помогает понять, как изменения в одном факторе могут повлиять на общий успех проекта. Это способствует лучшему пониманию и подготовке к возможным рискам.
        \item \textbf{Экспертное мнение:}
        Привлечение экспертов для интерпретации сложных взаимосвязей и анализа предоставленных данных поможет получить более точные и обоснованные выводы.
        \item \textbf{Дополнительные анализы:}
        Использование других методов анализа и инструментов, таких как SWOT-анализ или PEST-анализ, может дополнительно подтвердить или опровергнуть результаты, предоставленные программой.
    \end{itemize}

    Таким образом, использование разработанного веб-приложения и интерпретация его результатов требует совместной работы стейкхолдеров и экспертов. Программа предоставляет ценную информацию, которая может служить основой для дальнейших управленческих решений. Важно помнить, что конечные выводы и действия должны основываться на совокупности представленных данных и экспертного мнения, чтобы обеспечить успех IT-проекта.
    \section{Заключение}

    В данном руководстве были рассмотрены основные возможности и функции программы для моделирования восприятия факторов успеха IT-проекта с использованием нечетких когнитивных карт (НКК). Процесс установки и настройки, ознакомление с интерфейсом и базовыми функциями, а также этапы создания и редактирования НКК были подробно описаны.\\
    ~\\
    Нечеткие когнитивные карты предоставляют мощный инструмент для анализа сложных систем с множеством взаимодействующих компонентов, что позволяет лучше понимать и управлять факторами, влияющими на успех IT-проектов. Программа позволяет легко создавать и настраивать НКК, моделировать реальные сценарии и анализировать полученные данные.\\
    ~\\
    Были рассмотрены примеры использования программы для различных сценариев: анализ влияния команды на успех проекта, влияние технологий на продуктивность и качество продукта, оценка рисков проекта. Данные примеры демонстрируют, как программа может быть применена на практике для решения реальных задач и принятия обоснованных решений.\\
    ~\\
    Использование программы способствует более эффективному управлению проектами, предсказанию возможных проблем и принятию решений на основе детализированного анализа данных. Надеемся, что данное руководство станет полезным инструментом для освоения программы и поможет в достижении поставленных целей.\\
    ~\\
    В случае возникновения вопросов или предложений по улучшению программы, пожалуйста, свяжитесь с разработчиком через вкладку \textbf{Contacts} на главной странице программы. Обратная связь всегда приветствуется для дальнейшего совершенствования продукта.\\
    \section{Список использованных источников}
    \begin{thebibliography}{}
        \bibitem{litlink1} ГОСТ 19.101-77. Единая система программной документации. Термины и определения: утвержден и введен в действие Постановлением Государственного комитета стандартов Совета Министров СССР от 20 мая 1977 г. № 1268 срок введения: с 01.01.1980 г. – URL: https://www.swrit.ru/doc/espd/19.001-77.pdf (дата обращения: 01.12.2023). – Текст: электронный.
        \bibitem{litlink2} ГОСТ 19.102-77. Единая система программной документации. Термины и определения: утвержден и введен в действие Постановлением Государственного комитета стандартов Совета Министров СССР от 20 мая 1977 г. № 1268 срок введения: с 01.01.1980 г. – URL: https://www.swrit.ru/doc/espd/19.102-77.pdf (дата обращения: 01.12.2023). – Текст: электронный.
        \bibitem{litlink3} 19.103-77. Единая система программной документации. Термины и определения: утвержден и введен в действие Постановлением Государственного комитета стандартов Совета Министров СССР от 20 мая 1977 г. № 1268 срок введения: с 01.01.1980 г. – URL: https://www.swrit.ru/doc/espd/19.103-77.pdf (дата обращения: 01.12.2023). – Текст: электронный.
        \bibitem{litlink4} ГОСТ 19.104-78. Единая система программной документации. Термины и определения: утвержден и введен в действие Постановлением Государственного комитета стандартов Совета Министров СССР от 20 мая 1977 г. № 1268 срок введения: с 01.01.1980 г. – URL: https://www.swrit.ru/doc/espd/19.104-78.pdf (дата обращения: 01.12.2023). – Текст: электронный.
        \bibitem{litlink5} ГОСТ 19.105-78. Единая система программной документации. Термины и определения: утвержден и введен в действие Постановлением Государственного комитета стандартов Совета Министров СССР от 20 мая 1977 г. № 1268 срок введения: с 01.01.1980 г. – URL: https://www.swrit.ru/doc/espd/19.105-78.pdf (дата обращения: 01.12.2023). – Текст: электронный.
        \bibitem{litlink6} ГОСТ 19.106-78. Единая система программной документации. Термины и определения: утвержден и введен в действие Постановлением Государственного комитета стандартов Совета Министров СССР от 20 мая 1977 г. № 1268 срок введения: с 01.01.1980 г. – URL: https://www.swrit.ru/doc/espd/19.106-78.pdf (дата обращения: 01.12.2023). – Текст: электронный.
        \bibitem{litlink7} ГОСТ 19.404-79. Единая система программной документации. Термины и определения: утвержден и введен в действие Постановлением Государственного комитета стандартов Совета Министров СССР от 20 мая 1977 г. № 1268 срок введения: с 01.01.1980 г. – URL: https://www.swrit.ru/doc/espd/19.404-79.pdf (дата обращения: 01.12.2023). – Текст: электронный.
        \bibitem{litlink8} ГОСТ 19.603-78. Единая система программной документации. Термины и определения: утвержден и введен в действие Постановлением Государственного комитета стандартов Совета Министров СССР от 20 мая 1977 г. № 1268 срок введения: с 01.01.1980 г. – URL: https://www.swrit.ru/doc/espd/19.603-78.pdf (дата обращения: 01.12.2023). – Текст: электронный.
        \bibitem{litlink9} ГОСТ 19.404-79. Единая система программной документации. Термины и определения: утвержден и введен в действие Постановлением Государственного комитета стандартов Совета Министров СССР от 20 мая 1977 г. № 1268 срок введения: с 01.01.1980 г. – URL: https://www.swrit.ru/doc/espd/19.404-79.pdf (дата обращения: 01.12.2023). – Текст: электронный.

        \bibitem{litlink10} \textit{Учебный офис ФКН ПИ} (2023) СПРАВОЧНИК УЧЕБНОГО ПРОЦЕССА НИУ ВШЭ. Выпускная квалификационная работа (ВКР) // Сайт hse.ru (https://www.hse.ru/studyspravka/vkr) Просмотрено: 30.11.2023.
        \bibitem{litlink11} \textit{Жернова Мария Олеговна} (2023) Учебные планы 2020 года набора // Сайт hse.ru (https://www.hse.ru/ba/se/learn\_plans) Просмотрено: 12.12.2023.

        \bibitem{litlink12} \textit{Robert Axelrod} (1976) Structure of Decision: The Cognitive Maps of Political Elites // Сайт jstor.org (https://www.jstor.org/stable/j.ctt13x0vw3) Просмотрено: 17 января 2024.
        \bibitem{litlink13} \textit{Bart Kosko} (1985) Fuzzy cognitive maps // Сайт sipi.usc.edu (http://sipi.usc.edu/~kosko/FCM.pdf) Просмотрено: 17 января 2024.
        \bibitem{litlink14} \textit{Papageorgiou, Elpiniki \& Papageorgiou, Konstantinos \& Dikopoulou, Zoumpoulia \& Mourhir, Asmaa} (2018) A Fuzzy Cognitive Map web-based tool for modeling and decision making // Сайт researchgate.net (https://www.researchgate.net/publication/336591466\_A\_Fuzzy\_Cognitive\_Map\_web-based\_tool\_for\_modeling\_and\_decision\_making) Просмотрено: 17.01.2024.
        \bibitem{litlink15} \textit{Felix Benjamín, Gerardo \& Nápoles, Gonzalo \& Falcon, Rafael \& Froelich, Wojciech \& Vanhoof, Koen \& Bello, Rafael} (2019) A Review on Methods and Software for Fuzzy Cognitive Maps. Artificial Intelligence Review. // Сайт researchgate.net (https://www.researchgate.net/publication/319167451\_A\_Review\_on\_Methods\_and\_Software\_for\_Fuzzy\_Cognitive\_Maps/citation/download) Просмотрено: 17 января 2024.
        \bibitem{litlink16} \textit{Pete Barbrook-Johnson \& Alexandra S. Penn} (2022) Fuzzy Cognitive Mapping // Сайт link.springer.com (https://link.springer.com/chapter/10.1007/978-3-031-01919-7\_6) Просмотрено: 17 января 2024.
        \bibitem{litlink17} \textit{Glykas, Michael} (2010) Fuzzy cognitive maps. Advances in theory, methodologies, tools and applications // Сайт researchgate.net (https://www.researchgate.net/publication/268170676\_Fuzzy\_cognitive\_maps\_Advances\_in\_theory\_methodologies\_tools\_and\_applications) Просмотрено: 17 января 2024.
        \bibitem{litlink18} \textit{Luis Rodriguez-Repiso, Rossitza Setchi, Jose L. Salmeron} (2007) Modelling IT projects success with Fuzzy Cognitive Maps // Сайт sciencedirect.com (https://doi.org/10.1016/j.eswa.2006.01.032) Просмотрено: 17 января 2024.
        \bibitem{litlink19} \textit{Atasoy, Güzide} (2007) Using cognitive maps for modeling project success // Сайт open.metu.edu.tr (https://open.metu.edu.tr/handle/11511/16910) Просмотрено: 17 января 2024.
        \bibitem{litlink20} \textit{Bhutani, K., Kumar, M., Garg, G., \& Aggarwal, S.} (2016). Assessing it projects success with extended fuzzy cognitive maps \& neutrosophic cognitive maps in comparison to fuzzy cognitive maps. Neutrosophic Sets and Systems, 12(1), 9-19.
        \bibitem{litlink21} \textit{L.A. Zadeh} (1965) Fuzzy sets // Сайт www.sciencedirect.com (https://www.sciencedirect.com/science/article/pii/S001999586590241X) Просмотрено: 16 февраля 2024.
        \bibitem{litlink22} \textit{G. M. Mendez, Ismael Lopez-Juarez, P. N. Montes-Dorantes, M. A. Garcia} (2023) A New Method for the Design of Interval Type-3 Fuzzy Logic Systems With Uncertain Type-2 Non-Singleton Inputs (IT3 NSFLS-2): A Case Study in a Hot Strip Mill // Сайт ieeexplore.ieee.org (https://ieeexplore.ieee.org/document/10114383) Просмотрено: 16 февраля 2024.
    \end{thebibliography}
    \newpage
    \begin{center}
        \addcontentsline{toc}{section}{Приложения}
        \section*{Приложения}
    \end{center}
    \zz{}\textbf{Приложение 1\\}
    Ссылка на репозиторий проекта с исходным кодом и всеми использованными материалами.\\
    https://github.com/NikPeg/modeling\_perception\_success\_factors\\
    \zz{}\textbf{Приложение 2\\}
    Ссылка на проект интерфейса в сервисе Figma, отражающий примерную структуру будущего приложения.\\
    https://www.figma.com/file/PL5iRCOK6h7RpPK1ZqKQgE/modeling\_perception\_success\_factors?type=design\&node-id=0\%3A1&mode=design&t=p9Rw1aMudymyfiVe-1\\
    \zz{}\textbf{Приложение 3\\}
    \zz{}\textbf{Терминология\\}
    \begin{enumerate}
        \item \textbf{Информационные технологии (IT)}: Термин используется для обозначения комплекса технологий, связанных с созданием, хранением, обработкой и передачей информации с помощью компьютеров и компьютерных сетей.
        \item \textbf{Когнитивные карты}: Психологический инструмент, используемый для представления знаний, представлений и восприятий. Применяются в моделировании сложных систем и проблем.
        \item \textbf{Нечеткие когнитивные карты (Fuzzy Cognitive Maps, FCM)}: Расширение обычных когнитивных карт, позволяющее представить информацию об отношениях между элементами системы в виде нечетких значений.
        \item \textbf{IT-проект}: Проект, связанный с разработкой, внедрением или поддержкой информационных систем или технологий.
        \item \textbf{Моделирование}: Процесс создания модели - упрощенного представления реального объекта или процесса с целью его исследования и оптимизации.
        \item \textbf{Факторы успеха}: Элементы или условия, которые способствуют успешной реализации проекта.
        \item \textbf{Методы анализа}: Статистические и математические инструменты, используемые для изучения и распределения данных.
        \item \textbf{Алгоритмы}: Указания или набор правил, которые следует выполнить в определенном порядке для достижения конкретного результата.
        \item \textbf{Прогнозирование}: Использование статистических и математических методов для предсказания будущих показателей на основе определенного набора данных.
        \item \textbf{Данные о проекте}: Информация, собранная в процессе выполнения проекта, которая используется для анализа и прогнозирования.
        \item \textbf{Риск-менеджмент}: Процесс, включающий идентификацию, оценку и приоритизацию рисков (определенные как комбинации их вероятности и последствий) и последующую координацию и экономическую эффективность использования ресурсов для контроля вероятности и/или влияния неприемлемых событий.
    \end{enumerate}
\end{document}