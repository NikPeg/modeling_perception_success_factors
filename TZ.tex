\documentclass{article}
\usepackage{cmap}
\usepackage[T1,T2A]{fontenc}
\usepackage[utf8]{inputenc}
\usepackage[russian]{babel}
\usepackage[left=2cm,right=2cm,top=2cm,bottom=2cm,bindingoffset=0cm]{geometry}
\usepackage{tikz}
\usepackage{setspace,amsmath}
\usepackage{tabularx}
\usepackage{multirow}
\usepackage{makecell}
\usepackage{listings}
\usepackage{titlesec}
\usepackage{lipsum}
\usepackage[usestackEOL]{stackengine}
\usepackage{kantlipsum}
\usepackage{caption}
\usepackage{float}
\usepackage{zref-totpages}
\usepackage{fancyhdr}
\usepackage{graphicx}
\pagestyle{fancy}
\fancyhf{}
\fancyhead[C]{\thepage\\ RU.17701729.10.03-01 ПЗ 01-1}
\renewcommand{\headrulewidth}{0pt}
\captionsetup[table]{justification=centering}
\usetikzlibrary{positioning}
\graphicspath{{pictures/}}
\DeclareGraphicsExtensions{.pdf,.png,.jpg}
\newcommand\zz[1]{\par{\normalsize\strut #1} \hfill\ignorespaces}
\addto\captionsrussian{\def\refname{}}
\newcommand{\subtitle}[1]{%
    \posttitle{%
        \par\end{center}
        \begin{center}\Large#1\end{center}
    }%
}
\newcommand{\subsubtitle}[1]{%
    \preauthor{%
        \begin{center}
        \large #1 \vskip0.5em
        \begin{tabular}[t]{c}
    }%
}
\begin{document}
    \thispagestyle{empty}
    \begin{center}
        \textbf{
            ПРАВИТЕЛЬСТВО РОССИЙСКОЙ ФЕДЕРАЦИИ\\
            НАЦИОНАЛЬНЫЙ ИССЛЕДОВАТЕЛЬСКИЙ УНИВЕРСИТЕТ\\
            «ВЫСШАЯ ШКОЛА ЭКОНОМИКИ»\\
            Факультет компьютерных наук\\
            Образовательная программа «Программная инженерия»\\
            (ВШЭ ФКН ПИ)}\\
    \end{center}
    \bigskip
    \zz{СОГЛАСОВАНО}УТВЕРЖДАЮ
    \zz{Доцент департамента}Академический руководитель
    \zz{Программной инженерии,}образовательной программы
    \zz{ФКН, к.т.н.}«Программная инженерия»
    \zz{\noindent\rule{3cm}{0.4pt} К. Ю. Дегтярёв}старший преподаватель
    \zz{«\noindent\rule{1cm}{0.4pt}»\noindent\rule{2cm}{0.4pt}20\noindent\rule{0.5cm}{0.4pt}г.}\noindent\rule{3cm}{0.4pt} Н. А. Павлочев
    \zz{}«\noindent\rule{1cm}{0.4pt}»\noindent\rule{2cm}{0.4pt}20\noindent\rule{0.5cm}{0.4pt}г.
    \begin{center}
        \topskip=0pt
        \vspace*{\fill}
        \textbf{ПРОГРАММА ДЛЯ МОДЕЛИРОВАНИЯ ВОСПРИЯТИЯ\\
        ФАКТОРОВ УСПЕХА IТ-ПРОЕКТА С ИСПОЛЬЗОВАНИЕМ\\
        НЕЧЕТКИХ КОГНИТИВНЫХ КАРТ\\
        ~\\
        ~\\
        Техническое задание\\
        ~\\
        ЛИСТ УТВЕРЖДЕНИЯ\\
        ~\\
        RU.17701729.10.03-01 ПЗ 01-1-ЛУ}\\
        \vspace*{\fill}
    \end{center}
    \zz{~}Исполнитель
    \zz{~}Студент группы БПИ204
    \zz{~}образовательной программы
    \zz{~}«Программная инженерия»
    \zz{~}Пеганов Никита Сергеевич
    \zz{~}\noindent\rule{3cm}{0.4pt} Н. С. Пеганов
    \zz{~}«\noindent\rule{1cm}{0.4pt}»\noindent\rule{2cm}{0.4pt}20\noindent\rule{0.5cm}{0.4pt}г.
    \begin{center}
        \vspace*{\fill}{
            Москва \the\year{}}
    \end{center}
    \newpage
    \clearpage
    \pagenumbering{arabic}
    \begin{textbf}
        \\
        УТВЕРЖДЕН\\
        RU.17701729.10.03-01 ПЗ 01-1-ЛУ\\
    \end{textbf}
    \bigskip
    \begin{center}
        \topskip=0pt
        \vspace*{\fill}
        \textbf{ПРОГРАММА ДЛЯ МОДЕЛИРОВАНИЯ ВОСПРИЯТИЯ\\
        ФАКТОРОВ УСПЕХА IТ-ПРОЕКТА С ИСПОЛЬЗОВАНИЕМ\\
        НЕЧЕТКИХ КОГНИТИВНЫХ КАРТ\\
        ~\\
        ~\\
        Техническое задание\\
        ~\\
        RU.17701729.10.03-01 ПЗ 01-1-ЛУ}\\
        ~\\
        Листов \ztotpages\\
        \vspace*{\fill}
    \end{center}
    \begin{center}
        \vspace*{\fill}{
            Москва \the\year{}}
    \end{center}
    \newpage
    \tableofcontents
    \newpage
    \section {Аннотация}
    В представленной пояснительной записке описывается работа программы "{}IT-success-factors-model.exe"{}, которая используется для моделирования восприятия факторов успеха IT-проекта с использованием метода нечетких когнитивных карт. Задачей данной программы является обеспечение возможности визуализации, анализа и понимания динамики развития IT-проектов посредством моделирования взаимного влияния ключевых факторов их успешности.\\
    ~\\
    Основные требования к содержанию и оформлению данной пояснительной записки разработаны в соответствии с:
    \begin{itemize}
        \item ГОСТ 19.101-77 Виды программ и программных документов \cite{litlink1};
        \item ГОСТ 19.102-77 Стадии разработки \cite{litlink2};
        \item ГОСТ 19.103-77 Обозначения программ и программных документов \cite{litlink3};
        \item ГОСТ 19.104-78 Основные надписи \cite{litlink4};
        \item ГОСТ 19.105-78 Общие требования к программным документам \cite{litlink5};
        \item ГОСТ 19.106-78 Требования к программным документам, выполненным печатным
        способом \cite{litlink6};
        \item ГОСТ 19.404-79 Пояснительная записка. Требования к содержанию и оформлению \cite{litlink7}.
    \end{itemize}
    Изменения к данной пояснительной записке оформляются согласно ГОСТ 19.603-78 \cite{litlink8}, ГОСТ 19.604-78 \cite{litlink9}.
    \newpage
    \section {Введение}
    \subsection {Наименование программы на русском языке}
    Программа для моделирования восприятия факторов успеха IТ-проекта с использованием нечетких когнитивных карт.
    \subsection {Наименование программы на английском языке}
    A Program for Modeling the Perception of Success Factors of an IT-Project Using Fuzzy Cognitive Maps.
    \subsection {Документы, на основании которых ведется разработка}
    Программа разработана в рамках выполнения выпускной квалификационной работы — "{}Программа для моделирования восприятия факторов успеха IТ-проекта с использованием нечетких когнитивных карт"{}, в соответствии с учебным планом 4 курса бакалавриата направления 09.03.04 «Программная инженерия» \cite{litlink10}.\\
    ~\\
    Основание для разработки — учебный план подготовки бакалавров по направлению 09.03.04 «Программная инженерия» \cite{litlink11} и утвержденная академическим руководителем программы тема дипломной работы.
    \newpage
    \section {Назначение и область применения}
    \subsection {Назначение программы}
    Общим назначением разрабатываемой программы является визуализация, анализ и понимание факторов успеха IT-проектов. Это достигается путем использования нечетких когнитивных карт, что позволяет включить в модель любые переменные (факторы), даже те, которые сложно или невозможно измерить в количественных терминах. Основное назначение этого ПО — определение и визуализация взаимосвязей между различными факторами с точки зрения стейкхолдеров.\\
    ~\\
    Программа реализует нечеткие модели вычислений, с помощью которых аналитики могут оценивать и анализировать полученные данные, опираясь на предложенные нечеткие модели вычислений. Нечеткие когнитивные карты (Fuzzy Cognitive Maps, FCM) дают возможность моделировать одну и ту же систему по-разному, в зависимости от целей и профессиональных навыков людей или групп людей, фиксируя изменяемые во времени величины моделируемой ситуации.\\
    ~\\
    Программа генерирует FCM, которые можно использовать для визуализации сложных систем и отображения их развития во времени. При этом, в ряде случаев, применяется SWOT-анализ — это позволяет более полно охарактеризовать исследуемые факторы.\\
    ~\\
    С течением времени, могут меняться не только сами факторы, но и связи между ними. Программа позволяет учесть это, перестраивая и модифицируя карты. Это обеспечивает возможность итеративной корректировки модели и поиск новых зависимостей и уязвимостей.\\
    \subsection {Целевая аудитория продукта}
    Целевой аудиторией данной программы являются, преимущественно, специалисты, работающие в IT-секторе, а именно аналитики, менеджеры проектов и IT-директора. Это связано с тем, что программа позволяет моделировать восприятие факторов успеха IT-проектов и может быть полезной для исследования и управления различными аспектами таких проектов.\\
    ~\\
    Вместе с тем, данная программа может быть использована в обучающих целях и обладает потенциалом быть полезной для студентов и преподавателей IT-специальностей, особенно для тех, кто изучает или преподает курсы, связанные с управлением IT-проектами, анализом данных, искусственным интеллектом или когнитивной наукой.\\
    ~\\
    Наконец, потенциальными пользователями данной программы могут быть и авторы научных исследований из области IT и когнитивистики. Она может оказаться полезной при изучении восприятия факторов, влияющих на успех IT-проектов, и при исследовании механизмов принятия решений в рамках таких проектов.\\
    ~\\
    В то же время, следует отметить, что оперировать данной программой могут преимущественно люди, обладающие нужным навыком и знаниями для работы с нечеткими когнитивными картами. Подразумевается использование программы одним аналитиком и множеством стейкхолдеров для создания результата коллективного обсуждения.\\
    \subsection {Актуальность проблемы}
    В последнее время, в свете растущей зависимости бизнеса от технологий, успешное выполнение IT-проектов становится особенно важным для организаций различных сфер деятельности и масштаба. Однако, измерение и предсказание успеха в случае IT-проектов всё ещё являются сложной задачей, так как они зависят от множества факторов, характеризуемых неоднозначностью и взаимной связью с другими аспектами рассмотрения.\\
    ~\\
    В связи с этим, программа для моделирования восприятия факторов успеха IT-проектов с использованием нечетких когнитивных карт обретает значительную актуальность. Факторы успеха проекта часто являются нечетко определенными и интерпретируемыми, что делает использование нечетких когнитивных карт подходящим выбором для их анализа и моделирования.\\
    ~\\
    Методология когнитивного моделирования была предложена американским политологом и экономистом Робертом Аксельродом \cite{litlink12}. Когнитивное моделирование предназначалось для принятия решений в плохо определенных ситуациях. Нечеткие когнитивные карты, впервые предложенные Бартом Коско \cite{litlink13}, являются смешанным типом графического представления знаний, включающего в себя элементы когнитивных карт и нечеткой логики.\\
    ~\\
    В последние годы они вновь привлекают внимание исследователей, подобно тому как нейронные сети после своего «забвения» в 90-х годах 20-го века, сейчас снова переживают свой пик популярности. Например, нечеткие когнитивные карты используются в исследовательских работах, написанных в 2018, 2019 и 2022 годах \cite{litlink14, litlink15, litlink16}. Как и нейронные сети, нечеткие когнитивные карты могут быть применены для моделирования сложных взаимосвязей и получения результатов на основе неопределенной и нечеткой информации.\\
    ~\\
    В многочисленных исследовательских работах авторы рассматривают нечеткие когнитивные карты как удобный и наглядный аппарат моделирования. Факторы и связи между факторами располагаются в FCM в стуктуре, подобной структуре головного мозга человека (в очень упрощенном виде), поэтому получаемая модель легко воспринимается и удобна для обсуждения. Также нечеткие когнитивные карты отличаются универсальностью, что позволяет использовать их в множестве различных областей \cite{litlink17}.\\
    ~\\
    Несмотря на нейросетеподобную структуру нечетких когнитивных карт, в рамках данной выпускной квалификационной работы не предполагается использование сложных алгоритмов нейронных сетей. Главным образом, это обусловлено спецификой выбранной методологии — нечетких когнитивных карт. Данный подход предполагает создание модели с использованием сетевой структуры, раскрывающей прямые и обратные связи между различными факторами успеха IT-проекта.\\
    ~\\
    Моделирование факторов успеха IT-проекта — одна из областей, в которых успешно применяются нечеткие когнитивные карты. Например, в работе "{}Modelling IT projects success with Fuzzy Cognitive Maps"{} \cite{litlink18} авторы применяют FCM для моделирования факторов успеха мобильной платежной системы, проекта, связанного с быстро развивающимся миром мобильных телекоммуникаций. Описанная в работе методология использует четыре матрицы для представления результатов, которые методология обеспечивает на каждом из своих этапов. Это начальная матрица успеха (IMS), Фаззифицированная матрица успеха (FZMS), Матрица успеха силы отношений (SRMS) и Итоговая матрица успеха (FMS). Авторы статьи делают вывод, что Критические факторы успеха (CSF) — это те необходимые условия, которым должен удовлетворять проект, чтобы его воспринимали как успешный. Требуется улучшение процессов определения и оценки подходящих CSF для ИТ-проектов из-за возросшей сложности и неопределенности.\\
    ~\\
    В работе "{}Using cognitive maps for modeling project success"{} \cite{litlink19}, вышедшей в том же году, применяются когнитивные карты. Для наглядности в ней рассматривается реальный строительный проект, реализованный в Турции. В работе также описаны приемущества и недостатки когнитивных карт. Среди преимуществ когнитивных карт авторы отмечают их способность представлять сложные идеи и информацию в простой и понятной форме. Когнитивные карты также помогают улучшить понимание и организацию знаний, а также способствуют более эффективному принятию решений. Однако у когнитивных карт также есть недостатки. Они могут быть сложными для создания и интерпретации, особенно если они включают большое количество информации или сложные взаимосвязи. Кроме того, они могут быть субъективными, поскольку они основаны на знаниях и восприятии отдельного человека или группы людей.\\
    ~\\
    В статье "{}Assessing it projects success with extended fuzzy cognitive maps \& neutrosophic cognitive maps in comparison to fuzzy cognitive maps"{} \cite{litlink20} представлено исследование, в котором авторы сравнивают применение расширенных нечетких когнитивных карт и нейтрософских когнитивных карт в оценке успеха проекта мобильной платежной системы. Для этого они создали различные когнитивные карты с несколькими группами стейкхолдеров. В результате, авторы сделали вывод, что нейтрософские когнитивные карты показали лучшие результаты, чем нечеткие когнитивные карты и улучшенные когнитивные карты.\\
    ~\\
    Анализ литературы показывает, что использование когнитивных карт является эффективным инструментом для моделирования и оценки факторов успеха IT-проектов. Эти методы позволяют представить сложные идеи и информацию в простой и понятной форме, улучшить понимание и организацию знаний, а также способствуют более эффективному принятию решений.\\
    ~\\
    Однако, как отмечено в анализируемых работах, эти методы имеют свои недостатки, включая сложность создания и интерпретации карт, особенно при большом объеме информации и сложных взаимосвязях, а также субъективность, поскольку они основаны на знаниях и восприятии отдельного человека или группы людей.\\
    ~\\
    Также стоит отметить, что важность определения и оценки критических факторов успеха (CSF) для IT-проектов подчеркивается во всех рассмотренных работах. Это подтверждает актуальность нашего исследования и выбранной темы выпускной квалификационной работы.\\
    ~\\
    Таким образом, разработка и использование программы для моделирования восприятия факторов успеха IT-проектов с использованием нечетких когнитивных карт является полезным и актуальным подходом к решению сложной проблемы IT-управления и планирования.\\    \newpage
    \subsubsection {Функциональное назначение}
    Программа предназначена для моделирования восприятия различных факторов, которые способствуют или препятствуют успеху IT-проектов. Она использует принципы нечетких когнитивных карт для преобразования качественных оценок в количественные данные, что позволяет более точно анализировать и визуализировать динамику проекта.\\
    ~\\
    Основные функции программы включают:
    \begin{itemize}
        \item Создание списка факторов, влияющих на итоговый успех проекта. Эти факторы могут быть определены пользователем или группой пользователей, что обеспечивает гибкость системы и возможность учета уникальных особенностей каждой отдельной ситуации.
        \item Ввод данных о том, как каждый фактор влияет на другие, и преобразование этих данных в формальный вид с использованием нечетких множеств.
        \item Визуализация связей между факторами и расчет относительных значений в узлах, что позволяет увидеть, какой фактор является решающим и как он влияет на общую картину.
        \item Генерация выводов на основе анализа ситуации и формирование выводов о будущем ходе проекта.
        \item Запись и чтение созданных нечетких когнитивных карт в файл.
    \end{itemize}
    Таким образом, данная программа служит инструментом для анализа и улучшения процесса управления IT-проектами, позволяя более эффективно определять стратегии развития и принимать управленческие решения.
    \subsubsection {Эксплуатационное назначение}
    Эксплуатационное назначение разработанной программы заключается в моделировании восприятия влияния различных факторов на успех IT-проекта с использованием нечетких когнитивных карт.\\
    ~\\
    Программа предназначена для:
    \begin{itemize}
        \item Представления с каждым фактором определенной характеристики, которая помогает оператору понимать важность и актуальность данного фактора для проекта в целом;
        \item Возможности наблюдения и анализа весов связей между различными факторами, что позволяет оператору определить ключевые связи и элементы в структуре проекта;
        \item Предоставления информации о динамике изменения значений различных факторов во времени, что позволяет оператору реагировать на изменения во внешнем и внутреннем окружении проекта вовремя и принимать корректировки в стратегию проекта при необходимости;
        \item Выделения наиболее значимых факторов для успеха проекта, что позволяет сфокусироваться на ключевых элементах и не тратить ресурсы на менее важные аспекты;
        \item Получения дополнительной информации для лучшего понимания тех аспектов, которые в значительной или несущественной степени влияют на успех проекта.
    \end{itemize}
    Для каждого фактора учитывается гибкость настроек его влияния, относительные значения в узлах и влияние на другие факторы. Программа позволяет преобразовать оценки влияния с использованием нечеткой логики и формальных представлений нечетких множеств.\\
    ~\\
    Результатами работы программы становятся визуализированные когнитивные карты, на основе которых можно сделать выводы о наиболее важных моментах и факторах успеха IT-проекта.\\
    ~\\
    Одним из важных преимуществ программы является возможность имитации изменения взаимодействия факторов во времени, что позволяет проследить эволюцию проекта в долгосрочной перспективе.
    \subsection {Область применения программы}
    Программа для моделирования восприятия факторов успеха IT-проекта с использованием нечетких когнитивных карт предназначена для использования IT-специалистами, управляющими и исследователями в области управления информационными технологиями.\\
    ~\\
    Основные области применения программы включают:
    \begin{itemize}
        \item Разработку и управление IT-проектами. Моделирование факторов успеха проекта помогает управляющим эффективно управлять ресурсами и контролировать процесс реализации проекта;
        \item Исследование в области IT. Использование нечетких когнитивных карт позволяет формировать более точное и объективное представление об исследуемых объектах и процессах;
        \item Образование. Программа может быть использована для обучения студентов и участников профессиональных курсов основам управления IT-проектами и технологиям моделирования;
        \item Использование в комплексе с другими методами управления и предсказания для увеличения точности анализа и принятия решений.
    \end{itemize}
    Таким образом, данная программа может быть применена в различных областях связанных с IT-технологиями, включая научные исследования, обучение, планирование и управление IT-проектами.
    \newpage


    \section{Требования к программе или программному изделию}

    \subsection{Общие требования}
    Программа должна предоставлять функционал для создания, модификации и анализа нечетких когнитивных карт, которые моделируют влияние различных факторов на успех IT-проектов. Программа должна быть реализована как настольное приложение с графическим пользовательским интерфейсом.

    \subsection{Функциональные требования}
    \begin{enumerate}
        \item Программа должна предоставлять механизмы для создания моделей влияния факторов успеха IT-проекта, используя метод нечетких когнитивных карт.
        \item Оператор должен иметь возможность создавать индивидуальные модели, выбирая и модифицируя факторы влияния на успех IT-проекта.
        \item Программа должна предоставлять список наиболее часто встречающихся факторов IT-проекта, из которого оператор может выбирать элементы для включения в модель.
        \item Должен быть реализован функционал для модификации построенных моделей, включая добавление или удаление факторов, изменение взаимосвязей между факторами.
        \item Оператор должен иметь возможность сохранять построенные или измененные модели во внешний файл.
        \item Программа должна поддерживать сохранение различных версий когнитивной карты в процессе ее изменения.
        \item Должна быть предусмотрена функциональность для анализа информации о взаимосвязях факторов в различных вариантах моделей.
        \item Программа должна обеспечивать возможность ввода и анализа лингвистических термов и соответствующих им функций принадлежности.
        \item Должна быть реализована возможность замены числовых значений весов в когнитивной карте на формальное представление лингвистических термов.
        \item Программа должна предоставлять аналитические инструменты для изучения ключевой информации, связанной с когнитивной картой, для выводов о значимости и влиянии конкретных факторов на успешность IT-проекта.
    \end{enumerate}

    \subsection{Нефункциональные требования}
    \begin{enumerate}
        \item Интуитивно понятный графический интерфейс пользователя.
        \item Высокая надежность сохранения и восстановления данных.
        \item Поддержка работы на основных операционных системах, таких как Windows, macOS и Linux.
        \item Обеспечение безопасности данных пользователя.
        \item Документация пользователя и техническая поддержка.
    \end{enumerate}

    \subsection{Интерфейс}

    \subsubsection{Главная страница}
    Главная страница программы для моделирования восприятия факторов успеха IT-проекта с использованием нечетких когнитивных карт должна быть интуитивно понятной и удобной для пользователя. Основные элементы главной страницы включают:
    \begin{enumerate}
        \item \textbf{Логотип программы}: Расположен в верхней части страницы, является также ссылкой на главную страницу.
        \item \textbf{Информация о программе}: Краткое описание функциональности и предназначения программы.
        \item \textbf{Навигационное меню}: Содержит ссылки на основные разделы:
        \item \textbf{О программе} – страница с подробным описанием функций и возможностей программы.
        \item \textbf{Контакты} – информация для связи с разработчиками и техподдержкой.
        \item \textbf{Окно регистрации}: Позволяет новым пользователям создать аккаунт, введя e-mail и пароль.
        \item \textbf{Окно входа}: Для зарегистрированных пользователей, требует ввода e-mail и пароля для доступа к функционалу программы.
    \end{enumerate}

    \subsubsection{Страница проектов}
    На странице проектов пользователь может управлять своими текущими проектами и обращаться к шаблонам популярных IT-проектов:
    \begin{enumerate}
        \item \textbf{Список проектов пользователя}: Отображает все проекты, созданные пользователем, с возможностью их редактирования или удаления.
        \item \textbf{Шаблоны популярных IT-проектов}: Предоставляет базу готовых шаблонов, которые можно использовать как основу для новых проектов.
        \item \textbf{Кнопка создания нового проекта}: Позволяет начать работу над новым проектом с нуля или на основе выбранного шаблона.
    \end{enumerate}

    \subsubsection{Рабочая область}
    Рабочая область предназначена для моделирования и анализа факторов успеха IT-проектов:
    \begin{enumerate}
        \item \textbf{Панель инструментов}: Содержит инструменты для добавления, удаления и редактирования факторов и связей между ними.
        \item \textbf{Область моделирования}: Визуальное представление нечеткой когнитивной карты, где пользователь может взаимодействовать с элементами карты (факторами и связями).
        \item \textbf{Панель свойств}: Позволяет просматривать и редактировать свойства выбранного элемента на карте, такие как вес связи, значение фактора и т.д.
        \item \textbf{Функции анализа}: Инструменты для анализа построенной карты, включая расчеты влияния факторов и возможные сценарии развития проекта.
    \end{enumerate}

    \subsubsection{Дополнительные функции}
    \begin{enumerate}
        \item \textbf{Сохранение и экспорт}: Возможность сохранять текущий проект в файл или экспортировать его в различные форматы для дальнейшего использования или представления.
        \item \textbf{Помощь и поддержка}: Раздел с руководствами пользователя и часто задаваемыми вопросами, а также контактами службы поддержки для обращения при возникновении проблем.
    \end{enumerate}
    Этот интерфейс должен обеспечивать удобное и эффективное взаимодействие пользователя с программой, способствуя точному и глубокому анализу факторов успеха IT-проектов.
    \newpage
    \section {Требования к программной документации}
    Для обеспечения качественной разработки, сопровождения и эксплуатации программы, а также для удовлетворения потребностей пользователей и разработчиков, необходимо подготовить следующую программную документацию в соответствии с ГОСТ 19.101-77 и ГОСТ 34.602-89:

    \begin{enumerate}
        \item \textbf{Техническое задание} - документ, определяющий цели создания программы, основные требования к программе, условия эксплуатации, требования к программной документации, состав и содержание работ по созданию программы, порядок контроля и приемки.
        \item \textbf{Программа и методика испытаний} - документ, содержащий требования к испытаниям программы, порядок их проведения, а также критерии приемки программы.
        \item \textbf{Руководство оператора} - документ, предназначенный для обучения и поддержки пользователя в процессе работы с программой. Должно включать инструкции по установке, настройке и использованию программы, а также по устранению возможных ошибок.
        \item \textbf{Текст программы} - документ, содержащий исходный код программы с комментариями.
    \end{enumerate}

    Документация должна быть выполнена на русском языке и предоставлена в электронном виде в форматах DOCX и PDF. Каждый документ должен быть утвержден заказчиком и разработчиком программы до начала её эксплуатации.
    \newpage
    \section {Технико-экономические показатели}

    \subsection{Цели и задачи проекта}
    Программа предназначена для анализа и моделирования факторов, влияющих на успех IT-проектов, с использованием метода нечетких когнитивных карт. Основные задачи включают в себя создание, модификацию и анализ когнитивных карт, а также управление лингвистическими переменными и их функциями принадлежности.

    \subsection{Ожидаемые результаты}
    Разработка программного обеспечения позволит стейкхолдерам эффективно анализировать IT-проекты, опираясь на моделирование взаимосвязей между ключевыми факторами успеха.

    \subsection{Ресурсы проекта}
    \begin{itemize}
        \item Человеческие ресурсы: 1 разработчик.
        \item Технические ресурсы: серверы для разработки и тестирования, рабочий компьютер, лицензионное программное обеспечение.
        \item Виртуальные ресурсы: лицензированное программное обеспечение (IDE IntelliJ IDEA).
        \item Временные ресурсы: 9 месяцев (сентябрь 2023 года — июнь 2024 года).
    \end{itemize}
    \newpage
    \section {Стадии и этапы разработки}
    \begin{itemize}
        \item Сентябрь 2023 — обсуждение потенциальных тем выпускной квалификационной работы. Важно было определить актуальность темы, ее научную и практическую значимость, а также возможности для реализации.
        \item Ноябрь 2023 — выбор темы выпускной квалификационной работы. После обсуждения и анализа различных вариантов, был сделан окончательный выбор темы ВКР. Этот выбор был согласован с научным руководителем и утвержден кафедрой.
        \item Декабрь 2023 — изучение литературы по теме ВКР, создание первой версии пояснительной записки. Был проведен глубокий анализ научных источников, статей, книг и других материалов, связанных с выбранной темой. На основе полученных данных была начата работа над первой версией пояснительной записки, которая включает введение, обзор литературы и предварительное описание предполагаемых методов исследования.
        \item Январь 2024 — оформление пояснительной записки, исправление замечаний, выбор технических средств для разработки. В этот период дорабатывалась пояснительная записка, учитывались замечания и предложения научного руководителя. Также определились и выбераны необходимые технические средства и программное обеспечение для разработки и исследования.
        \item Февраль-март 2024 — разработка программного обеспечения. На этом этапе основное внимание было уделено разработке программного обеспечения или других технических аспектов работы. Это включает в себя программирование, настройку оборудования и первичные тесты функциональности.
        \item Апрель 2024 — преддипломная практика, создание технического задания. Во время преддипломной практики удалось применить теоретические знания на практике, а также получить дополнительные данные и материалы для работы. Также было создано техническое задание, которое станет основой для дальнейшей разработки и тестирования.
        \item Май 2024 — завершение разработки программного обеспечения, тестирование и приемка. Завершены все аспекты разработки и проведено комплексное тестирование программного обеспечения, подтверждено соответствие техническому заданию и требованиям.
        \item Июнь 2024 — защита ВКР. Подготовлены презентационные материалы и отчеты для защиты работы перед комиссией. Защита включает в себя демонстрацию результатов работы, ответы на вопросы экспертов и обсуждение полученных результатов и выводов.
    \end{itemize}
    \newpage
    \section {Порядок тестирования и приемки}

    \subsection{Тестирование}
    \begin{itemize}
        \item \textbf{Функциональное тестирование}: Проверка всех функций программы на соответствие требованиям.
        \item \textbf{Тестирование производительности}: Оценка скорости работы программы и ее способности обрабатывать необходимые объемы данных.
        \item \textbf{Тестирование удобства использования}: Оценка интерфейса и удобства работы с программой конечными пользователями.
        \item \textbf{Тестирование на соответствие}: Убедиться, что программа соответствует всем необходимым стандартам.
    \end{itemize}

    \subsection{Исправление ошибок}
    \begin{itemize}
        \item \textbf{Анализ результатов тестирования}: Сбор и анализ данных о найденных ошибках и проблемах.
        \item \textbf{Устранение ошибок}: Исправление обнаруженных недочетов и повторное тестирование для проверки исправлений.
    \end{itemize}

    \subsection{Приемка}
    \begin{itemize}
        \item \textbf{Предварительная приемка}: Демонстрация программы научному руководителю для предварительной оценки и получения обратной связи.
        \item \textbf{Исправление замечаний заказчика}: Внесение изменений по результатам обратной связи от научного руководителя.
        \item \textbf{Окончательная приемка}: Подписание акта о приемке программы заказчиком после окончательной проверки и удовлетворения всех требований.
    \end{itemize}

    \subsection{Поддержка и обслуживание}
    \begin{itemize}
        \item \textbf{Обучение пользователей}: Проведение тренингов и семинаров для конечных пользователей.
        \item \textbf{Техническая поддержка}: Организация службы поддержки для решения возникающих вопросов и проблем у пользователей.
        \item \textbf{Обновления и улучшения}: Регулярное обновление программного обеспечения для улучшения функциональности и безопасности.
    \end{itemize}

    Этот порядок контроля и приемки поможет обеспечить высокое качество и соответствие программы для моделирования восприятия факторов успеха IT-проекта с использованием нечетких когнитивных карт требованиям и ожиданиям заказчика.

    \newpage


    \section{Список использованных источников}
    \begin{thebibliography}{}
        \bibitem{litlink1} ГОСТ 19.101-77. Единая система программной документации. Термины и определения: утвержден и введен в действие Постановлением Государственного комитета стандартов Совета Министров СССР от 20 мая 1977 г. № 1268 срок введения: с 01.01.1980 г. – URL: https://www.swrit.ru/doc/espd/19.001-77.pdf (дата обращения: 01.12.2023). – Текст: электронный.
        \bibitem{litlink2} ГОСТ 19.102-77. Единая система программной документации. Термины и определения: утвержден и введен в действие Постановлением Государственного комитета стандартов Совета Министров СССР от 20 мая 1977 г. № 1268 срок введения: с 01.01.1980 г. – URL: https://www.swrit.ru/doc/espd/19.102-77.pdf (дата обращения: 01.12.2023). – Текст: электронный.
        \bibitem{litlink3} 19.103-77. Единая система программной документации. Термины и определения: утвержден и введен в действие Постановлением Государственного комитета стандартов Совета Министров СССР от 20 мая 1977 г. № 1268 срок введения: с 01.01.1980 г. – URL: https://www.swrit.ru/doc/espd/19.103-77.pdf (дата обращения: 01.12.2023). – Текст: электронный.
        \bibitem{litlink4} ГОСТ 19.104-78. Единая система программной документации. Термины и определения: утвержден и введен в действие Постановлением Государственного комитета стандартов Совета Министров СССР от 20 мая 1977 г. № 1268 срок введения: с 01.01.1980 г. – URL: https://www.swrit.ru/doc/espd/19.104-78.pdf (дата обращения: 01.12.2023). – Текст: электронный.
        \bibitem{litlink5} ГОСТ 19.105-78. Единая система программной документации. Термины и определения: утвержден и введен в действие Постановлением Государственного комитета стандартов Совета Министров СССР от 20 мая 1977 г. № 1268 срок введения: с 01.01.1980 г. – URL: https://www.swrit.ru/doc/espd/19.105-78.pdf (дата обращения: 01.12.2023). – Текст: электронный.
        \bibitem{litlink6} ГОСТ 19.106-78. Единая система программной документации. Термины и определения: утвержден и введен в действие Постановлением Государственного комитета стандартов Совета Министров СССР от 20 мая 1977 г. № 1268 срок введения: с 01.01.1980 г. – URL: https://www.swrit.ru/doc/espd/19.106-78.pdf (дата обращения: 01.12.2023). – Текст: электронный.
        \bibitem{litlink7} ГОСТ 19.404-79. Единая система программной документации. Термины и определения: утвержден и введен в действие Постановлением Государственного комитета стандартов Совета Министров СССР от 20 мая 1977 г. № 1268 срок введения: с 01.01.1980 г. – URL: https://www.swrit.ru/doc/espd/19.404-79.pdf (дата обращения: 01.12.2023). – Текст: электронный.
        \bibitem{litlink8} ГОСТ 19.603-78. Единая система программной документации. Термины и определения: утвержден и введен в действие Постановлением Государственного комитета стандартов Совета Министров СССР от 20 мая 1977 г. № 1268 срок введения: с 01.01.1980 г. – URL: https://www.swrit.ru/doc/espd/19.603-78.pdf (дата обращения: 01.12.2023). – Текст: электронный.
        \bibitem{litlink9} ГОСТ 19.404-79. Единая система программной документации. Термины и определения: утвержден и введен в действие Постановлением Государственного комитета стандартов Совета Министров СССР от 20 мая 1977 г. № 1268 срок введения: с 01.01.1980 г. – URL: https://www.swrit.ru/doc/espd/19.404-79.pdf (дата обращения: 01.12.2023). – Текст: электронный.

        \bibitem{litlink10} \textit{Учебный офис ФКН ПИ} (2023) СПРАВОЧНИК УЧЕБНОГО ПРОЦЕССА НИУ ВШЭ. Выпускная квалификационная работа (ВКР) // Сайт hse.ru (https://www.hse.ru/studyspravka/vkr) Просмотрено: 30.11.2023.
        \bibitem{litlink11} \textit{Жернова Мария Олеговна} (2023) Учебные планы 2020 года набора // Сайт hse.ru (https://www.hse.ru/ba/se/learn\_plans) Просмотрено: 12.12.2023.

        \bibitem{litlink12} \textit{Robert Axelrod} (1976) Structure of Decision: The Cognitive Maps of Political Elites // Сайт jstor.org (https://www.jstor.org/stable/j.ctt13x0vw3) Просмотрено: 17 января 2024.
        \bibitem{litlink13} \textit{Bart Kosko} (1985) Fuzzy cognitive maps // Сайт sipi.usc.edu (http://sipi.usc.edu/~kosko/FCM.pdf) Просмотрено: 17 января 2024.
        \bibitem{litlink14} \textit{Papageorgiou, Elpiniki \& Papageorgiou, Konstantinos \& Dikopoulou, Zoumpoulia \& Mourhir, Asmaa} (2018) A Fuzzy Cognitive Map web-based tool for modeling and decision making // Сайт researchgate.net (https://www.researchgate.net/publication/336591466\_A\_Fuzzy\_Cognitive\_Map\_web-based\_tool\_for\_modeling\_and\_decision\_making) Просмотрено: 17.01.2024.
        \bibitem{litlink15} \textit{Felix Benjamín, Gerardo \& Nápoles, Gonzalo \& Falcon, Rafael \& Froelich, Wojciech \& Vanhoof, Koen \& Bello, Rafael} (2019) A Review on Methods and Software for Fuzzy Cognitive Maps. Artificial Intelligence Review. // Сайт researchgate.net (https://www.researchgate.net/publication/319167451\_A\_Review\_on\_Methods\_and\_Software\_for\_Fuzzy\_Cognitive\_Maps/citation/download) Просмотрено: 17 января 2024.
        \bibitem{litlink16} \textit{Pete Barbrook-Johnson \& Alexandra S. Penn} (2022) Fuzzy Cognitive Mapping // Сайт link.springer.com (https://link.springer.com/chapter/10.1007/978-3-031-01919-7\_6) Просмотрено: 17 января 2024.
        \bibitem{litlink17} \textit{Glykas, Michael} (2010) Fuzzy cognitive maps. Advances in theory, methodologies, tools and applications // Сайт researchgate.net (https://www.researchgate.net/publication/268170676\_Fuzzy\_cognitive\_maps\_Advances\_in\_theory\_methodologies\_tools\_and\_applications) Просмотрено: 17 января 2024.
        \bibitem{litlink18} \textit{Luis Rodriguez-Repiso, Rossitza Setchi, Jose L. Salmeron} (2007) Modelling IT projects success with Fuzzy Cognitive Maps // Сайт sciencedirect.com (https://doi.org/10.1016/j.eswa.2006.01.032) Просмотрено: 17 января 2024.
        \bibitem{litlink19} \textit{Atasoy, Güzide} (2007) Using cognitive maps for modeling project success // Сайт open.metu.edu.tr (https://open.metu.edu.tr/handle/11511/16910) Просмотрено: 17 января 2024.
        \bibitem{litlink20} \textit{Bhutani, K., Kumar, M., Garg, G., \& Aggarwal, S.} (2016). Assessing it projects success with extended fuzzy cognitive maps \& neutrosophic cognitive maps in comparison to fuzzy cognitive maps. Neutrosophic Sets and Systems, 12(1), 9-19.
        \bibitem{litlink21} \textit{L.A. Zadeh} (1965) Fuzzy sets // Сайт www.sciencedirect.com (https://www.sciencedirect.com/science/article/pii/S001999586590241X) Просмотрено: 16 февраля 2024.
        \bibitem{litlink22} \textit{G. M. Mendez, Ismael Lopez-Juarez, P. N. Montes-Dorantes, M. A. Garcia} (2023) A New Method for the Design of Interval Type-3 Fuzzy Logic Systems With Uncertain Type-2 Non-Singleton Inputs (IT3 NSFLS-2): A Case Study in a Hot Strip Mill // Сайт ieeexplore.ieee.org (https://ieeexplore.ieee.org/document/10114383) Просмотрено: 16 февраля 2024.
    \end{thebibliography}
    \newpage
    \begin{center}
        \addcontentsline{toc}{section}{Приложения}
        \section*{Приложения}
    \end{center}
    \zz{}\textbf{Приложение 1\\}
    Ссылка на репозиторий проекта с исходным кодом и всеми использованными материалами.\\
    https://github.com/NikPeg/modeling\_perception\_success\_factors\\
    \zz{}\textbf{Приложение 2\\}
    Ссылка на проект интерфейса в сервисе Figma, отражающий примерную структуру будущего приложения.\\
    https://www.figma.com/file/PL5iRCOK6h7RpPK1ZqKQgE/modeling\_perception\_success\_factors?type=design\&node-id=0\%3A1&mode=design&t=p9Rw1aMudymyfiVe-1\\
    \zz{}\textbf{Приложение 3\\}
    \zz{}\textbf{Терминология\\}
    \begin{enumerate}
        \item \textbf{Информационные технологии (IT)}: Термин используется для обозначения комплекса технологий, связанных с созданием, хранением, обработкой и передачей информации с помощью компьютеров и компьютерных сетей.
        \item \textbf{Когнитивные карты}: Психологический инструмент, используемый для представления знаний, представлений и восприятий. Применяются в моделировании сложных систем и проблем.
        \item \textbf{Нечеткие когнитивные карты (Fuzzy Cognitive Maps, FCM)}: Расширение обычных когнитивных карт, позволяющее представить информацию об отношениях между элементами системы в виде нечетких значений.
        \item \textbf{IT-проект}: Проект, связанный с разработкой, внедрением или поддержкой информационных систем или технологий.
        \item \textbf{Моделирование}: Процесс создания модели - упрощенного представления реального объекта или процесса с целью его исследования и оптимизации.
        \item \textbf{Факторы успеха}: Элементы или условия, которые способствуют успешной реализации проекта.
        \item \textbf{Методы анализа}: Статистические и математические инструменты, используемые для изучения и распределения данных.
        \item \textbf{Алгоритмы}: Указания или набор правил, которые следует выполнить в определенном порядке для достижения конкретного результата.
        \item \textbf{Прогнозирование}: Использование статистических и математических методов для предсказания будущих показателей на основе определенного набора данных.
        \item \textbf{Данные о проекте}: Информация, собранная в процессе выполнения проекта, которая используется для анализа и прогнозирования.
        \item \textbf{Риск-менеджмент}: Процесс, включающий идентификацию, оценку и приоритизацию рисков (определенные как комбинации их вероятности и последствий) и последующую координацию и экономическую эффективность использования ресурсов для контроля вероятности и/или влияния неприемлемых событий.
    \end{enumerate}
\end{document}