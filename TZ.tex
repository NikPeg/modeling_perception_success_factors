\documentclass{article}
\usepackage{cmap}
\usepackage[T1,T2A]{fontenc}
\usepackage[utf8]{inputenc}
\usepackage[russian]{babel}
\usepackage[left=2cm,right=2cm,top=2cm,bottom=2cm,bindingoffset=0cm]{geometry}
\usepackage{tikz}
\usepackage{setspace,amsmath}
\usepackage{tabularx}
\usepackage{multirow}
\usepackage{makecell}
\usepackage{listings}
\usepackage{titlesec}
\usepackage{lipsum}
\usepackage[usestackEOL]{stackengine}
\usepackage{kantlipsum}
\usepackage{caption}
\usepackage{float}
\usepackage{zref-totpages}
\usepackage{fancyhdr}
\usepackage{graphicx}
\pagestyle{fancy}
\fancyhf{}
\fancyhead[C]{\thepage\\ RU.17701729.10.03-01 ПЗ 01-1}
\renewcommand{\headrulewidth}{0pt}
\captionsetup[table]{justification=centering}
\usetikzlibrary{positioning}
\graphicspath{{pictures/}}
\DeclareGraphicsExtensions{.pdf,.png,.jpg}
\newcommand\zz[1]{\par{\normalsize\strut #1} \hfill\ignorespaces}
\addto\captionsrussian{\def\refname{}}
\newcommand{\subtitle}[1]{%
    \posttitle{%
        \par\end{center}
        \begin{center}\Large#1\end{center}
    }%
}
\newcommand{\subsubtitle}[1]{%
    \preauthor{%
        \begin{center}
        \large #1 \vskip0.5em
        \begin{tabular}[t]{c}
    }%
}
\begin{document}
    \thispagestyle{empty}
    \begin{center}
        \textbf{
            ПРАВИТЕЛЬСТВО РОССИЙСКОЙ ФЕДЕРАЦИИ\\
            НАЦИОНАЛЬНЫЙ ИССЛЕДОВАТЕЛЬСКИЙ УНИВЕРСИТЕТ\\
            «ВЫСШАЯ ШКОЛА ЭКОНОМИКИ»\\
            Факультет компьютерных наук\\
            Образовательная программа «Программная инженерия»\\
            (ВШЭ ФКН ПИ)}\\
    \end{center}
    \bigskip
    \zz{СОГЛАСОВАНО}УТВЕРЖДАЮ
    \zz{Доцент департамента}Академический руководитель
    \zz{Программной инженерии,}образовательной программы
    \zz{ФКН, к.т.н.}«Программная инженерия»
    \zz{\noindent\rule{3cm}{0.4pt} К. Ю. Дегтярёв}старший преподаватель
    \zz{«\noindent\rule{1cm}{0.4pt}»\noindent\rule{2cm}{0.4pt}20\noindent\rule{0.5cm}{0.4pt}г.}\noindent\rule{3cm}{0.4pt} Н. А. Павлочев
    \zz{}«\noindent\rule{1cm}{0.4pt}»\noindent\rule{2cm}{0.4pt}20\noindent\rule{0.5cm}{0.4pt}г.
    \begin{center}
        \topskip=0pt
        \vspace*{\fill}
        \textbf{ПРОГРАММА ДЛЯ МОДЕЛИРОВАНИЯ ВОСПРИЯТИЯ\\
        ФАКТОРОВ УСПЕХА IТ-ПРОЕКТА С ИСПОЛЬЗОВАНИЕМ\\
        НЕЧЕТКИХ КОГНИТИВНЫХ КАРТ\\
        ~\\
        ~\\
        Пояснительная записка\\
        ~\\
        ЛИСТ УТВЕРЖДЕНИЯ\\
        ~\\
        RU.17701729.10.03-01 ПЗ 01-1-ЛУ}\\
        \vspace*{\fill}
    \end{center}
    \zz{~}Исполнитель
    \zz{~}Студент группы БПИ204
    \zz{~}образовательной программы
    \zz{~}«Программная инженерия»
    \zz{~}Пеганов Никита Сергеевич
    \zz{~}\noindent\rule{3cm}{0.4pt} Н. С. Пеганов
    \zz{~}«\noindent\rule{1cm}{0.4pt}»\noindent\rule{2cm}{0.4pt}20\noindent\rule{0.5cm}{0.4pt}г.
    \begin{center}
        \vspace*{\fill}{
            Москва \the\year{}}
    \end{center}
    \newpage
    \clearpage
    \pagenumbering{arabic}
    \begin{textbf}\\
    УТВЕРЖДЕН\\
    RU.17701729.10.03-01 ПЗ 01-1-ЛУ\\
    \end{textbf}
    \bigskip
    \begin{center}
        \topskip=0pt
        \vspace*{\fill}
        \textbf{ПРОГРАММА ДЛЯ МОДЕЛИРОВАНИЯ ВОСПРИЯТИЯ\\
        ФАКТОРОВ УСПЕХА IТ-ПРОЕКТА С ИСПОЛЬЗОВАНИЕМ\\
        НЕЧЕТКИХ КОГНИТИВНЫХ КАРТ\\
        ~\\
        ~\\
        Пояснительная записка\\
        ~\\
        RU.17701729.10.03-01 ПЗ 01-1-ЛУ}\\
        ~\\
        Листов \ztotpages\\
        \vspace*{\fill}
    \end{center}
    \begin{center}
        \vspace*{\fill}{
            Москва \the\year{}}
    \end{center}
    \newpage
    \tableofcontents
    \newpage
    \section {Аннотация}
    В представленной пояснительной записке описывается работа программы "{}IT-success-factors-model.exe"{}, которая используется для моделирования восприятия факторов успеха IT-проекта с использованием метода нечетких когнитивных карт. Задачей данной программы является обеспечение возможности визуализации, анализа и понимания динамики развития IT-проектов посредством моделирования взаимного влияния ключевых факторов их успешности.\\
    ~\\
    Основные требования к содержанию и оформлению данной пояснительной записки разработаны в соответствии с:
    \begin{itemize}
        \item ГОСТ 19.101-77 Виды программ и программных документов \cite{litlink1};
        \item ГОСТ 19.102-77 Стадии разработки \cite{litlink2};
        \item ГОСТ 19.103-77 Обозначения программ и программных документов \cite{litlink3};
        \item ГОСТ 19.104-78 Основные надписи \cite{litlink4};
        \item ГОСТ 19.105-78 Общие требования к программным документам \cite{litlink5};
        \item ГОСТ 19.106-78 Требования к программным документам, выполненным печатным
        способом \cite{litlink6};
        \item ГОСТ 19.404-79 Пояснительная записка. Требования к содержанию и оформлению \cite{litlink7}.
    \end{itemize}
    Изменения к данной пояснительной записке оформляются согласно ГОСТ 19.603-78 \cite{litlink8}, ГОСТ 19.604-78 \cite{litlink9}.
    \newpage
    \section {Введение}
    \subsection {Наименование программы на русском языке}
    Программа для моделирования восприятия факторов успеха IТ-проекта с использованием нечетких когнитивных карт.
    \subsection {Наименование программы на английском языке}
    A Program for Modeling the Perception of Success Factors of an IT-Project Using Fuzzy Cognitive Maps.
    \subsection {Документы, на основании которых ведется разработка}
    Программа разработана в рамках выполнения выпускной квалификационной работы — "{}Программа для моделирования восприятия факторов успеха IТ-проекта с использованием нечетких когнитивных карт"{}, в соответствии с учебным планом 4 курса бакалавриата направления 09.03.04  «Программная инженерия» \cite{litlink10}.\\
    ~\\
    Основание для разработки — учебный план подготовки бакалавров по направлению 09.03.04 «Программная инженерия» \cite{litlink11} и утвержденная академическим руководителем программы тема дипломной работы.
    \newpage
    \section {Назначение и область применения}
    \subsection {Назначение программы}
    Общим назначением разрабатываемой программы является визуализация, анализ и понимание факторов успеха IT-проектов. Это достигается путем использования нечетких когнитивных карт, что позволяет включить в модель любые переменные (факторы), даже те, которые сложно или невозможно измерить в количественных терминах.  Основное назначение этого ПО — определение и визуализация взаимосвязей между различными факторами с точки зрения стейкхолдеров.\\
    ~\\
    Программа реализует нечеткие модели вычислений, с помощью которых аналитики могут оценивать и анализировать полученные данные, опираясь на предложенные нечеткие модели вычислений. Нечеткие когнитивные карты (Fuzzy Cognitive Maps, FCM) дают возможность моделировать одну и ту же систему по-разному, в зависимости от целей и профессиональных навыков людей или групп людей, фиксируя изменяемые во времени величины моделируемой ситуации.\\
    ~\\
    Программа генерирует FCM, которые можно использовать для визуализации сложных систем и отображения их развития во времени. При этом, в ряде случаев, применяется SWOT-анализ — это позволяет более полно охарактеризовать исследуемые факторы.\\
    ~\\
    С течением времени, могут меняться не только сами факторы, но и связи между ними. Программа позволяет учесть это, перестраивая и модифицируя карты. Это обеспечивает возможность итеративной корректировки модели и поиск новых зависимостей и уязвимостей.\\
    \subsection {Целевая аудитория продукта}
    Целевой аудиторией данной программы являются, преимущественно, специалисты, работающие в IT-секторе, а именно аналитики, менеджеры проектов и IT-директора. Это связано с тем, что программа позволяет моделировать восприятие факторов успеха IT-проектов и может быть полезной для исследования и управления различными аспектами таких проектов.\\
    ~\\
    Вместе с тем, данная программа может быть использована в обучающих целях и обладает потенциалом быть полезной для студентов и преподавателей IT-специальностей, особенно для тех, кто изучает или преподает курсы, связанные с управлением IT-проектами, анализом данных, искусственным интеллектом или когнитивной наукой.\\
    ~\\
    Наконец, потенциальными пользователями данной программы могут быть и авторы научных исследований из области IT и когнитивистики. Она может оказаться полезной при изучении восприятия факторов, влияющих на успех IT-проектов, и при исследовании механизмов принятия решений в рамках таких проектов.\\
    ~\\
    В то же время, следует отметить, что оперировать данной программой могут преимущественно люди, обладающие нужным навыком и знаниями для работы с нечеткими когнитивными картами. Подразумевается использование программы одним аналитиком и множеством стейкхолдеров для создания результата коллективного обсуждения.\\
    \subsection {Актуальность проблемы}
    В последнее время, в свете растущей зависимости бизнеса от технологий, успешное выполнение IT-проектов становится особенно важным для организаций различных сфер деятельности и масштаба. Однако, измерение и предсказание успеха в случае IT-проектов всё ещё являются сложной задачей, так как они зависят от множества факторов, характеризуемых неоднозначностью и взаимной связью с другими аспектами рассмотрения.\\
    ~\\
    В связи с этим, программа для моделирования восприятия факторов успеха IT-проектов с использованием нечетких когнитивных карт обретает значительную актуальность. Факторы успеха проекта часто являются нечетко определенными и интерпретируемыми, что делает использование нечетких когнитивных карт подходящим выбором для их анализа и моделирования.\\
    ~\\
    Методология когнитивного моделирования была предложена американским политологом и экономистом Робертом Аксельродом \cite{litlink12}. Когнитивное моделирование предназначалось для принятия решений в плохо определенных ситуациях. Нечеткие когнитивные карты, впервые предложенные Бартом Коско \cite{litlink13}, являются смешанным типом графического представления знаний, включающего в себя элементы когнитивных карт и нечеткой логики.\\
    ~\\
    В последние годы они вновь привлекают внимание исследователей, подобно тому как нейронные сети после своего «забвения» в 90-х годах 20-го века, сейчас снова переживают свой пик популярности. Например, нечеткие когнитивные карты используются в исследовательских работах, написанных в 2018, 2019 и 2022 годах \cite{litlink14, litlink15, litlink16}. Как и нейронные сети, нечеткие когнитивные карты могут быть применены для моделирования сложных взаимосвязей и получения результатов на основе неопределенной и нечеткой информации.\\
    ~\\
    В многочисленных исследовательских работах авторы рассматривают нечеткие когнитивные карты как удобный и наглядный аппарат моделирования. Факторы и связи между факторами располагаются в FCM в стуктуре, подобной структуре головного мозга человека (в очень упрощенном виде), поэтому получаемая модель легко воспринимается и удобна для обсуждения. Также нечеткие когнитивные карты отличаются универсальностью, что позволяет использовать их в множестве различных областей \cite{litlink17}.\\
    ~\\
    Несмотря на нейросетеподобную структуру нечетких когнитивных карт, в рамках данной выпускной квалификационной работы не предполагается использование сложных алгоритмов нейронных сетей. Главным образом, это обусловлено спецификой выбранной методологии — нечетких когнитивных карт. Данный подход предполагает создание модели с использованием сетевой структуры, раскрывающей прямые и обратные связи между различными факторами успеха IT-проекта.\\
    ~\\
    Моделирование факторов успеха IT-проекта — одна из областей, в которых успешно применяются нечеткие когнитивные карты. Например, в работе "{}Modelling IT projects success with Fuzzy Cognitive Maps"{} \cite{litlink18} авторы применяют FCM для моделирования факторов успеха мобильной платежной системы, проекта, связанного с быстро развивающимся миром мобильных телекоммуникаций. Описанная в работе методология использует четыре матрицы для представления результатов, которые методология обеспечивает на каждом из своих этапов. Это начальная матрица успеха (IMS), Фаззифицированная матрица успеха (FZMS), Матрица успеха силы отношений (SRMS) и Итоговая матрица успеха (FMS). Авторы статьи делают вывод, что Критические факторы успеха (CSF) — это те необходимые условия, которым должен удовлетворять проект, чтобы его воспринимали как успешный. Требуется улучшение процессов определения и оценки подходящих CSF для ИТ-проектов из-за возросшей сложности и неопределенности.\\
    ~\\
    В работе "{}Using cognitive maps for modeling project success"{} \cite{litlink19}, вышедшей в том же году, применяются когнитивные карты. Для наглядности в ней рассматривается реальный строительный проект, реализованный в Турции. В работе также описаны приемущества и недостатки когнитивных карт. Среди преимуществ когнитивных карт авторы отмечают их способность представлять сложные идеи и информацию в простой и понятной форме. Когнитивные карты также помогают улучшить понимание и организацию знаний, а также способствуют более эффективному принятию решений. Однако у когнитивных карт также есть недостатки. Они могут быть сложными для создания и интерпретации, особенно если они включают большое количество информации или сложные взаимосвязи. Кроме того, они могут быть субъективными, поскольку они основаны на знаниях и восприятии отдельного человека или группы людей.\\
    ~\\
    В статье "{}Assessing it projects success with extended fuzzy cognitive maps \& neutrosophic cognitive maps in comparison to fuzzy cognitive maps"{} \cite{litlink20} представлено исследование, в котором авторы сравнивают применение расширенных нечетких когнитивных карт и нейтрософских когнитивных карт в оценке успеха проекта мобильной платежной системы. Для этого они создали различные когнитивные карты с несколькими группами стейкхолдеров. В результате, авторы сделали вывод, что нейтрософские когнитивные карты показали лучшие результаты, чем нечеткие когнитивные карты и улучшенные когнитивные карты.\\
    ~\\
    Анализ литературы показывает, что использование когнитивных карт является эффективным инструментом для моделирования и оценки факторов успеха IT-проектов. Эти методы позволяют представить сложные идеи и информацию в простой и понятной форме, улучшить понимание и организацию знаний, а также способствуют более эффективному принятию решений.\\
    ~\\
    Однако, как отмечено в анализируемых работах, эти методы имеют свои недостатки, включая сложность создания и интерпретации карт, особенно при большом объеме информации и сложных взаимосвязях, а также субъективность, поскольку они основаны на знаниях и восприятии отдельного человека или группы людей.\\
    ~\\
    Также стоит отметить, что важность определения и оценки критических факторов успеха (CSF) для IT-проектов подчеркивается во всех рассмотренных работах. Это подтверждает актуальность нашего исследования и выбранной темы выпускной квалификационной работы.\\
    ~\\
    Таким образом, разработка и использование программы для моделирования восприятия факторов успеха IT-проектов с использованием нечетких когнитивных карт является полезным и актуальным подходом к решению сложной проблемы IT-управления и планирования.\\
    \subsubsection {Функциональное назначение}
    Программа предназначена для моделирования восприятия различных факторов, которые способствуют или препятствуют успеху IT-проектов. Она использует принципы нечетких когнитивных карт для преобразования качественных оценок в количественные данные, что позволяет более точно анализировать и визуализировать динамику проекта.\\
    ~\\
    Основные функции программы включают:
    \begin{itemize}
        \item Создание списка факторов, влияющих на итоговый успех проекта. Эти факторы могут быть определены пользователем или группой пользователей, что обеспечивает гибкость системы и возможность учета уникальных особенностей каждой отдельной ситуации.
        \item Ввод данных о том, как каждый фактор влияет на другие, и преобразование этих данных в формальный вид с использованием нечетких множеств.
        \item Визуализация связей между факторами и расчет относительных значений в узлах, что позволяет увидеть, какой фактор является решающим и как он влияет на общую картину.
        \item Генерация выводов на основе анализа ситуации и формирование выводов о будущем ходе проекта.
        \item Запись и чтение созданных нечетких когнитивных карт в файл.
    \end{itemize}
    Таким образом, данная программа служит инструментом для анализа и улучшения процесса управления IT-проектами, позволяя более эффективно определять стратегии развития и принимать управленческие решения.
    \subsubsection {Эксплуатационное назначение}
    Эксплуатационное назначение разработанной программы заключается в моделировании восприятия влияния различных факторов на успех IT-проекта с использованием нечетких когнитивных карт.\\
    ~\\
    Программа предназначена для:
    \begin{itemize}
        \item Представления с каждым фактором определенной характеристики, которая помогает оператору понимать важность и актуальность данного фактора для проекта в целом;
        \item Возможности наблюдения и анализа весов связей между различными факторами, что позволяет оператору определить ключевые связи и элементы в структуре проекта;
        \item Предоставления информации о динамике изменения значений различных факторов во времени, что позволяет оператору реагировать на изменения во внешнем и внутреннем окружении проекта вовремя и принимать корректировки в стратегию проекта при необходимости;
        \item Выделения наиболее значимых факторов для успеха проекта, что позволяет сфокусироваться на ключевых элементах и не тратить ресурсы на менее важные аспекты;
        \item Получения дополнительной информации для лучшего понимания тех аспектов, которые в значительной или несущественной степени влияют на успех проекта.
    \end{itemize}
    Для каждого фактора учитывается гибкость настроек его влияния, относительные значения в узлах и влияние на другие факторы. Программа позволяет преобразовать оценки влияния с использованием нечеткой логики и формальных представлений нечетких множеств.\\
    ~\\
    Результатами работы программы становятся визуализированные когнитивные карты, на основе которых можно сделать выводы о наиболее важных моментах и факторах успеха IT-проекта.\\
    ~\\
    Одним из важных преимуществ программы является возможность имитации изменения взаимодействия факторов во времени, что позволяет проследить эволюцию проекта в долгосрочной перспективе.
    \subsection {Область применения программы}
    Программа для моделирования восприятия факторов успеха IT-проекта с использованием нечетких когнитивных карт предназначена для использования IT-специалистами, управляющими и исследователями в области управления информационными технологиями.\\
    ~\\
    Основные области применения программы включают:
    \begin{itemize}
        \item Разработку и управление IT-проектами. Моделирование факторов успеха проекта помогает управляющим эффективно управлять ресурсами и контролировать процесс реализации проекта;
        \item Исследование в области IT. Использование нечетких когнитивных карт позволяет формировать более точное и объективное представление об исследуемых объектах и процессах;
        \item Образование. Программа может быть использована для обучения студентов и участников профессиональных курсов основам управления IT-проектами и технологиям моделирования;
        \item Использование в комплексе с другими методами управления и предсказания для увеличения точности анализа и принятия решений.
    \end{itemize}
    Таким образом, данная программа может быть применена в различных областях связанных с IT-технологиями, включая научные исследования, обучение, планирование и управление IT-проектами.
    \newpage
    \section {Технические характеристики}
    \subsection {Постановка задачи на разработку программы}
    Разрабатываемая программа должна быть наделена следующими функциями:
    \begin{enumerate}
        \item  Обеспечивать оператору механизмы для создания моделей влияния факторов успеха IT-проекта, основываясь на предоставленных им данных и применяя метод нечетких когнитивных карт;
        \item  Давать возможность оператору создавать свои индивидуальные модели влияния действующих факторов на успех IT-проекта;
        \item  Позволять оператору выбирать факторы IT-проекта из списка существующих, наиболее часто встречающихся факторов;
        \item  Реализовывать функционал, позволяющий вносить изменения в построенные модели, в том числе добавлять или удалять факторы, изменять взаимосвязи между факторами и так далее;
        \item  Давать возможность оператору сохранять построенные или измененные модели во внешний файл;
        \item  Предоставлять возможность оператору сохранять разные версии когнитивной карты в процессе изменения;
        \item  Обеспечивать оператору возможность анализа информации о взаимосвязях факторов в различных вариантах;
        \item  Обеспечивать оператору возможность ввода и анализа лингвистических термов и соответствующих им функций принадлежности;
        \item  Давать возможность оператору заменять значения весов когнитивной карты на формальное представление лингвистических термов;
        \item  Предоставлять оператору возможность анализа ключевой информации, связанной с когнитивной картой, которая используется для выводов о значимости и влиянии конкретных факторов на успешность IT-проекта.
    \end{enumerate}
\end{document}